\chapter{The effect of thermal barrier coatings on wave propagation}\label{chapter:TBC}

In this section the effect of thermal barrier coatings on signal propagation is investigated through the generation of dispersion curves, and COMSOL simulation. 

Thermal barrier coatings are made up of multiple layers, a bond coat, a thermally grown oxide layer (TGO), and a top coat. Both the composition of TBC materials as well as the application technique has an effect on the material properties. Differences in air plasma spray (APS) application methods causes variances in porosity as well as bonding efficiency \cite{Beghini2001}. Electron-beam physical vapor deposition (EB-PVD) produces elongated intercolumnar pores perpendicular to the thickness, whereas air plasma spray methods exhibit inter-splat pores parallel to the TBC surface \cite{Akwaboa2010}. Both APS and EB-PVD produce orientation specific Young's modulus, where the cross-sectional $E$ is larger than the plan-section $E$. In general EB-PVD results in higher values of Young's modulus (20-120 GPa \cite{Lugscheider2001,Guo2006}) for YSZ top coats than APS (8.7-54.6 GPa \cite{Duan1998}). The Young's modulus of YSZ coatings shows a significant dependence on porosity and microstructure, where Young's modulus decreases with increasing porosity. The $E$ of ZrO\textsubscript{2}-4 mol\% Y\textsubscript{2}O\textsubscript{3} applied by EB-PVD has been shown to decrease from around 200 GPa at 8\% porosity, to 80 GPa at 28\% porosity~\cite{Jang2005}. A number of studies~\cite{Yamazaki2007, Nath2015} have shown that the Young's modulus of TBC materials increases after thermal aging, as fine microcracks and porosities are sealed. The increase is in the region of 30 GPa. Exposure to higher temperatures causes a larger increase in $E$.

Considering that YSZ exhibits substantially lower Young's modulus than the other materials there is likely to be a comparatively larger through-thickness displacement in the top coat, which will make propagating waves substantially more sensitive to defects in the TBC surface. This effect will be partially reduced as the value of $E$ increases after thermal aging. 

In a non-symmetric laminate such as this the modes do not have a clear symmetric or antisymmetric character, especially when the different layers have substantially different material properties, and therefore wave speeds. The modes are therefore denoted using $B$ (as proposed by \href{https://www.dlr.de/zlp/en/desktopdefault.aspx/tabid-14332/24874_read-61142/#/gallery/33485}{The Dispersion Calculator}~\cite{Huber}), rather than categorising them as either symmetric, $S$, or antisymmetric, $A$. 

An initial study has been carried out that considers a typical TBC configuration at room temperature, with a relatively high value (48 GPa) of Young's modulus for the YSZ top coat. Dispersion curves and through thickness displacement profiles have been generated to demonstrate the effect on wave propagation. 

A second study has been carried out using temperature dependant material properties for both the superalloy substrate (Inconel 718) and the TBC materials. In this study the values of Young's modulus for the YSZ top coat are lower than in the room temperature study, which is likely to cause an even larger shift in displacement for the top coat layer. A COMSOL model has been used to simulate wave propagation at increasing temperatures, to investigate the suitability of using the $B_1$ mode for temperature monitoring. 

\section{Fixed temperature study}

A typical configuration is considered in Table~\ref{table:TBCmatprops}, where the values for Young's modulus and Poisson's ratio are provided by Li~\cite{Li2017b}. Table~\ref{table:TBCwavespeeds} shows the longitudinal and shear wave velocities of each TBC material used in this study, as calculated by COMSOL from the material properties used in Table~\ref{table:TBCmatprops}. 

The energy and phase velocity dispersion curves shown in Figures~\ref{fig:inconeldispTBCgroup} and~\ref{fig:inconeldispTBCphase} respectively differ substantially from those of only Inconel 718 (see \cref{fig:groupshift}). The velocity of the two lowest order modes do not converge towards the Rayleigh wave speed at high frequency-thickness products, as the $B_1$ mode propagates $\sim$1000~\si{\metre\per\second} faster than the $B_0$ mode.

Figure~\ref{fig:inconelTBCdisplacement} shows the through-thickness displacement profiles for five different modes, at frequency-thickness values that correspond to points of group velocity maxima for the associated modes. At low frequency-thickness products (below the cut-off frequency of third mode, $B_2$) the modes respond in a similar manner to $A_0$ and $S_0$ in a single material, as the first mode ($B_0$) exhibits large out-of-plane motion (Figure~\ref{fig:B0}), while the second mode ($B_1$) exhibits large in-plane motion (Figure~\ref{fig:B1}). Above $B_1$, however, the top coat exhibits a large through-thickness displacement in comparison to the rest of the materials, which is especially apparent for $B_2$ (\cref{fig:B2}). Modes $B_5$, $B_7$, and $B_9$ have group velocity maxima distinctly separate from other modes, which is advantageous for the excitation/identification of single modes, however these higher order modes exhibit relatively less displacement than lower order modes (\cref{fig:B5}), which limits their sensitivity to holes, and results in lower amplitude signals.

\begin{table}[h]
    \centering
    \begin{adjustbox}{width=1\textwidth,center=\textwidth}
    \begin{tabular}{llllll}
        \toprule
        \textbf{Layer type} & \textbf{Material} & \makecell[l]{\textbf{Young's Modulus} \\ (GPa)} & \textbf{Poisson's ratio} & \makecell[l]{\textbf{Density}\\(kg/m$^3$)} & \makecell[l]{\textbf{Thickness}\\(\textmu m)} \\ 
        \midrule
        Top coat  & ZrO\textsubscript{2}-8 wt\% Y\textsubscript{2}O\textsubscript{3} (8YSZ) & 48  & 0.1  & 5770~\cite{Zhang2012} & 200  \\
        TGO       & $\alpha$-Al\textsubscript{2}O\textsubscript{3}                 & 400 & 0.23 & 3987~\cite{Bodisova2006}  & 10   \\
        Bond coat & NiCrAlY                 & 200 & 0.3  & 7500~\cite{Of2014} & 100  \\
        Substrate & Inconel 718~\cite{SpecialMetals2007}             & 202 & 0.29 & 8226 & 1000 \\ 
        \bottomrule
    \end{tabular}
\end{adjustbox}
\caption{Material properties of thermal barrier coatings.}\label{table:TBCmatprops}
\end{table}

\begin{table}[h]
    \centering
    
    \begin{tabular}{llll}
        \toprule
        \textbf{Layer type} & \textbf{Material} & \makecell[l]{\textbf{Longitudinal velocity}\\(\si{\metre\per\second})} & \makecell[l]{\textbf{Shear velocity}\\(\si{\metre\per\second})} \\ 
        \midrule
        Top coat  & ZrO\textsubscript{2}-8 wt\% Y\textsubscript{2}O\textsubscript{3} (8YSZ)     & 2916.84  & 1944.56   \\
        TGO       & $\alpha$-Al\textsubscript{2}O\textsubscript{3}              & 10784.57 & 6386.15  \\
        Bond coat & NiCrAlY                     & 5991.45 & 3202.56  \\
        Substrate & Inconel 718                 & 5672.72 & 3085.12  \\ 
        \bottomrule
    \end{tabular}

\caption{Wave velocities of thermal barrier coatings.}\label{table:TBCwavespeeds}
\end{table}

\begin{figure}[p]
    \centering
        \begin{subfigure}[c]{0.75\textwidth}
            \centering
            \includegraphics[width=1\textwidth]{./figures/IncTBCGroup.eps}
            \caption{Energy velocity.}\label{fig:inconeldispTBCgroup}
        \end{subfigure}
        \vskip\baselineskip
        \begin{subfigure}[c]{0.75\textwidth}
            \centering
            \includegraphics[width=1\textwidth]{./figures/IncTBCPhase.eps}
            \caption{Phase velocity.}\label{fig:inconeldispTBCphase}
        \end{subfigure}
        \caption{Dispersion curves for Inconel 718 with TBC applied (\cref{table:TBCmatprops}).}
\end{figure}

\begin{figure}[p]
    \raggedright
        \begin{subfigure}[c]{0.45\textwidth}
            \centering
            \includegraphics[width=1\textwidth]{./figures/B0.eps}
            \caption{$B_0$ Displacement -- 1 MHz.}\label{fig:B0}
        \end{subfigure}
        \hfill
        \begin{subfigure}[c]{0.45\textwidth}
            \centering
            \includegraphics[width=1\textwidth]{./figures/B1.eps}
            \caption{$B_1$ Displacement -- 1 MHz.}\label{fig:B1}
        \end{subfigure}
        \vskip\baselineskip
        \begin{subfigure}[c]{0.45\textwidth}
            \centering
            \includegraphics[width=1\textwidth]{./figures/B2.eps}
            \caption{$B_2$ Displacement -- 2.4 MHz.}\label{fig:B2}
        \end{subfigure}
        \hfill
        \begin{subfigure}[c]{0.45\textwidth}
            \centering
            \includegraphics[width=1\textwidth]{./figures/B3.eps}
            \caption{$B_3$ Displacement -- 2.4 MHz.}\label{fig:B3}
        \end{subfigure}
        \vskip\baselineskip
        \begin{subfigure}[c]{0.65\textwidth}
            \centering
            \includegraphics[width=1\textwidth]{./figures/B5.eps}
            \caption{$B_5$ Displacement -- 4.4 MHz.}\label{fig:B5}
        \end{subfigure}
    \caption{Through thickness displacement profiles for Inconel 718 with TBC applied (\cref{table:TBCmatprops}).}\label{fig:inconelTBCdisplacement}
\end{figure}

\clearpage
\section{Temperature dependant study}

Temperature dependant Young's modulus for TBC materials are provided in~\cref{table:TBCmatprops}, and shown in~\cref{fig:TBC_E}. All three sources of values for the top coat (YSZ) show linear reductions in $E$ with temperature, while the reduction in $E$ for the bond coat (MCrAlY) increases with temperature. Saucedo-Mora \textit{et al.}\cite{Saucedo-Mora2015} suggest that the data provided by Beghini \textit{et al.}\cite{Beghini2001} shows a significant decrease in $E$ due to microcracking. The elevated initial values of $E$ in comparison to the other sources suggests that the tested material has undergone thermal aging~\cite{Yamazaki2007,Nath2015}. The change in Young's modulus with temperature for the TGO is ignored, as the effect is considered insignificant due to the thickness of the layer. Temperature dependant material properties for Inconel 718 are shown in \cref{fig:dpe}. The Poisson's ratio of the TBC materials is considered to be temperature independent for this study, 0.33 for the bond coat, and 0.2 for the top coat. Changes in density due to thermal expansion in the TBC are not considered in this study, as the effect on wave propagation is small in comparison to changes in Young's modulus, as discussed in~\cref{effectoftemponwaveprop}.

As temperature increases there is a large shift in the shape of the dispersion curves (from \cref{fig:inconeldispTBCgroup20c} to \cref{fig:inconeldispTBCgroup1020c}), which makes it difficult to select a suitable mode/frequency to target. The lowest two modes, $B_0$ \& $B_1$, are the only two modes that show relatively stable areas of low dispersion, in the regions of 0.4-0.8 MHz, however the effect of dispersion still increases with temperature, especially for $B_1$. $B_5$ is the only other mode that exhibits an energy velocity peak that would allow the single mode to be identified across the temperature range of interest, however there is only a very small frequency window (\sim 3.1 MHz) in which dispersion is relatively stable.

\begin{table}[h]
    \centering
    \begin{tabular}{@{}llllllll@{}}
    \toprule
    \textbf{Source} & \textbf{Material} & \textbf{$E_{20 \si{\degreeCelsius}}$} & \textbf{$E_{220 \si{\degreeCelsius}}$} & \textbf{$E_{420 \si{\degreeCelsius}}$} & \textbf{$E_{620 \si{\degreeCelsius}}$} & \textbf{$E_{820 \si{\degreeCelsius}}$} & \textbf{$E_{1020 \si{\degreeCelsius}}$} \\ \midrule
    Bednarz \textit{et al.}\cite{Bednarz2005} & Top Coat & 17.50   & 16.34  & 15.18  & 14.02  & 12.86  & 11.70  \\
    Beghini \textit{et al.}\cite{Beghini2001} & Top Coat & 46.50   & 41.75  & 37.00     & 32.25  & 27.50   & 22.75 \\
    Gregori \textit{et al.}\cite{Gregori2007} & Top Coat & 22.00     & 21.05  & 20.14  & 19.26  & 18.43  & 17.62 \\ 
    Bednarz \textit{et al.}\cite{Bednarz2005} & Bond Coat & 151.85 & 150.75 & 145.25 & 132.33 & 108.92 & 71.89 \\
    COMSOL & Inconel 718 & 201.86 & 193.23 & 181.86 & 166.89 & 147.46 & 122.72 \\ \bottomrule
    \end{tabular}
    \caption{Temperature dependant Young's modulus of substrate and TBC materials.}\label{table:TBCmatprops_tempdep}
\end{table}

\begin{figure}[h]
    \centering
    \includegraphics[width=0.8\textwidth]{./figures/TBC_E.eps}
    \caption{Values of Young's modulus for TBC materials.}\label{fig:TBC_E}
\end{figure}

\clearpage

\subsection{COMSOL Simulations}

A smaller wedge model is used in comparison to the simulations carried out in \cref{simstudymain} to reduce computation time, however the same process of subtracting wedge-to-wedge time from the total time of flight, as well as calculating the wedge foot offset is used. The same wedge spacing (100 mm) as used in the previous study was also used in this study. The sampling rate used in the simulation is 6.72\times$10^7$, which provides a maximum theoretical velocity resolution of \pm 2.97 \si{\metre\per\second}. Simulations are carried out at six temperatures, 20\si{\degreeCelsius}, 220\si{\degreeCelsius}, 420\si{\degreeCelsius}, 620\si{\degreeCelsius}, 820\si{\degreeCelsius}, and 1020\si{\degreeCelsius}.

The heat transfer physics model has been disabled for the wedges, with only the plate affected by a change in temperature. The large temperature range causes a large change to the material properties of the Inconel 718 and TBC, which in turn causes a large change in wave speed. As the change is so large the wedge angle has to be adjusted to continually target the same area of the frequency spectrum, as shown in Table~\ref{table:wedgeangleIncTBC}. The longitudinal velocity of the wedge material is 2477~\unit{\metre\per\second}, as measured experimentally.

\cref{fig:IncTBC_waveprop} shows the time of flight shift of the $B_1$ mode from 20\si{\degreeCelsius} to 1020 \si{\degreeCelsius}. The response at 20\si{\degreeCelsius} is shown in green, while the response at 1020 \si{\degreeCelsius} is shown in red. The black dashed lines indicate the peaks of each envelope, which are used in the calculation of time of flight. As temperature increases the effect of dispersion becomes more apparent, and there is less of a defined central peak to the wave packet. 

Time-displacement data is transformed into frequency-wavenumber data using 2D-FFT from spatial B-scan data, using 90 point probes equally spaced 0.8 mm apart. This allows individual modes to be identified, and compared with dispersion curves for verification. Both in-plane ($x$) and out-of-plane ($y$) displacement was monitored. It should be noted that the displacement response in the plate measured using a point probe will differ from the response received at a second receiver wedge. The use of a second wedge further isolates a particular mode of interest, while the displacement in the plate may still show the presence of other modes.

\begin{table}[h]
    \centering
    \begin{tabulary}{\textwidth}{LLL}
        \toprule
    \textbf{\textbf{Temperature (\si{\degreeCelsius})}} & \textbf{Phase velocity (\unit{\metre\per\second})} & \textbf{Wedge angle (\degree)} \\
        \midrule
    20           & 4837.87           & 34.7                \\
    220          & 4751.88            & 35.5                \\
    420          & 4636.66            & 36.6                \\
    620          & 4464.19            & 38.4                \\
    820          & 4211.13           & 41.4                \\
    1020          & 3867.29            & 46.7                \\
    \bottomrule
    \end{tabulary}%
    \caption{Wedge angle required for $B_1$ mode excitation in Inconel 718 with TBC from 20\si{\degreeCelsius} to 1020\si{\degreeCelsius}.}\label{table:wedgeangleIncTBC}
\end{table}

\cref{fig:IncTBC_Comsol} shows predicted and simulated group velocity of the $B_1$ mode (0.8 MHz) from 20\si{\degreeCelsius} to 1020\si{\degreeCelsius} in Inconel 718 with TBC applied. The increased reduction of velocity with temperature for the simulated result is due to the target frequency falling into a more dispersive area of the spectrum, which increases the error associated with measuring time of flight from the peaks of the signal envelopes. 

\begin{figure}[h]
    \centering
    \includegraphics[width=0.8\textwidth]{./figures/IncTBCwaveprop2temp.eps}
    \caption{Wave propagation of $B_1$ mode (0.8 MHz) at 20\si{\degreeCelsius} (green) and 1020 \si{\degreeCelsius} (red) in Inconel 718 with TBC applied.}\label{fig:IncTBC_waveprop}
\end{figure}

\begin{figure}[h]
    \centering
    \includegraphics[width=0.8\textwidth]{./figures/IncTBCcomsolresult.eps}
    \caption{Predicted and simulated group velocity of $B_1$ mode (0.8 MHz) from 20\si{\degreeCelsius} to 1020 \si{\degreeCelsius} in Inconel 718 with TBC applied.}\label{fig:IncTBC_Comsol}
\end{figure}

\begin{figure}[p]
    \centering
        \begin{subfigure}[c]{0.75\textwidth}
            \centering
            \includegraphics[width=1\textwidth]{./figures/IncTBCGroup20c.eps}
            \caption{20\si{\degreeCelsius}}\label{fig:inconeldispTBCgroup20c}
        \end{subfigure}
        \vskip\baselineskip
        \begin{subfigure}[c]{0.75\textwidth}
            \centering
            \includegraphics[width=1\textwidth]{./figures/IncTBCGroup1020c.eps}
            \caption{1020\si{\degreeCelsius}}\label{fig:inconeldispTBCgroup1020c}
        \end{subfigure}
        \caption{Energy velocity dispersion curves for Inconel 718 with TBC applied (\cref{table:TBCmatprops_tempdep}).}
\end{figure}


\printbibliography[title={Chapter~\thechapter~Bibliography}]
