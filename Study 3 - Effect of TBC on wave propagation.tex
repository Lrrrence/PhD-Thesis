\chapter{The effect of thermal barrier coatings on wave propagation}

In this section the effect of thermal barrier coatings on signal propagation is investigated through the generation of dispersion curves, and COMSOL simulation. 

Thermal barrier coatings are made up of multiple layers, a bond coat, a thermally grown oxide layer (TGO), and a top coat. The Young's modulus of YSZ coatings shows a significant dependence on porosity and microstructure, where Young's modulus decreases with increasing porosity. The $E$ of ZrO\textsubscript{2}-4 mol\% Y\textsubscript{2}O\textsubscript{3} has been shown to decrease from around 200 GPa at 8\% porosity, to 80 GPa at 28\% porosity~\cite{Jang2005}. A number of studies~\cite{Yamazaki2007, Nath205} have shown that the Young's modulus of TBC materials increases after thermal aging, as fine microcracks and porosities are sealed. The increase is in the region of 30 GPa. Exposure to higher temperatures causes a larger increase in $E$.

Considering that YSZ exhibits substantially lower Young's modulus than the other materials there is likely to be a comparatively larger through-thickness displacement in the top coat, which will make propagating waves substantially more sensitive to defects in the TBC surface. This effect will be partially reduced as the value of $E$ increases after thermal aging. 

In a non-symmetric laminate such as this the modes do not have a clear symmetric or antisymmetric character, especially when the different layers have substantially different material properties, and therefore wave speeds. The modes are therefore denoted using $B$ (as proposed by \href{https://www.dlr.de/zlp/en/desktopdefault.aspx/tabid-14332/24874_read-61142/#/gallery/33485}{The Dispersion Calculator}~\cite{Huber}), rather than categorising them as either symmetric, $S$, or antisymmetric, $A$. 

\section{Prediction study}

A typical configuration is considered in Table~\ref{table:TBCmatprops}, where the values for Young's modulus and Poisson's ratio are provided by Li~\cite{Li2017b}. Table~\ref{table:TBCwavespeeds} shows the longitudinal and shear wave velocities of each TBC material used in this study, as calculated by COMSOL from the material properties used in Table~\ref{table:TBCmatprops}. 

The energy and phase velocity dispersion curves shown in Figures~\ref{fig:inconeldispTBCgroup} and~\ref{fig:inconeldispTBCphase} respectively differ substantially from those of only Inconel 718. The velocity of the two lowest order modes do not converge towards the Rayleigh wave speed at high frequency-thickness products, as the $B_1$ mode propagates $\sim$1000~\si{\metre\per\second} faster than the $B_0$ mode.

Figure~\ref{fig:inconelTBCdisplacement} shows the through-thickness displacement profiles for five different modes, at frequency-thickness values that correspond to points of group velocity maxima for the associated modes. At low frequency-thickness products (below the cut-off frequency of third mode, $B_2$) the modes respond in a similar manner to $A_0$ and $S_0$ in a single material, as the first mode ($B_0$) exhibits large out-of-plane motion (Figure~\ref{fig:B0}), while the second mode ($B_1$) exhibits large in-plane motion (Figure~\ref{fig:B1}). Above $B_1$, however, the top coat exhibits a large through-thickness displacement in comparison to the rest of the materials, which is especially apparent for $B_2$ (\cref{fig:B2}). Modes $B_5$, $B_7$, and $B_9$ have group velocity maxima distinctly separate from other modes, which is advantageous for the excitation/identification of single modes, however these higher order modes exhibit relatively less displacement than lower order modes (\cref{fig:B5}), which limits their sensitivity to holes, and results in lower amplitude signals.

\begin{table}[h]
    \centering
    \begin{adjustbox}{width=1\textwidth,center=\textwidth}
    \begin{tabular}{llllll}
        \toprule
        \textbf{Layer type} & \textbf{Material} & \makecell[l]{\textbf{Young's Modulus} \\ (GPa)} & \textbf{Poisson's ratio} & \makecell[l]{\textbf{Density}\\(kg/m$^3$)} & \makecell[l]{\textbf{Thickness}\\(\textmu m)} \\ 
        \midrule
        Top coat  & ZrO\textsubscript{2}-8 wt\% Y\textsubscript{2}O\textsubscript{3} (8YSZ) & 48  & 0.1  & 5770~\cite{Zhang2012} & 200  \\
        TGO       & $\alpha$-Al\textsubscript{2}O\textsubscript{3}                 & 400 & 0.23 & 3987~\cite{Bodisova2006}  & 10   \\
        Bond coat & NiCrAlY                 & 200 & 0.3  & 7500~\cite{Of2014} & 100  \\
        Substrate & Inconel 718~\cite{SpecialMetals2007}             & 202 & 0.29 & 8226 & 1000 \\ 
        \bottomrule
    \end{tabular}
\end{adjustbox}
\caption{Material properties of thermal barrier coatings.}\label{table:TBCmatprops}
\end{table}

\begin{table}[h]
    \centering
    
    \begin{tabular}{llll}
        \toprule
        \textbf{Layer type} & \textbf{Material} & \makecell[l]{\textbf{Longitudinal velocity}\\(\si{\metre\per\second})} & \makecell[l]{\textbf{Shear velocity}\\(\si{\metre\per\second})} \\ 
        \midrule
        Top coat  & ZrO\textsubscript{2}-8 wt\% Y\textsubscript{2}O\textsubscript{3} (8YSZ)     & 2916.84  & 1944.56   \\
        TGO       & $\alpha$-Al\textsubscript{2}O\textsubscript{3}              & 10784.57 & 6386.15  \\
        Bond coat & NiCrAlY                     & 5991.45 & 3202.56  \\
        Substrate & Inconel 718                 & 5672.72 & 3085.12  \\ 
        \bottomrule
    \end{tabular}

\caption{Wave velocities of thermal barrier coatings.}\label{table:TBCwavespeeds}
\end{table}

\begin{figure}[p]
    \centering
        \begin{subfigure}[c]{0.75\textwidth}
            \centering
            \includegraphics[width=1\textwidth]{./figures/IncTBCGroup.eps}
            \caption{Energy velocity.}\label{fig:inconeldispTBCgroup}
        \end{subfigure}
        \vskip\baselineskip
        \begin{subfigure}[c]{0.75\textwidth}
            \centering
            \includegraphics[width=1\textwidth]{./figures/IncTBCPhase.eps}
            \caption{Phase velocity.}\label{fig:inconeldispTBCphase}
        \end{subfigure}
        \caption{Dispersion curves for Inconel 718 with TBC.}
\end{figure}

\begin{figure}[p]
    \raggedright
        \begin{subfigure}[c]{0.45\textwidth}
            \centering
            \includegraphics[width=1\textwidth]{./figures/B0.eps}
            \caption{$B_0$ Displacement -- 1 MHz.}\label{fig:B0}
        \end{subfigure}
        \hfill
        \begin{subfigure}[c]{0.45\textwidth}
            \centering
            \includegraphics[width=1\textwidth]{./figures/B1.eps}
            \caption{$B_1$ Displacement -- 1 MHz.}\label{fig:B1}
        \end{subfigure}
        \vskip\baselineskip
        \begin{subfigure}[c]{0.45\textwidth}
            \centering
            \includegraphics[width=1\textwidth]{./figures/B2.eps}
            \caption{$B_2$ Displacement -- 2.4 MHz.}\label{fig:B2}
        \end{subfigure}
        \hfill
        \begin{subfigure}[c]{0.45\textwidth}
            \centering
            \includegraphics[width=1\textwidth]{./figures/B3.eps}
            \caption{$B_3$ Displacement -- 2.4 MHz.}\label{fig:B3}
        \end{subfigure}
        \vskip\baselineskip
        \begin{subfigure}[c]{0.65\textwidth}
            \centering
            \includegraphics[width=1\textwidth]{./figures/B5.eps}
            \caption{$B_5$ Displacement -- 4.4 MHz.}\label{fig:B5}
        \end{subfigure}
    \caption{Through thickness displacement profiles for Inconel 718 with TBC applied.}\label{fig:inconelTBCdisplacement}
\end{figure}

\clearpage
\section{Simulation study}

Three wave modes with clear peaks in group velocity were targeted, to be excited with a wedge transducer. $B_1$ at 1~MHz-mm (30.1\degree), $B_3$ at 2.4~MHz-mm (25.3\degree), and $B_5$ at 4.4~MHz-mm (25.6\degree). These frequencies products correspond to peaks in the energy velocity dispersion curves, where velocity is at a maximum for these modes. The use of wedge transducers further isolates each mode.  

Temperature dependant material properties of TBC materials are provided by Sfar \textit{et al.}\cite{Sfar2002} and Saucedo-Mora \textit{et al.}\cite{Saucedo-Mora2015}.


\printbibliography[title={Chapter~\thechapter~Bibliography}]
