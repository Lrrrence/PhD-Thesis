\chapter{Theoretical temperature sensitivity of Lamb waves}\label{theory}

From the outcome of the literature review, a temperature monitoring system for nozzle guide vanes based on the temperature dependent wave velocity of Lamb waves has been investigated. In this section the temperature dependence of various Lamb wave modes has been calculated theoretically by producing dispersion curves from material properties. Dispersion curves have been generated for Aluminium, to be validated experimentally in Section~\ref{experiments}.

\section{Generation of dispersion curves}

Theoretical dispersion curves calculated from the material properties of Aluminium have been produced using \href{https://www.dlr.de/zlp/en/desktopdefault.aspx/tabid-14332/24874_read-61142/#/gallery/33485}{The Dispersion Calculator}~\cite{Huber}. For isotropic media Rayleigh-Lamb equations are solved numerically to generate dispersion curves, based on the book by Rose~\cite{Rose2014}. The Rayleigh-Lamb frequency equations can be written as:

\begin{equation}
    \frac{\tan(qh)}{\tan(ph)}=-\frac{4k^2pq}{(q^2-k^2)^2}
\end{equation}

For symmetric modes, and:

\begin{equation}
    \frac{\tan(qh)}{\tan(ph)}=-\frac{q^2-k^2}{(4k^2pq)^2}
\end{equation}

For antisymmetric modes. Where $p$ is given by:

\begin{equation}
    p^2=(\frac{\omega}{c_L})-k^2
\end{equation}

And $q$ is given by:

\begin{equation}
    q^2=(\frac{\omega}{c_T})-k^2
\end{equation}

Where $c_L$ is bulk longitudinal velocity, $c_T$ is bulk shear velocity, $c_p$ is phase velocity, $h$ is thickness, and $\omega$ is angular frequency. The wavenumber $k$ is numerically equal to $\omega/c_p$. The phase velocity is related to the wavelength by the simple relation $c_p = (\omega/2\pi)\lambda$. Once phase velocity is calculated stress, displacement, and group velocity can also be calculated. Group velocity is given by:

\begin{equation}
    c_g=c^2_p\left[c_p-(fd)\frac{dc_p}{d(fd)}\right]^{-1}
\end{equation}

Where $fd$ denotes frequency thickness product. Note that, when the derivative of $c_p$ with respect to $fd$ becomes zero, $c_g = c_p$. Note also that, as the derivative of $c_p$ with respect to $fd$ approaches infinity (i.e., at cut-off), $c_g$ approaches zero.

\section{The effect of temperature on wave propagation in aluminium}

A change in temperature will affect material properties, namely Young's modulus, Poisson's ratio, and Density. The change in density (thermal expansion) and Poisson's ratio is negligible over the temperature of interest (20\si{\degreeCelsius}--100\si{\degreeCelsius}) in comparison to the change in Young's modulus. 

\subsection{Thermal expansion}
Thermal expansion will cause a change in material dimensions which will also affect the wave speed, although the effect is small. The coefficient of thermal expansion, $\alpha$, for Aluminium 1050 at 20\si{\degreeCelsius} is 2.3$\times 10^{-5}$, rising to 2.5$\times 10^{-5}$ at 100\si{\degreeCelsius}, as described by the equation:
%
\begin{equation}
    \alpha (T) = 1.243109{\times}10^{-5}+5.050772{\times}10^{-8}\times T^1-5.806556{\times}10^{-11}\times T^2+3.014305{\times}10^{-14} \times T^3
\end{equation}
%
Where $T$ is temperature in Kelvin, the equation is valid from 230 K to 900 K~\cite{Nix1941,Feder1958,Gibbons1958}.

The change in time of flight due to thermal expansion can be expressed using the equation~\cite{Croxford2007}:
%
\begin{equation}
    \delta t = \frac{d}{v} \left(\alpha - \frac{k}{v}\right)\delta T
\end{equation}
%
Where $d$ is the propagation distance (0.1 m), $v$ is the group velocity, and $k$ is the temperature sensitivity of group velocity, $\delta v/\delta T = k$. The group velocity can then be recalculated with the effect of thermal expansion included. The average change in velocity due to thermal expansion over the temperature range of interest (20\si{\degreeCelsius}--100\si{\degreeCelsius}) is -1.59 m s$^{-1}$ for the $S_0$ mode, -0.89 m s$^{-1}$ for the $A_1$ mode, and -1.47 m s$^{-1}$ for the $S_1$ mode. The change is sufficiently small that thermal expansion can be excluded from the COMSOL model.

\subsection{Poisson's ratio}

The Poisson's ratio, $\nu$, for Aluminium 1050 at 20\si{\degreeCelsius} is 0.3310, rising to 0.3325 at 100\si{\degreeCelsius}, as described by the equation~\cite{Lalpoor2009}:
%
\begin{equation}
\begin{split}
    \nu (T) = 0.3238668+3.754548{\times}10^{-6}\times T^1+2.213647{\times}10^{-7}\times T^2-6.565023{\times}10^{-10}\times T^3 \\ +~4.21277{\times}10^{-13}\times T^4+3.170505{\times}10^{-16}\times T^5
\end{split}
\end{equation}
%
Where $T$ is temperature in Kelvin, the equation is valid from 0 K to 773 K. 

This is equivalent to a velocity change of 0.33 m s$^{-1}$ for the $S_0$ mode at 10\si{\degreeCelsius} assuming no change to density or Young's modulus.

\subsection{Young's modulus}
The change in Young's modulus with temperature in Aluminium is represented by the Equation~\cite{Sakai1996,McLellan1987,Stokes1960,Naimon1975}: 
%
\begin{equation}
    E\left( T \right) = 7.770329{\times}10^{10} + 2036488.0 \times T^1 - 189160.7\times T^2+425.2931 \times T^3-0.3545736 \times T^4
\end{equation} 
%
Where $T$ is temperature in Kelvin, and $E$ is Young's modulus in Pascals. 
The values of Young's modulus produced by this equation have been used to generate the temperature dependant dispersion curves for aluminium shown in Figure~\ref{fig:grouptempshuftalu}. The angles of excitation required to excite the $S_0$, $A_1$, and $S_1$ modes are shown based on the frequency-thickness products of 1~MHz-mm (28\degree), 2.5~MHz-mm (21\degree), and 4~MHz-mm (25\degree) respectively. These frequency-thickness products correspond closely to group velocity maxima for each of the targeted modes, which is advantageous for separating the modes in the time domain~\cite{Alleyne1992}.

The group velocity of the $S_0$, $A_1$, and $S_1$ modes have been extracted from the curves at frequency-thickness products of 1~MHz-mm, 2.5~MHz-mm, and 4~MHz-mm respectively. The extracted group velocities are shown in Figure~\ref{fig:alugroupvel}. The temperature sensitivities of the modes are: 1.45--1.52 m s$^{-1}$\si{\degreeCelsius}$^{-1}$ for the $S_0$ mode, 0.77--0.87 m s$^{-1}$\si{\degreeCelsius}$^{-1}$ for the $A_1$ mode, and 1.37--1.45 m s$^{-1}$\si{\degreeCelsius}$^{-1}$ for the $S_1$ mode, over the temperature range 10--110\si{\degreeCelsius} (see Equation~\ref{eqn: eq k}). These values are frequency and mode dependent. The temperature sensitivities increase with temperature, as the modes reduces in frequency (shift left), and reduces in wave velocity (shift down). 

\begin{figure}[!htbp]
    \centering
    \includegraphics[width=.8\textwidth]{./figures/grouptempshift_new.eps}
    \caption{$A_0$, $S_0$, $A_1$, and $S_1$ group velocity dispersion curve shift with temperature from 20\si{\degreeCelsius} to 100\si{\degreeCelsius} for Aluminium 1050 H14.}\label{fig:grouptempshuftalu}
\end{figure}

\begin{figure}[!htbp]
    \centering
    \includegraphics[width=.8\textwidth]{./figures/1mhzshift_new.eps}
    \caption{$S_0$, $A_1$, and $S_1$ group velocity change with temperature from 10\si{\degreeCelsius} to 110\si{\degreeCelsius} for Aluminium 1050 H14.}\label{fig:alugroupvel}
\end{figure}
\newpage
\section{Multi-modal Lamb waves}

The multi-modal nature of Lamb waves makes their analysis complex, however this can be simplified by targeting specific modes using careful selection of excitation frequency, to target a particular frequency-thickness product. This difficulty can be seen in Figure~\ref{fig:multimode} where a large number of different modes combine to create a multi-modal wave packet. The varying wave speeds of each mode with frequency causes dispersion, which has an increasing effect the longer the waves propagate.  

\begin{figure}[!htbp]
    \centering
    \includegraphics[width=\textwidth]{./figures/multimodeedit.eps}
    \caption{An example of multi-modal wave packets. Simulated 100 mm wave propagation of 10--cycle Hamming windowed 1~MHz sine pulse in a 4 mm thick aluminium plate.}\label{fig:multimode}
\end{figure}

Below the cut-off frequency of the $A_1$ mode only the two fundamental modes ($A_0$ \& $S_0$) are present. Their phase/group velocities are sufficiently different from one another that the modes can be distinguished between in the time domain, assuming a long enough propagation distance. The use of wedge transducers can further isolate between the modes, as the wedge angle required to excite them are largely different. 

Above the cut-off frequency of the $A_1$ mode there are an increasing number of modes present. The relatively small difference in phase velocity between the higher order pairs ($A_1$ \& $S_1$, $A_2$ \& $S_2$, etc.) means that even with the use of wedge transducers it is not possible to selectively excite only one mode. The ability to differentiate between the modes is improved by targeting points of group velocity minima/maxima, where the modes are travelling with the largest differences in velocity, which helps to separate them in the time domain.

\printbibliography[title={Chapter~\thechapter~Bibliography}]