\documentclass{report}
\usepackage[utf8]{inputenc}
\usepackage[refsection=chapter,style=numeric]{biblatex}
\addbibresource{library.bib}
\addbibresource{References.bib}
\usepackage{gensymb}
\usepackage{color,soul}
\usepackage{graphicx}
\usepackage{xcolor}
\usepackage{amssymb}
\usepackage{tabu}
\usepackage{tabulary}
\usepackage{booktabs}
\usepackage{moreverb,url}
\usepackage[breaklinks,colorlinks,bookmarksopen,bookmarksnumbered,citecolor=sotonRed,urlcolor=sotonMarineBlue,linkcolor=sotonRed]{hyperref}
\usepackage{textcomp}
\usepackage{siunitx}
\usepackage{tabularx}
\usepackage{amsmath}
\usepackage[raggedright]{titlesec}
% stops headings from hyphenating
\usepackage{geometry} % to change margins
\usepackage{ragged2e} % provides \RaggedLeft
\usepackage{pdflscape}
\usepackage{pgfgantt}
\usepackage{standalone}
\usepackage{typearea} 
\usepackage{parskip}
\usepackage{notoccite} % fix citation order
\usepackage{pdfpages}
\usepackage[margin=10pt,font=small,labelfont=bf,
labelsep=endash]{caption}

\definecolor{barblue}{RGB}{153,204,254}
\definecolor{groupblue}{RGB}{51,102,254}
\definecolor{linkred}{RGB}{165,0,33} 
\definecolor{sotonMarineBlue}{RGB}{1,67,89} % Soton marine blue (P 7469C)
\definecolor{sotonGrey}{RGB}{153,153,166} % Soton grey (P 443C)
\definecolor{sotonRed}{RGB}{171,18,16} % Soton Red (P 484C)

\geometry{a4paper,textwidth=15cm, textheight=21.3cm, marginratio={1:1,5:7}}

\renewcommand{\today}{\ifcase \month \or January\or February\or March\or %
April\or May \or June\or July\or August\or September\or October\or November\or %
December\fi, \number \year} 

\begin{document}

\begin{titlepage}
    \begin{center}
      
        \centering
        \includegraphics[width=.4\textwidth]{./figures/UoSLogo.png}         
        
        \vspace{0.5cm}

        \Large
        Electronics and Computer Science\\
        Smart Electronic Materials and Systems Research Group\\
        University of Southampton\\
                  
        \vfill

        \Huge
        \textbf{Temperature monitoring of nozzle guide vanes using ultrasonic guided waves}
        
        \vfill

        \LARGE
        \textit{by}  \\
        \textbf{Lawrence Martyn Yule}\\
        \Large
        ORCiD:~\href{https://orcid.org/0000-0002-0324-6642}{0000-0002-0324-6642}
            
        \vfill

        \textit{A thesis for the degree of\\Doctor of Philosophy}
                  
        \vfill
        \today
        %\vspace{1cm}
        

   
    \end{center}
\end{titlepage}

\begin{abstract}

This study will investigate the use of ultrasonic guided waves (Lamb waves) for the temperature monitoring of nozzle guide vanes (NGVs). These components are found in the turbine section of jet engines and are operated at extremely high temperatures. A literature review has been carried out that covers the existing temperature monitoring systems for turbine blades and nozzle guide vanes. Both offline and online methods are presented and their advantages and disadvantages are examined. The use of offline systems is well established but their online equivalents are difficult to implement because of the limited access to components. There is the need for an improved sensor that is capable of mapping temperature in real time with minimum interference to the operating conditions of the engine, allowing operating temperatures to be increased to the limits of the components and maximising efficiency. Acoustic monitoring techniques are already used for a large number of structural health monitoring (SHM) applications and have the potential to be adapted for use in temperature monitoring for turbine blades and NGVs. High temperatures severely affect the response of ultrasonic transducers. However, waveguides and buffer rods can be used to distance transducers from extreme conditions, while piezoelectric materials such as YCOB and AlN have been developed for use at high temperatures. The geometry of turbine blades and NGVs allows Lamb waves to propagate through their structure, and the presence of numerous cooling holes will produce acoustic reflections that have the potential to be utilised for temperature mapping.

A number of studies are underway to investigate the effects of the physical environment on wave propagation, and to determine the Lamb wave mode best suited to the application. The generation of dispersion curves from material properties allows theoretical temperature sensitivity to be determined, which has been verified experimentally. A test system has been developed to target modes of interest, and analyse the effect of temperature (and other factors) on wave propagation. A COMSOL model has also been developed that will be used in future studies to investigate the effect of cooling holes, surface coatings, and curved surfaces on wave propagation. The model will also be used to investigate different sensor configurations that can be implemented on an NGV. 

\end{abstract}

{
  \hypersetup{hidelinks}
  \tableofcontents
  \listoffigures
  \begingroup
  \let\clearpage\relax
  \listoftables
  \endgroup
}

\input{1 Intro.tex}
\input{2 Literature Review.tex}
\input{3 Acoustic temperature sensing.tex}
\chapter{Theoretical temperature sensitivity of Lamb waves}\label{theory}

From the outcome of the literature review, a temperature monitoring system for nozzle guide vanes based on the temperature dependent wave velocity of Lamb waves has been investigated. In this section the temperature dependence of various Lamb wave modes has been calculated theoretically by producing dispersion curves from material properties. Dispersion curves have been generated for Aluminium, to be validated experimentally in Section~\ref{experiments}.

\section{Generation of dispersion curves}

Theoretical dispersion curves calculated from the material properties of Aluminium have been produced using \href{https://www.dlr.de/zlp/en/desktopdefault.aspx/tabid-14332/24874_read-61142/#/gallery/33485}{The Dispersion Calculator}~\cite{Huber}. For isotropic media Rayleigh-Lamb equations are solved numerically to generate dispersion curves, based on the book by Rose~\cite{Rose2014}. The Rayleigh-Lamb frequency equations can be written as:

\begin{equation}
    \frac{\tan(qh)}{\tan(ph)}=-\frac{4k^2pq}{(q^2-k^2)^2}
\end{equation}

For symmetric modes, and:

\begin{equation}
    \frac{\tan(qh)}{\tan(ph)}=-\frac{q^2-k^2}{(4k^2pq)^2}
\end{equation}

For antisymmetric modes. Where $p$ is given by:

\begin{equation}
    p^2=(\frac{\omega}{c_L})-k^2
\end{equation}

And $q$ is given by:

\begin{equation}
    q^2=(\frac{\omega}{c_T})-k^2
\end{equation}

Where $c_L$ is bulk longitudinal velocity, $c_T$ is bulk shear velocity, $c_p$ is phase velocity, $h$ is thickness, and $\omega$ is angular frequency. The wavenumber $k$ is numerically equal to $\omega/c_p$. The phase velocity is related to the wavelength by the simple relation $c_p = (\omega/2\pi)\lambda$. Once phase velocity is calculated stress, displacement, and group velocity can also be calculated. Group velocity is given by:

\begin{equation}
    c_g=c^2_p\left[c_p-(fd)\frac{dc_p}{d(fd)}\right]^{-1}
\end{equation}

Where $fd$ denotes frequency thickness product. Note that, when the derivative of $c_p$ with respect to $fd$ becomes zero, $c_g = c_p$. Note also that, as the derivative of $c_p$ with respect to $fd$ approaches infinity (i.e., at cut-off), $c_g$ approaches zero.

\section{The effect of temperature on wave propagation in aluminium}

A change in temperature will affect material properties, namely Young's modulus, Poisson's ratio, and Density. The change in density (thermal expansion) and Poisson's ratio is negligible over the temperature of interest (20\si{\degreeCelsius}--100\si{\degreeCelsius}) in comparison to the change in Young's modulus. 

\subsection{Thermal expansion}
Thermal expansion will cause a change in material dimensions which will also affect the wave speed, although the effect is small. The coefficient of thermal expansion, $\alpha$, for Aluminium 1050 at 20\si{\degreeCelsius} is 2.3$\times 10^{-5}$, rising to 2.5$\times 10^{-5}$ at 100\si{\degreeCelsius}, as described by the equation:
%
\begin{equation}
    \alpha (T) = 1.243109{\times}10^{-5}+5.050772{\times}10^{-8}\times T^1-5.806556{\times}10^{-11}\times T^2+3.014305{\times}10^{-14} \times T^3
\end{equation}
%
Where $T$ is temperature in Kelvin, the equation is valid from 230 K to 900 K~\cite{Nix1941,Feder1958,Gibbons1958}.

The change in time of flight due to thermal expansion can be expressed using the equation~\cite{Croxford2007}:
%
\begin{equation}
    \delta t = \frac{d}{v} \left(\alpha - \frac{k}{v}\right)\delta T
\end{equation}
%
Where $d$ is the propagation distance (0.1 m), $v$ is the group velocity, and $k$ is the temperature sensitivity of group velocity, $\delta v/\delta T = k$. The group velocity can then be recalculated with the effect of thermal expansion included. The average change in velocity due to thermal expansion over the temperature range of interest (20\si{\degreeCelsius}--100\si{\degreeCelsius}) is -1.59 m s$^{-1}$ for the $S_0$ mode, -0.89 m s$^{-1}$ for the $A_1$ mode, and -1.47 m s$^{-1}$ for the $S_1$ mode. The change is sufficiently small that thermal expansion can be excluded from the COMSOL model.

\subsection{Poisson's ratio}

The Poisson's ratio, $\nu$, for Aluminium 1050 at 20\si{\degreeCelsius} is 0.3310, rising to 0.3325 at 100\si{\degreeCelsius}, as described by the equation~\cite{Lalpoor2009}:
%
\begin{equation}
\begin{split}
    \nu (T) = 0.3238668+3.754548{\times}10^{-6}\times T^1+2.213647{\times}10^{-7}\times T^2-6.565023{\times}10^{-10}\times T^3 \\ +~4.21277{\times}10^{-13}\times T^4+3.170505{\times}10^{-16}\times T^5
\end{split}
\end{equation}
%
Where $T$ is temperature in Kelvin, the equation is valid from 0 K to 773 K. 

This is equivalent to a velocity change of 0.33 m s$^{-1}$ for the $S_0$ mode at 10\si{\degreeCelsius} assuming no change to density or Young's modulus.

\subsection{Young's modulus}
The change in Young's modulus with temperature in Aluminium is represented by the Equation~\cite{Sakai1996,McLellan1987,Stokes1960,Naimon1975}: 
%
\begin{equation}
    E\left( T \right) = 7.770329{\times}10^{10} + 2036488.0 \times T^1 - 189160.7\times T^2+425.2931 \times T^3-0.3545736 \times T^4
\end{equation} 
%
Where $T$ is temperature in Kelvin, and $E$ is Young's modulus in Pascals. 
The values of Young's modulus produced by this equation have been used to generate the temperature dependant dispersion curves for aluminium shown in Figure~\ref{fig:grouptempshuftalu}. The angles of excitation required to excite the $S_0$, $A_1$, and $S_1$ modes are shown based on the frequency-thickness products of 1~MHz-mm (28\degree), 2.5~MHz-mm (21\degree), and 4~MHz-mm (25\degree) respectively. These frequency-thickness products correspond closely to group velocity maxima for each of the targeted modes, which is advantageous for separating the modes in the time domain~\cite{Alleyne1992}.

The group velocity of the $S_0$, $A_1$, and $S_1$ modes have been extracted from the curves at frequency-thickness products of 1~MHz-mm, 2.5~MHz-mm, and 4~MHz-mm respectively. The extracted group velocities are shown in Figure~\ref{fig:alugroupvel}. The temperature sensitivities of the modes are: 1.45--1.52 m s$^{-1}$\si{\degreeCelsius}$^{-1}$ for the $S_0$ mode, 0.77--0.87 m s$^{-1}$\si{\degreeCelsius}$^{-1}$ for the $A_1$ mode, and 1.37--1.45 m s$^{-1}$\si{\degreeCelsius}$^{-1}$ for the $S_1$ mode, over the temperature range 10--110\si{\degreeCelsius} (see Equation~\ref{eqn: eq k}). These values are frequency and mode dependent. The temperature sensitivities increase with temperature, as the modes reduces in frequency (shift left), and reduces in wave velocity (shift down). 

\begin{figure}[!htbp]
    \centering
    \includegraphics[width=.8\textwidth]{./figures/grouptempshift_new.eps}
    \caption{$A_0$, $S_0$, $A_1$, and $S_1$ group velocity dispersion curve shift with temperature from 20\si{\degreeCelsius} to 100\si{\degreeCelsius} for Aluminium 1050 H14.}\label{fig:grouptempshuftalu}
\end{figure}

\begin{figure}[!htbp]
    \centering
    \includegraphics[width=.8\textwidth]{./figures/1mhzshift_new.eps}
    \caption{$S_0$, $A_1$, and $S_1$ group velocity change with temperature from 10\si{\degreeCelsius} to 110\si{\degreeCelsius} for Aluminium 1050 H14.}\label{fig:alugroupvel}
\end{figure}
\newpage
\section{Multi-modal Lamb waves}

The multi-modal nature of Lamb waves makes their analysis complex, however this can be simplified by targeting specific modes using careful selection of excitation frequency, to target a particular frequency-thickness product. This difficulty can be seen in Figure~\ref{fig:multimode} where a large number of different modes combine to create a multi-modal wave packet. The varying wave speeds of each mode with frequency causes dispersion, which has an increasing effect the longer the waves propagate.  

\begin{figure}[!htbp]
    \centering
    \includegraphics[width=\textwidth]{./figures/multimodeedit.eps}
    \caption{An example of multi-modal wave packets. Simulated 100 mm wave propagation of 10--cycle Hamming windowed 1~MHz sine pulse in a 4 mm thick aluminium plate.}\label{fig:multimode}
\end{figure}

Below the cut-off frequency of the $A_1$ mode only the two fundamental modes ($A_0$ \& $S_0$) are present. Their phase/group velocities are sufficiently different from one another that the modes can be distinguished between in the time domain, assuming a long enough propagation distance. The use of wedge transducers can further isolate between the modes, as the wedge angle required to excite them are largely different. 

Above the cut-off frequency of the $A_1$ mode there are an increasing number of modes present. The relatively small difference in phase velocity between the higher order pairs ($A_1$ \& $S_1$, $A_2$ \& $S_2$, etc.) means that even with the use of wedge transducers it is not possible to selectively excite only one mode. The ability to differentiate between the modes is improved by targeting points of group velocity minima/maxima, where the modes are travelling with the largest differences in velocity, which helps to separate them in the time domain.

\printbibliography[title={Chapter~\thechapter~Bibliography}]
\chapter{COMSOL simulations of Lamb wave propagation}\label{simulations}

The multiphysics simulation package COMSOL has been used to simulate potential temperature monitoring systems, investigating the effect of temperature on Lamb wave propagation.

The literature covering the use of COMSOL for modelling Lamb wave excitation using wedge transducers is limited, however it has been shown that Lamb waves can be successfully generated using this method~\cite{Nikolaevtsev2016a}.

\section{Variable angle wedge simulation}

A 2D model has been produced of the experimental test setup described in Section~\ref{experiments}. This allows for validation of the time of flight measurements, and can be used to separate the effect of temperature on the wedges from the substrate. The effect of temperature on the Lamb wave alone can therefore be analysed. 

\subsection{Geometry}

The model consists of two variable angle wedges (PMMA), which are based on the geometry of the Olympus variable angle wedges used in the experimental investigation, placed on top of an aluminium plate. The thickness of the plate can be varied to target different Lamb wave modes at different frequency-thickness products. The initial thickness is set to 1 mm to target the $S_0$ mode at 1 MHz--mm. The transmitting wedge has a simplified piezoelectric transducer (PZT-5H from COMSOL's material library) attached to it's rotating block, to which the excitation signal is applied. The geometry can be seen in Figure \ref{fig:COMSOLdiagram}. The received signal is measured at the receiver wedge's rotating block boundary. More realistic transducer configurations are not considered in this study, as the focus is on the effect of temperature on the propagating wave. A boundary area is set underneath the plate to act as the heat source, again mimicking the experimental setup. This is simplified to allow the temperature to be directly set, rather than simulating a hot plate.

\begin{figure}[h]
    \centering
    \includegraphics[width=.8\textwidth]{./figures/comsoldiagram.png}
    \caption{COMSOL geometry diagram}\label{fig:COMSOLdiagram}
\end{figure}

\subsection{Material properties}

The change in Young's Modulus with temperature is included in the material properties for both the wedges and the aluminium using piecewise functions. Unfortunately the specific material properties of the Olympus wedges used in the experimental study are unknown, and the values for PMMA found in literature vary quite dramatically. Figure~\ref{fig:PMMAE} shows five sources of temperature dependant Young's modulus for PMMA, where the shaded green region indicates the temperature range of interest in this study (20-100\si{\degreeCelsius}). Polynomial fits have been applied to the data sets to allow them to be used in COMSOL (apart from COMSOL's in-built values which are given as a function) and are provided in Table~\ref{table:PMMAE}, where $T$ is the temperature in Kelvin. The Alexandria~\cite{Abdel-Wahab2017} and Goodyear Aerospace~\cite{Hassard1973} sources have been disregarded as the large reduction in $E$ over the temperature range of interest is not considered realistic, based on the use of the Olympus wedges in the experimental study. The Birmingham sources~\cite{Sahputra2018} (provided at two different annealing temperatures, 600K and 1000K) differ greatly from the values provided by COMSOL's material library entry for PMMA~\cite{Fukuhara1995}. Room temperature values of Young's modulus (for which there are many more sources) range from 1.8-5.0 GPa, however they are mostly commonly given at $\sim$3.0 GPa. In order to accurately represent the wedge material used experimentally, the value of Young's modulus has been inferred from measurements of longitudinal wave velocity. 

\begin{table}[h]
    \centering
    \begin{tabulary}{\textwidth}{LLL}
        \toprule
        \textbf{Property} & \textbf{PMMA} & \textbf{Aluminium}  \\
        \midrule
        Heat capacity at constant pressure (J/(kg$\cdot$K)) & 1470 & 904 \\
        Density (kg/m$^3$) & 1190 & 2700\\
        Thermal conductivity (W/(m$\cdot$K)) & 0.18  & 237\\
        Young's modulus (Pa) & XXX & XXX \\
        Poisson's ratio & 0.35 & 0.3375\\
        \bottomrule
    \end{tabulary} 
    \caption{COMSOL material properties}\label{table:matprop}
\end{table}
%
\begin{figure}[h]
    \centering
    \includegraphics[width=.8\textwidth]{./figures/PMMAE.eps}
    \caption{Young's modulus of PMMA.}\label{fig:PMMAE}
\end{figure}
%
\begin{table}[h]
    \centering
\begin{adjustbox}{width=1.2\textwidth,center=\textwidth}
    \begin{tabular}{ll}
        \toprule
        \textbf{Source} & \textbf{Function}\\
        \midrule
        COMSOL Material Library~\cite{Fukuhara1995} & $E = 4.3102{\times}10^9 + 6.9344{\times}10^7 \times T^1 - 5.2821{\times}10^5 \times T^2+ 1.5796{\times}10^3 \times T^3-1.7421\times T^4$ \\
        Birmingham 600K Annealing~\cite{Sahputra2018} & $E = 58.3 \times T^3 + -7.6500{\times}10^4 \times T^2 + 2.1367{\times}10^7 \times T + 2.8500{\times}10^9$ \\
        Birmingham 1000K Annealing~\cite{Sahputra2018}  & $E = -116.7 \times T^3 + 9.3000{\times}10^4 \times T^2 + -2.8333{\times}10^7 \times T + 7.0200{\times}10^9$ \\
        Alexandria~\cite{Abdel-Wahab2017} & $E = -2.6773{\times}10^4 \times T^3 + 2.4371{\times}10^7 \times T^2 + -7.4029{\times}10^9 \times T + 7.5540{\times}10^{11}$ \\
        Goodyear Aerospace~\cite{Hassard1973} & $E = -1.5514{\times}10^5 \times T^2 + 6.4090{\times}10^7 \times T + -1.5120{\times}10^9$ \\
        \bottomrule
    \end{tabular}
\end{adjustbox}
\caption{Functions for PMMA Young's modulus.}\label{table:PMMAE}
\end{table}     
%
\subsection{Determination of Young's modulus from wedge wave velocity}

The longitudinal wave velocity of the wedges has been measured experimentally by placing the two wedges together, as shown in Figure~\ref{fig:w2wonaxis}. Time of flight is measured using an envelope peak finding function, and wave velocity calculated using the propagation distance (0.0715 m). The measured velocity of 2477 m s$^{-1}$ differs from that provided by the manufacturer (2720 m s$^{-1}$). Now that the wave velocity is known, the material properties of the model can be adjusted until the wave velocity matches. The model was set to run a parametric sweep of Young's modulus for the wedges, from 3.5$\times$10$^9$ to 6$\times$10$^9$ in 0.5$\times$10$^9$ increments, matching the range of potential values from literature. The wave velocity for each of these steps is calculated, and a polynomial fit of the data is generated in MATLAB. The quadratic equation produced is used to find the value of E closest to the wave velocity measured experimentally. The model was then rerun at smaller increments of E (4.20$\times$10$^9$ to 4.40$\times$10$^9$ in increments of 0.05$\times$10$^9$) to improve the accuracy of the polynomial fit. The model is then computed using this value of $E$ to verify that the velocity matches the prediction.

In order to determine the temperature dependant Young's modulus for the wedge material, the test setup was placed inside of an oven, and time of flight was measured up to 45\si{\degreeCelsius}. Increasing the temperature of the oven above this caused the signal amplitude to decrease dramatically, making time of flight measurement unreliable. The temperature of the oven was allowed time to stabilise, along with time of flight. The velocity calculated at this temperature was then used to find the associated value of $E$, as described previously. These values were then entered into COMSOL using an interpolation function, extrapolating the value of $E$ for higher and lower temperatures linearly. This method is sufficiently accurate for the small temperature range used in this study, and better represents the real material than using values derived from literature. 

\begin{figure}[h]
    \centering
    \includegraphics[width=.7\textwidth]{./figures/w2wdiagramonaxis.eps}
    \caption{Cross-sectional diagram of on-axis wedge-to-wedge time-of-flight measurement setup.}\label{fig:w2wonaxis}
\end{figure}

The experimentally measured longitudinal wave velocity of the wedges is also used to calculate the wedge angle required to excite particular modes, based on Snell's law. 

\subsection{Aluminium plate properties}

The choice of temperature dependent $E$ for Aluminium is more straight forward, as sources for bulk aluminium and aluminium 1050 are provided by the COMSOL material library, and they are very similar to values provided by Hopkins~\cite{Hopkins2012}, as shown in Figure~\ref{fig:AluE}. The shaded green region indicates the temperature range of interest in this study (20-100\si{\degreeCelsius}).The functions used to generate the curves are given in Table~\ref{table:AluE}.
%
\begin{figure}[h]
    \centering
    \includegraphics[width=.8\textwidth]{./figures/Alu_E.eps}
    \caption{Young's modulus of Aluminium.}\label{fig:AluE}
\end{figure}
%
\begin{table}[h]
    \centering
\begin{adjustbox}{width=1.2\textwidth,center=\textwidth}
    \begin{tabular}{ll}
        \toprule
        \textbf{Source} & \textbf{Function}\\
        \midrule
        COMSOL Aluminium 1050 & $7.7703{\times}10^{10} + 2.0365{\times}10^6 \times T^1 -1.8916{\times}10^5 \times T^2 + 4.2529{\times}10^2 \times T^3 -3.5457 {\times}{10^{-1}} \times T^4 $ \\
        COMSOL Aluminium Bulk & $7.6593{\times}10^{10} + 2.0074{\times}10^6 \times T^1 -1.8646{\times}10^5 \times T^2 + 4.1922{\times}10^2 \times T^3 -3.4951 {\times}{10^{-1}} \times T^4 $\\
        Hopkins~\cite{Hopkins2012} & $-4{\times}10^7 \times T + 8{\times}10^{10}$\\
        \bottomrule
    \end{tabular}
\end{adjustbox}
\caption{Functions for Aluminium Young's modulus.}\label{table:AluE}
\end{table}    
%
The change in Poisson's ratio and density is assumed to negligible and is not included in the simulation. Thermal expansion is also considered to have a negligible effect on the propagation distance and is excluded (calculated to have an average reduction in wave velocity of the $S_0$ mode in aluminium of -1.20 m s$^{-1}$ over the temperature range 20-100\si{\degreeCelsius}). 

\subsection{Physics \& mesh settings}

The modules Solid Mechanics, Electrostatics, and Heat Transfer in Solids are used in this simulation, along with a multiphysics node to couple Solid Mechanics with Electrostatics for the piezoelectric effect. Both the wedges and the plate are set to isotropic linear elastic materials, with low reflecting boundaries applied to the wedges.

The simple piezoelectric transducer for the transmitting wedge is set up as follows: A zero charge node is used for the edges of the material, initial values are set to 0 V, a ``Charge Conservation, Piezoelectric'' node is set for the material, a ground boundary is selected for the wedge side of the material, and a terminal node is set for the opposite boundary. Within the terminal node the type is set to Voltage and the input is set to V0(t). The excitation signal is a 1 MHz 5--cycle Hamming windowed sine pulse generated in MATLAB and included in COMSOL using an analytic function (Definitions$>$Functions$>$Analytic), given in Equation~\ref{eq:sinepulse} and shown in Figure~\ref{fig:excitation}.
%
\begin{figure}[h]
    \centering
    \includegraphics[width=.8\textwidth]{./figures/excitation.eps}
    \caption{Hamming windowed 5--cycle pulse.}\label{fig:excitation}
\end{figure}
%
\begin{equation}
   \text{Hamming windowed 5--cycle pulse} = \sin(2 \pi f_0 t)\times t< \left(\frac{np}{f_0}\right) \times 0.54 - 0.46\times \cos \left(\frac{2 \pi t}{t_0 \times np} \right) \label{eq:sinepulse}
\end{equation}
Where $f_0$ is 1~MHz, the number of cycles ($np$) is 5, and $t_0$ is equal to $1/f_0$.

For the Heat Transfer in Solids module all the domains are set to solid, and initial values are set to 20\si{\degreeCelsius}. The boundaries that are exposed to the air are selected in a Heat Flux node, where convective heat flux is selected. A user defined heat transfer coefficient of 15~W/(m$^2\cdot$K) is used for the plate, and 5~W/(m$^2\cdot$K) for the wedges. These values were adjusted to produce the temperature gradients measured experimentally in both the plate and the wedges. The external temperature is set to 20\si{\degreeCelsius}. The temperature of the boundary underneath the plate is adjusted as required (20\si{\degreeCelsius} to 100\si{\degreeCelsius} in 20\si{\degreeCelsius} increments for this study). An example of the temperature gradients produced from the stationary study step are shown in Figure~\ref{fig:COMSOLtemp100c}, where the temperature boundary underneath the plate is set to 100\si{\degreeCelsius}.

\begin{figure}[h]
    \centering
    \includegraphics[width=.8\textwidth]{./figures/comsoltemp100c.png}
    \caption{Simulated temperature gradients from stationary study at 100\si{\degreeCelsius}.}\label{fig:COMSOLtemp100c}
\end{figure}

The mesh size for each material is determined by excitation frequency. The excitation wavelength for each of the materials is calculated by dividing their longitudinal wave speed by $f_0$. A free triangular mesh is created for each of the materials, and the maximum element size for each of them is set to LocalWavelength/N. If higher frequency content is expected, the wavelength for each material should be based on the highest frequency expected rather than $f_0$. In order to accurately resolve a wave, at least 10--12 elements per local wavelength are required~\cite{COMSOL2013}. This assumes linear discretization for all modules. Using 12 elements results in an average skewness rating (measure of element quality, 0--1) of 0.9345 over 154728 elements~\cite{COMSOL2017}. This is equivalent to a sample rate of 1.2$\times$10$^8$.

\subsection{Study settings}

This study has two steps, firstly a stationary study to simulate the effect of temperature on the system until an equilibrium is reached, and secondly a time dependant study to simulate wave propagation that has it's initial conditions set by the stationary study. The settings for the initial study are adjusted to solve for heat transfer but not solve for electrostatics/the piezoelectric effect. Changing temperature causes a change in Young's modulus, which subsequently affects wave velocity.

The time dependant study includes electrostatics/the piezoelectric effect to allow for wave generation, but does not include heat transfer. This reduces computation time as it is not necessary to model changing temperature as the time dependant model solves, only to use the fixed values of Young's modulus that have been passed on from the stationary study. The time dependant study has its ``Output times'' set to: range(0,dt,sim\textunderscore length) where ``dt'' is a global definition parameter equal to CFL/(N$\times f_0$). The CFL (Courant Friedrichs Lewy) number is suggested by COMSOL~\cite{COMSOL2021} to be less than 0.2, optimally 0.1 (when the default second order, quadratic, mesh elements are used). This value represents the relationship between wave speed, $c$, maximum mesh size, $h$, and time step length, $\Delta t$: $CFL = c\Delta t/h$. This can be rewritten in terms of frequency as the maximum mesh size $h$ has already been manually defined by $N$, the number of elements per local wavelength for each material: $CFL = fN\Delta t$. This can then be rearranged to give the time step: $\Delta t = CFL/Nf$. 

Under ``Values of Dependant Variables'' the settings are changed to user controlled, method is changed to Solution, and the study is set to the stationary study. The time step is manually set under Solver Configurations$>$Solution 1$>$Time dependant solver$>$Time stepping. Here the ``Steps taken by solver'' parameter is changed to ``Manual'' and the ``Time Step'' is set to: $CFL/(N\times f_0)$. 

To reduce file size only the data at the wedge boundaries is stored by the solver. This can be achieved by adding an ``Explicit Selection'' node in the Geometry section, and selecting both the transmit and receive wedge boundaries. Within the time dependant study settings select ``For selection'' under ``Store fields in output'' and select the boundary group~\cite{COMSOL2021a}. 

A parametric sweep node was used to cycle through the temperature boundary values (20\si{\degreeCelsius} to 100\si{\degreeCelsius} in 20\si{\degreeCelsius} increments) and save the output of the time dependant model for each value. This is repeated for the model in the wedge-to-wedge configuration (mimicking the experimental setup shown in Figure~\ref{fig:testdiagramw2w}). The simulations were run on the University of Southampton's IRIDIS 5 supercomputing platform~\cite{Southampton2021}.

\section{Simulation results}

Exaggerated deformation of pressure in the plate as seen in Figure~\ref{fig:simmodes} makes the presence of the $A_0$ and $S_0$ modes clearly visible. The modes are separated in the time domain after a short distance ($\sim$~50 mm) due to the difference in group velocity. 

\begin{figure}[h]
    \centering
    \includegraphics[width=.8\textwidth]{./figures/simmodes.png}
    \caption{Presence of the $A_0$ \& $S_0$ modes.}\label{fig:simmodes}
\end{figure}

To visualise wave propagation and calculate time of flight the pressure at both transmitter and receiver wedge boundaries are exported, and the time of flight is measured using an envelope peak extraction method, to allow direct comparison with experimental results. This method of time of flight measurement can also be applied to more dispersive signals, which cannot be achieved using cross correlation methods. Various signal processing techniques for time of flight measurement are discussed in detail by Guers~\cite{Guers2011}. An example of wave propagation at room temperature can be seen in Figure~\ref{fig:COMSOLsimsignal}. 

\begin{figure}[h]
    \centering
    \includegraphics[width=.8\textwidth]{./figures/simpulse20csensors.eps}
    \caption{COMSOL simulation of $S_0$ mode propagation at room temperature.}\label{fig:COMSOLsimsignal}
\end{figure}

Calculated total time of flight (through both the wedges and the plate) is marginally longer than experimentally measured time of flight, which can be attributed to a number of factors. Differences in material properties, their change with temperature, variance in geometry, wedge angle, wedge spacing, and sample rate, all have an impact on time of flight. Wedge foot offset (the distance a wave travel under each wedge foot) is calculated in the same way for both the simulation and the experiments, however the value differs, which indicates a difference in geometry between them. Despite this difference the difference in calculated velocities is small, as using accurate estimations of wedge foot offset corrects for the difference in total time of flight. Time of flight in the wedge-to-wedge configuration is in line with experimental measurements, which suggests that the geometry and material properties of the wedges are realistic. The material properties of the aluminium plate are the same as those used in the theoretical study, which should (in theory) mean that the velocity in the simulated plate is the same as was extracted from dispersion curves. Frequency analysis of the transmitted wave shows that it is still centred at 1 MHz as expected. 

Figure~\ref{fig:s0result} shows the change in velocity with temperature for the $S_0$ Lamb wave mode in Aluminium, comparing theoretical temperature sensitivity extracted from dispersion curves, experimental measurement data (Section~\ref{S0 experiments}), and COMSOL simulations of the experimental setup. The results from the COMSOL model are in good agreement with those taken experimentally, which also match up well to the theoretical temperature sensitivity of Aluminium extracted from dispersion curves. The experimental result is within 4.89 $\pm$ 2.27 m s$^{-1}$ or 0.05\% of the theoretical velocity on average. The COMSOL results are within 3.25 m s$^{-1}$ or 0.02\% of the theoretical result on average.

\begin{figure}[h]
    \centering
    \includegraphics[width=.8\textwidth]{./figures/s0andcomsolresult.eps}
    \caption{Velocity change with temperature for $S_0$ Lamb wave mode in Aluminium. Comparison between theoretical, experimental, and simulated results.}\label{fig:s0result}
\end{figure}

\printbibliography[title={Chapter~\thechapter~Bibliography}]
\chapter{Experimental investigation of Lamb wave temperature sensitivity}\label{experiments}

The wave velocity temperature sensitivity of the $S_0$, $A_1$, and $S_1$ Lamb wave modes have been experimentally measured, to be validated against theoretical predictions in Chapter~\ref{theory}.

Two 1 MHz piezoelectric transducers attached to acrylic wedges (Olympus variable angle wedge) in a pitch-catch configuration have been coupled to a 1~mm thick aluminium plate with a liquid couplant (Figure~\ref{fig:testsetup}). A signal generator (GW Instek MFG-2203M) has been used to generate a 5-cycle Hamming windowed tone burst at 1~MHz. Signals are digitised using a Picoscope 3406DMSO USB Oscilloscope. Based on a sampling rate of 5$\times$10$^8$ the theoretical maximum temporal resolution is 2~ns. Signal processing is carried out in MATLAB. A zero-phase bandpass filter is applied to the signals to remove unwanted noise. Time of flight ($t_F$) is measured between transducers and wave velocity is calculated from the distance between transducers. The temperature of the aluminium plate is controlled using a hot plate.

\begin{table}[b]
    \centering
    \begin{tabulary}{\textwidth}{L}
        \hline
        \textbf{Measurement Hardware}     \\
        \hline
        2x Olympus ABWX-2001 Variable angle wedges \\
        2x Olympus A539S-SM 1 MHz transducers \\
        Olympus ultrasonic couplant B \\
        GW Instek MFG-2203M Signal generator \\
        Picoscope 3406DMSO USB Oscilloscope \\
        Thermadata T-type temperature loggers \\
        VWR Hot plate \\
        \hline
    \end{tabulary} 
    \caption{Experimental measurement hardware.}\label{table:hardware}
\end{table}

\newpage

\begin{figure}[h]
    \centering
    \includegraphics[width=.9\textwidth]{./figures/testdiagramsimple.eps}
    \caption{Cross-sectional diagram of total time-of-flight measurement setup.}\label{fig:testdiagramtotal}
\end{figure}

\begin{figure}[h]
    \centering
    \includegraphics[width=.7\textwidth]{./figures/w2wdiagram.eps}
    \caption{Cross-sectional diagram of wedge-to-wedge time-of-flight measurement setup.}\label{fig:testdiagramw2w}
\end{figure}

Wave velocity is calculated using Equation~\ref{velocitycalc}/\ref{velocitycalcfull}. The propagation time through the wedges (measured using the configuration shown in Figure~\ref{fig:testdiagramw2w}) has been subtracted from the total $t_F$ to ensure that only the propagation time through the plate is measured.  
%
\begin{equation} \label{velocitycalc}
v = \frac{d}{t_F}
\end{equation} 
%
\begin{equation} \label{velocitycalcfull}
v = \left( \frac{d\;\text{between wedges} + d\;\text{wedge foot offset}}{\text{Total}\;t_F - \text{Wedge-to-wedge}\;t_F} \right)
\end{equation} 
%
\\
Where the $d$ wedge foot offset is an unknown distance from the front edge of the wedge to where the wave enters the plate from the wedge. This distance has been calculated by measuring wave velocity at room temperature for the mode of interest at multiple wedge spacings (0.08 m to 0.14 m in 0.01 m increments) and looping through a range of plausible offset distances until the standard deviation across the range of wedge spacings is at a minimum. This ensures that the variation in measurement results is due to measurement error (e.g. small variances in setting the distance between wedges) rather than an incorrect estimation of wedge foot offset. This value varies with wedge angle and is recalculated for each wave mode measured.

The wedges allow for careful selection of excitation angle so that modes of interest can be targeted. The angle is determined based on Snell’s law:
%
\begin{equation} 
\text{Angle}\ \theta = \text{Sin}^{- 1} \left( \frac{\text{Longitudinal\ wedge\ velocity}}{\text{Lamb\ wave\ phase\ velocity}} \right)
\end{equation} 
%
Measurement of wave velocity depends on measurement of time of flight ($t_F$), which can be described by the Equation~\cite{Croxford2007a}:
%
\begin{equation}\label{tofcalc}
t_F = \frac{d}{c}
\end{equation} 
%
Where $d$ is the distance travelled at wave speed $c$, both of which are functions of temperature, $T$. The sensitivity of the time of flight to temperature can then be expressed as:
%
\begin{equation} 
\delta t_{F} = \frac{d}{c}\left( \alpha - \frac{k}{c} \right) \delta \text{T}
\end{equation} 
%
Where $\alpha$ is the coefficient of thermal expansion of the medium and $k$ is the rate of change of wave velocity with temperature:
%
\begin{equation} {\label{eqn: eq k}}
k = \frac{\delta \text{c}}{\delta \text{T}}
\end{equation} 
%
\newpage
\section{Test Method}

A hot plate is used to raise the temperature of the aluminium plate to the desired temperature. The temperature of the aluminium plate is monitored using a thermocouple placed in the centre of the plate at the hottest point. The total $t_F$ is measured until it stabilises using the test setup shown in Figure~\ref{fig:testdiagramtotal}. The temperature of the entire system must be allowed to stabilise before taking the measurement to ensure that the temperature of the wedge is the same as the plate. Total $t_F$ is now measured for the set temperature. Multiple measurements are taken after adjusting both wedge positions. The wedges are removed from the surface and placed together to measure the wedge-to-wedge $t_F$ as shown in Figure~\ref{fig:testdiagramw2w}. Multiple measurements are taken after adjusting wedge-to-wedge position. The $t_F$ measurement process is repeated after allowing the total $t_F$ to re-stabilise. Velocity is calculated using Equation~\ref{velocitycalcfull}. A mean average is calculated from the results of the repeated total $t_F$ measurements, and velocity is calculated for every wedge-to-wedge result. An average velocity is calculated along with standard deviation. 

The temperature gradient across the plate has been measured by placing four equally spaced thermocouples along the transmission path, from the centre of the plate to the furthest edge of a wedge transducer in 3 cm increments. The wedges are removed from the plate to place the thermocouples, and the temperature of the hot plate is raised to match the temperature recorded by the thermocouple placed in the centre of the plate during measurement of total $t_F$. Measurements are repeated after moving the thermocouples to the other half of the transmission path. A mean average temperature has been calculated for the total transmission path at each hot plate temperature setting. 

\begin{figure}[ht]
    \centering
    \includegraphics[width=.8\textwidth]{./figures/hjkhh7UmEx.png}
    \caption{Photograph of test setup.}\label{fig:testsetup}
\end{figure}

\clearpage
\section{S0 mode (1 MHz-mm)}\label{S0 experiments}
%
The wedge angle required for the $S_0$ mode is:
%
\begin{equation} 
31{^\circ} = \text{Sin}^{- 1} \left( \frac{2720}{5258} \right)
\end{equation} 
%
The $A_0$ mode cannot be excited using this method as the phase velocity at this frequency (2312 m s$^{-1}$) is slower than the longitudinal velocity of the wedge. If the $A_0$ mode is present in the signal it will not affect measurement of the $S_0$ mode as it’s group velocity is significantly different than that of the $S_0$ mode, which will cause a distinct second wave packet. 

\begin{figure}[h]
    \centering
    \includegraphics[width=.8\textwidth]{./figures/s0xcorr5cyc.eps}
    \caption{Time of flight ($t_F$) measurement of $S_0$ mode wave propagation using cross-correlation function at room temperature.}\label{fig:S0crosscorr}
\end{figure}

\subsection{S0 mode results}

\begin{figure}[h]
    \centering
    \includegraphics[width=.8\textwidth]{./figures/aluplatemeasured.eps}
    \caption{Group velocity change with temperature for the $S_0$ mode in Aluminium 1050 H14.}\label{fig:result}
\end{figure}

Figure~\ref{fig:result} shows experimentally measured wave velocity of the $S_0$ mode plotted against theoretical wave velocity extracted from dispersion curves. Error bars show the standard deviation from the mean. After accounting for the temperature gradient across the transmission path by calculating a temperature average the change in velocity is comparable with predicted velocity extracted from dispersion curves, within 4.89 $\pm$ 2.27 m s$^{-1}$ on average. The temperature sensitivity of the system is 1.26--1.78 m s$^{-1}$\si{\degreeCelsius}$^{-1}$ over the range 24\si{\degreeCelsius}--94\si{\degreeCelsius}. The sensitivity is extracted from a second-order polynomial fit of the data (r$^2$ = 0.9992). The slope away from predicted results (increasing with temperature) can be attributed to the increasing temperature gradient, both in the plate and in the wedges. The gradient is shown to be almost linear (r$^2$=0.9967) across the measurement distance. Increasing temperature is also likely to have an effect on the operation of the piezoelectric transducer (amplitude and centre frequency), however this effect is negligible over the tested temperature range. The wedge angle required to excite the $S_0$ mode will also vary with temperature, however the change is only around 1\si{\degree} between 20\si{\degreeCelsius} and 100\si{\degreeCelsius}.
\newpage
\section{A1 mode (2.5 MHz-mm)}

The wedge angle required for the $A_1$ mode is:
%
\begin{equation} 
24{^\circ} = \text{Sin}^{- 1} \left( \frac{2720}{6654} \right)
\end{equation} 
%
\begin{figure}[h]
    \centering
    \includegraphics[width=.8\textwidth]{./figures/a1xcorr5cyc.eps}
    \caption{Time of flight ($t_F$) measurement of $A_1$ mode wave propagation using cross-correlation function at room temperature.}\label{fig:A1crosscorr}
\end{figure}

\subsection{A1 mode results}

Figure~\ref{fig:A1result} shows experimentally measured wave velocity of the $A_1$ mode plotted against theoretical wave velocity extracted from dispersion curves. Error bars show the standard deviation from the mean. After accounting for the temperature gradient across the transmission path by calculating a temperature average the change in velocity is comparable with predicted velocity extracted from dispersion curves, within 2.43 $\pm$ 1.97 m s$^{-1}$ on average. The temperature sensitivity of the system is 1.09--1.17 m s$^{-1}$\si{\degreeCelsius}$^{-1}$ over the range 26\si{\degreeCelsius}--97\si{\degreeCelsius}. The sensitivity is extracted from a second-order polynomial fit of the data (r$^2$ = 0.9990).

\begin{figure}[h!]
    \centering
    \includegraphics[width=.8\textwidth]{./figures/a1moderesult.eps}
    \caption{Group velocity change with temperature for the $A_1$ mode in Aluminium 1050 H14.}\label{fig:A1result}
\end{figure}
\newpage
\section{S1 mode (4 MHz-mm)}

This region of frequency-thickness product is multi-modal, with both the $A_1$ and $S_1$ modes present. Similarities in phase velocity leads to similar excitation angles, which causes both modes to be excited. Using a cross-correlation method for measuring time-of-flight is no longer appropriate, as the received signal differs substantially from the input signal. An envelope peak method is employed instead, whereby analytic envelopes for both the excitation signal and received signal are generated by Hilbert FIR filtering, with a filter length of 2000 samples. This produces a smooth envelope with clearly defined peaks, as seen in Figure~\ref{fig:S1timedomain}. A peak finding algorithm is used to detect the envelope peaks, as denoted by the dashed lines. The $S_1$ mode ($\sim$~4550 m s$^{-1}$) arrives at the receiver before the $A_1$ mode ($\sim$~2550 m s$^{-1}$) as it has a considerably higher group velocity. At this propagation distance the two modes are clearly separated in the time domain, with the $A_1$ mode showing considerably more dispersion.  

The wedge angle required for the $S_1$ mode is:
%
\begin{equation} 
28{^\circ} = \text{Sin}^{- 1} \left( \frac{2720}{5874} \right)
\end{equation} 
%
\subsection{S1 mode results}

Figure~\ref{fig:S1result} shows experimentally measured wave velocity of the $S_1$ mode plotted against theoretical wave velocity extracted from dispersion curves. Error bars show the standard deviation from the mean. After accounting for the temperature gradient across the transmission path by calculating a temperature average the change in velocity is comparable with predicted velocity extracted from dispersion curves, within 4.44 $\pm$ 7.15 m s$^{-1}$ on average. The temperature sensitivity of the system is 1.80 m s$^{-1}$\si{\degreeCelsius}$^{-1}$ over the range 25\si{\degreeCelsius}--103\si{\degreeCelsius}. The sensitivity is extracted from a linear fit of the data (r$^2$ = 0.9777).

\begin{figure}[h!]
    \centering
    \includegraphics[width=.8\textwidth]{./figures/s1hilbert.eps}
    \caption{Time of flight ($t_F$) measurement of $S_1$ mode using envelope peak method at room temperature.}\label{fig:S1timedomain}
\end{figure}

\begin{figure}[h!]
    \centering
    \includegraphics[width=.8\textwidth]{./figures/s1moderesult.eps}
    \caption{Group velocity change with temperature for the $S_1$ mode in Aluminium 1050 H14.}\label{fig:S1result}
\end{figure}

\newpage
\section{Experimental sensitivity analysis}

There are a number of experimental error sources to consider. The physical distance between wedges is controlled using 3D printed spacers that keep the wedges aligned at set distances. The movement of the wedges on the surface of the plate increases with temperature as the viscosity of the couplant decreases. Variations in placement cause the calculated velocity to vary by around $\pm$ 5 m s$^{-1}$ across multiple (30) wedge re-alignments. The measurement of wedge-to-wedge time to be subtracted from the total $t_F$ is temperature dependant and relies on accurate alignment of the wedge feet, as well as a good connection between them (signal amplitude is highly dependant on couplant). Variation in alignment causes around a $\pm$ 10 m s$^{-1}$ velocity change. The wedge foot offset in Equation~\ref{velocitycalcfull} has a large effect on the calculated wave velocity. The exact offset distance is unknown and is assumed to be the point at which the centre line of the transducer aligns with the plate surface. Varying this value raises or lowers the velocity of all results considerably ($\pm$ 1 mm = $\pm$ 35 m s$^{-1}$). The hot plate does not heat the test plate evenly, especially at distances greater than 10 cm between wedges where they overhang the edges of the hot plate. The gradient (the difference in temperature between the centre of the plate and the location of the wedges) increases with temperature. The measured velocity is monitoring the average temperature of the transmission path. The gradient has been measured by placing a number (4) of thermocouples along the transmission path, from the centre of the plate (maximum temperature) to the point at which a wave is transmitted between a wedge foot and the plate. The calculation of aluminium dispersion curves at different temperatures is based on a change in Young's modulus. This is predicted from Hopkin's formula \cite{Hopkins2012} that may not give the correct values for Aluminium 1050 H14, but Aluminium in general.

The largest source of error is the measurement of wedge-to-wedge $t_F$, as a small error in alignment causes a large change ($\pm$ 10 m s$^{-1}$) in the wave velocity calculation. This is accounted for through the averaging of multiple (30) measurements, the standard deviation for this range is shown using error bars on Figure~\ref{fig:result}. Variation in wedge foot offset distance dramatically shifts the calculated velocity. This value cannot be directly measured and so the result relies on accurate calculation as discussed in the method.

\section{Conclusion}

The theoretical effect of temperature on various Lamb wave modes in aluminium plates has been investigated by generating dispersion curves based on varying material properties (Figure~\ref{fig:alutempshift}). This can be repeated in the future for other materials at higher temperatures (e.g.\ Inconel 718 up to 1100\si{\degreeCelsius} in Figure~\ref{fig:groupshift}). The temperature sensitivity of the $S_0$ mode at 1~MHz has been extracted from these curves (Figure~\ref{fig:alugroupvel}) to be validated experimentally.

An experimental investigation has been carried out in order to validate theoretical predictions. Wedge transducers in a pitch-catch configuration have been used to excite the $S_0$ mode in a 1mm thick aluminium plate. The time of flight between transducers has been measured using a cross-correlation method and wave velocity calculated based on the distance between transducers. This confirms that the $S_0$ mode has been excited. The change in $S_0$ wave velocity due to temperature is in line with theoretical predictions over the range 20\si{\degreeCelsius}--100\si{\degreeCelsius} as shown in Figure~\ref{fig:result}. 

It is clear that wedge transducers are not the optimum method of transmitting/receiving a wave through a nozzle guide vane at high temperatures. They cannot be permanently mounted to the structure due to the need for a liquid couplant, and their relatively large footprint would make finding a suitable mounting location a challenge. Their operation at high temperatures is limited by the wedge material, which in the case of acrylic melts at around 160\si{\degreeCelsius}. The wedge material needs to have a longitudinal wave velocity less than that of the targeted Lamb wave phase velocity, which limits the choice of material severely, mostly to plastics with low melting points. The great benefit of wedge transducers is the ability to selectively target Lamb wave modes, which reduces the complexity of data analysis compared with exciting multiple modes simultaneously. This is difficult to achieve using other transducer configurations but it may instead be possible to excite a higher order region that travels as a single wave packet. The use of PWAS transducers could allow for operation at high temperatures (assuming suitable choice of piezoelectric material) and would be relatively easy to mount to an NGV structure, having a small footprint, although the high temperatures are likely to make bonding difficult. Another option is to couple into the structure using waveguides, distancing the transducers from the high temperature environment. Future research will investigate the temperature sensitivity of higher order modes (such as $A_1$ and $S_1$), as operating at higher frequencies can improve resolution (allowing for the detection of smaller phase shifts) and response rates. The ability to monitor wave velocity variations in multi-modal wave packets will also be considered when investigating transducer configurations suitable for higher temperature operation.

\printbibliography
\chapter{Reflection experiments}

The feasibility of using acoustic reflections to monitor a change in temperature is explored in this chapter. Initial tests are carried out using reflections from the edge of a plate, moving on to a single hole, and finally to multiple holes.

\section{Edge reflections}

Measurements are carried out on a 1mm thick aluminium plate, targeting the $S_0$ mode at a frequency-thickness product of 1 MHz-mm. Two wedge transducers are arranged in a pulse-echo configuration, whereby the wedges are placed close to one another, one acting as the transmitter and one as the receiver. A single transducer setup was considered however the difference in amplitude between the excitation signal (10 V) and reflected signal ($\sim$50 mV) makes digitisation difficult, without either losing the reflected signal in noise or overloading the excitation input. By using two transducers the sensitivity of each channel can be adjusted to match the signal amplitude accordingly. 

Both transducers were placed equidistant from the edge of the plate, aligned so that angle of incidence from the transmitter wedge matches that of the angle of reflection for the receiver wedge. Multiple measurements were carried out after moving the wedges further away from the plate edge while ensuring that the distance between wedge centres remained the same. The angles of the wedges were adjusted to match the change in angle of incidence/refection for each distance. This experiment confirmed that reflections were measured rather than direct transmission between wedges.
\input{8 Research summary.tex}

%\printbibliography
\printbibliography
%\bibliographystyle{IEEEtran}
%\bibliography{library.bib, References.bib}
\end{document}