\chapter{Experimental investigation of single hole reflection amplitude}

The change in reflected signal amplitude with respect to hole size is investigated in this chapter. Two wedge transducers are arranged in a pulse-echo configuration, whereby the wedges are placed close to one another, one acting as the transmitter and one as the receiver. A single transducer setup was considered however the difference in amplitude between the excitation signal (10 V) and reflected signal ($\sim$50 mV) makes digitisation difficult, without either losing the reflected signal in noise or overloading the excitation input. By using two transducers the sensitivity of each channel can be adjusted to match the signal amplitude accordingly. 

\begin{figure}[ht]
    \centering
    \includegraphics[width=.8\textwidth]{./figures/topdownholespacer.eps}
    \caption{Diagram of single hole reflection measurement setup.}\label{fig:topdownholespacer}
\end{figure}
\newpage
\section{Determination of transmission distance}

As the calculation of group velocity is dependant on accurate knowledge of the distance between transducers, an initial test is carried out to determine the distance from transmitter to receiver via hole reflection. Firstly, the transducers are aligned on axis with no hole present, spaced apart by 100 mm (as in previous experiments) and time of flight is measured ($t_{\textrm{aligned}}$). The wedge-to-wedge time ($t_{\textrm{wedge}}$) and offset distance ($d_{\textrm{offset}}$) are also measured, as described in Chapter~\ref{experiments}. The group velocity ($v$) can now be calculated using Equation~\ref{eq:velocitytext}. Next, the time of flight from a reflection is measured ($t_{\textrm{reflected}}$), as shown in Figure~\ref{fig:topdownholespacer}. Now the distance travelled in the plate between transducer faces ($d_{\textrm{reflected}}$) can be calculated (Equation~\ref{eq:dreftext}) by multiplying the velocity ($v$) by the reflected time of flight ($t_{\textrm{reflected}}$) minus the wedge-to-wedge time of flight ($t_{\textrm{wedge}}$), minus the offset distance ($d_{\textrm{offset}}$). This distance can now be used in future tests to measure a change in velocity with temperature.  

%
\begin{equation} \label{eq:velocitytext}
    v = \left( \frac{d_{\textrm{aligned}} + d_{\textrm{offset}}}{t_{\textrm{aligned}} - t_{\textrm{wedge}}} \right)
    \tagaddtext{[\si{\meter\per\second}]}
\end{equation} 
%
\begin{equation} \label{eq:velocitynum}
    5099.18 = \left( \frac{0.1 + 0.04587}{5.812{\times}10^{-5} - 2.951{\times}10^{-5}} \right)
    \tagaddtext{[\si{\meter\per\second}]}
\end{equation} 
%
\begin{equation} \label{eq:dreftext}
    d_{\textrm{reflected}} = v \times \left(t_{\textrm{reflected}} - t_{\textrm{wedge}} \right) - d_{\textrm{offset}}
    \tagaddtext{[\si{\meter}]}
\end{equation} 
%
\begin{equation} \label{eq:drefnum}
    0.178 = 5099.18 \times \left(7.335{\times}10^{-5} - 2.951{\times}10^{-5} \right) - 0.04587
    \tagaddtext{[\si{\meter}]}
\end{equation} 
%

A 3D printed spacer has been designed to allow transducers to be accurately and repeatably aimed at a hole. The angle of incidence/reflection can easily be adjusted. 

In this configuration a reflection is received from the hole, followed by a stronger reflection from the edge of the plate. The angle between transducers is kept at a minimum to most closely mimic a pulse-echo configuration using only one transducer.

\newpage

\section{The effect of hole size on signal amplitude}

The setup shown in Figure~\ref{fig:topdownholespacer} is used to compare the difference in reflection amplitude between different hole sizes. The tests are carried out on three different plate thicknesses (1 mm, 2.5 mm, and 4 mm), targeting three different Lamb wave modes ($S_0$, $A_1$, and $S_1$ respectively). The signal amplitude is highly dependant on the coupling between wedges and plate, so fresh couplant is applied for every test, and an average amplitude is calculated after multiple removals and replacements of the wedges. The reflection angle (36\degree) and distance (0.178 m) between transducers is kept consistent across all measurements. A 60 dB voltage amplifier is applied to the receiver transducer. 

Reflection amplitude is measured at hole sizes of 1.5 mm, 2 mm, 2.5 mm, 3 mm, 3.2 mm, 3.5 mm, and 4 mm. Absolute peak amplitude is measured using MATLAB. The wedges were removed from the plate and the couplant was reapplied five times for each hole size, taking ten measurements per reapplication, for a total of fifty measurements per hole size. The edges of the holes were deburred on both sides of the plate after every new hole was drilled.

\subsection{$S_0$ Result}

\begin{figure}[ht]
    \centering
    \includegraphics[width=.8\textwidth]{./figures/holeampboxplot.eps}
    \caption{Boxplot showing change in reflection signal amplitude with hole size for the $S_0$ mode.}\label{fig:holeampboxplot}
\end{figure}

Results for the $S_0$ mode are shown in Figure~\ref{fig:holeampboxplot}. The mean is shown in green. There is a linear relationship (R$^2$~=~0.9994) between increasing hole size and increasing amplitude. The average increase in amplitude for a \diameter~0.5~mm change in hole size is 0.216~V $\pm$ 0.014~V.

\newpage

\subsection{$A_1$ Result}

\begin{figure}[ht]
    \centering
    \includegraphics[width=.8\textwidth]{./figures/holeampboxplotA1.eps}
    \caption{Boxplot showing change in reflection signal amplitude with hole size for the $A_1$ mode.}\label{fig:holeampboxplotA1}
\end{figure}

Results for the $A_1$ mode are shown in Figure~\ref{fig:holeampboxplotA1}. The mean is shown in green. There is a linear relationship (R$^2$~=~0.9981) between increasing hole size and increasing amplitude. The average increase in amplitude for a \diameter~0.5~mm change in hole size is 0.034~V $\pm$ 0.05~V.

\newpage

\subsection{$S_1$ Result}

\begin{figure}[ht]
    \centering
    \includegraphics[width=.8\textwidth]{./figures/holeampboxplotS1.eps}
    \caption{Boxplot showing change in reflection signal amplitude with hole size for the $S_1$ mode.}\label{fig:holeampboxplotS1}
\end{figure}

Results for the $S_1$ mode are shown in Figure~\ref{fig:holeampboxplotS1}. The mean is shown in green. There is a linear relationship (R$^2$~=~0.9909) between increasing hole size and increasing amplitude. The average increase in amplitude for a \diameter~0.5~mm change in hole size is 0.027~V $\pm$ 0.03~V.

\subsection{Conclusions}

The $S_0$ mode is significantly more sensitive to changes in hole size than the $A_1$ or $S_1$ modes. The large in-plane displacement of the $S_0$ mode (Figure~\ref{fig:S0throughthickness}) is very sensitive to defects at any depth (such as holes), whereas the $A_1$ and $S_1$ modes (Figures~\ref{fig:A1throughthickness} and~\ref{fig:S1throughthickness} respectively) exhibit a lower sensitivity due to their middle thickness nodes. Results are in line with those reported by Jeong \textit{et al.}~\cite{Jeong2000}.

The increased sensitivity of the $S_0$ mode may be advantageous in differentiating between different hole sizes on an NGV, where the holes along the leading edge of the vane may use a different geometry than the rest of the holes.

\newpage

\begin{figure}[ht]
    \centering
    \includegraphics[width=.7\textwidth]{./figures/S0throughthickness.eps}
    \caption{Through thickness displacement profile of the $S_0$ mode.}\label{fig:S0throughthickness}
\end{figure}

\begin{figure}[ht]
    \centering
    \includegraphics[width=.7\textwidth]{./figures/A1throughthickness.eps}
    \caption{Through thickness displacement profile of the $A_1$ mode.}\label{fig:A1throughthickness}
\end{figure}

\begin{figure}[ht]
    \centering
    \includegraphics[width=.7\textwidth]{./figures/S1throughthickness.eps}
    \caption{Through thickness displacement profile of the $S_1$ mode.}\label{fig:S1throughthickness}
\end{figure}