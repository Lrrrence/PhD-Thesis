\chapter{Reflection experiments}

The feasibility of using acoustic reflections to monitor a change in temperature is explored in this chapter. Initial tests are carried out using reflections from the edge of a plate, moving on to a single hole, and finally to multiple holes.

\section{Edge reflections}

Measurements are carried out on a 1mm thick aluminium plate, targeting the $S_0$ mode at a frequency-thickness product of 1 MHz-mm. Two wedge transducers are arranged in a pulse-echo configuration, whereby the wedges are placed close to one another, one acting as the transmitter and one as the receiver. A single transducer setup was considered however the difference in amplitude between the excitation signal (10 V) and reflected signal ($\sim$50 mV) makes digitisation difficult, without either losing the reflected signal in noise or overloading the excitation input. By using two transducers the sensitivity of each channel can be adjusted to match the signal amplitude accordingly. 

Both transducers were placed equidistant from the edge of the plate, aligned so that angle of incidence from the transmitter wedge matches that of the angle of reflection for the receiver wedge. Multiple measurements were carried out after moving the wedges further away from the plate edge while ensuring that the distance between wedge centres remained the same. The angles of the wedges were adjusted to match the change in angle of incidence/refection for each distance. This experiment confirmed that reflections were measured rather than direct transmission between wedges.