\chapter{Conclusions and future research}
\section{Conclusions}

A comprehensive literature review was undertaken to identify the currently used temperature monitoring methods for nozzle guide vanes. Within the online monitoring sector there is scope for the development of a new sensor that can provide spatial data without influencing component or system operation. An ultrasonic guided wave based system was identified as a potential alternative, and further investigations were undertaken to understand the effect of temperature on Lamb wave propagation, and the ability to monitor temperature at different locations through acoustic reflections.

The feasibility of using ultrasonic guided waves for temperature monitoring has been considered by analysing potential sensor options, piezoelectric materials for high temperature operation, temperature effects on material properties, and the most appropriate mode for the application. 

The temperature influence study consisted of experimental work to measure the temperature sensitivity of three Lamb wave modes ($S_0$, $A_1$, and $S_1$) in aluminium plates using wedge transducers to selectively target each mode of interest. The results of the experimental work was used to validate a finite element model of the test setup, to be used in further studies. The model was then used as a feasibility study for extending the temperature range up to 1500\si{\degreeCelsius} for an Inconel 718 plate, which is a Nickel-based super alloy typically used for high temperature jet engine components.

The reflection study consisted of a number of elements. Firstly, an experimental study of reflection amplitude from different sized holes, repeated for  three Lamb wave modes ($S_0$, $A_1$, and $S_1$) in a pulse-echo configuration. This study showed the comparatively higher sensitivity to hole size for the $S_0$ mode in comparison to the $A_1$ and $S_1$ modes, which may be advantageous in discriminating between reflections from different areas of nozzle guide vanes. A similar study was repeated in a pitch-catch configuration.

A three dimensional finite element model has been developed to investigate more complex cooling hole configurations at higher temperatures than the experimental equivalent.

One of the biggest challenges to overcome for the implementation of such as system is in sensor configuration. The limited space and access points for a vane as well as the high temperatures severely limit the available options, however there are a number of piezoelectric materials that can operate under high temperature conditions, and wave guides can be used to distance the sensors from the gas path. Hertzian contact points are a viable option for coupling into the structure. 

Limiting excitation to a single mode would be preferable, however there are a limited number of options available for this application. Two-sided excitation is an interesting option for preferentially exciting either symmetric or antisymmetric modes, however it has been shown through simulation that this method is ineffective in anisotropic layers, such as a substrate with TBC applied to one surface only.

Although the work carried out looking at a single material (Aluminium or Inconel 718) looked to have promising advantages to working at higher order modes (above $S_0$), the application of TBCs complicates signal propagation, causing there to be less suitable areas of the spectrum (based on dispersion curve analysis) available for use. That is not to say that specific combinations of material properties and layer thicknesses cannot produce more desirable areas to target, however. Operating below the cut-off frequency of $B_2$ is likely to be the most effective method of targeting/identifying single modes.   

\section{Future research}

There is great scope for further research in this field:

\begin{itemize}
    \item Sensor design.
    \item Signal processing systems.
    \item Effects of complex geometry (curved surfaces) on wave propagation.
    \item Experimental testing of a system on an NGV.
    \item Application of technology to other systems.
\end{itemize}