\chapter{Conclusions and future research}
\section{Conclusions}
\subsection{Research summary}

A comprehensive literature review has been undertaken to identify the currently used temperature monitoring methods for nozzle guide vanes. Both offline and online systems were evaluated. Within the online monitoring sector there is scope for the development of a new sensor that can operate without influencing component or system operation. An ultrasonic guided wave based system was identified as a potential alternative, and further investigations were undertaken to consider the viability of using such a system.

The feasibility of using ultrasonic guided waves for temperature monitoring has been investigated by considering other applications of the technology, analysing potential sensor options, considering the range of piezoelectric materials suitable for high temperature operation, considering how temperature affects material properties, and determining the most appropriate mode for the application. The effect of cooling hole arrays and thermal barrier coatings on wave propagation have been investigated through simulations.

The temperature influence study consisted of experimental work to measure the temperature sensitivity of three Lamb wave modes ($S_0$, $A_1$, and $S_1$) in aluminium plates using wedge transducers to selectively target each mode of interest. The results of the experimental work were used to validate a finite element model of the test setup, to be used in further studies. The model was then used as a feasibility study for extending the temperature range up to 1500\si{\degreeCelsius} for an Inconel 718 plate, which is a Nickel-based super alloy typically used for high temperature jet engine components.

The thermal barrier coating study has considered the range of materials, make up, and application methods typically used, and evaluated the effect of temperature on their structure. Dispersion curves generated for the multi-layered composite show how the higher order modes increase in complexity in comparison with the response in a single material. Through-thickness displacement varies across the thickness as material properties vary, with the top coat often exhibiting considerably larger displacement than then other layers. Although the work carried out looking at a single material (Aluminium or Inconel 718) looked to have promising advantages to working at higher order modes (above $S_0$), the application of TBCs complicates signal propagation, causing there to be less suitable areas of the spectrum (based on dispersion curve analysis) available for use. That is not to say that specific combinations of material properties and layer thicknesses cannot produce more desirable areas to target, however. Operating below the cut-off frequency of $B_2$ is likely to be the most effective method of targeting/identifying single modes.   

The reflection study consisted of a number of elements. Firstly, an experimental study of reflection amplitude from different sized holes, repeated for three Lamb wave modes ($S_0$, $A_1$, and $S_1$) in a pulse-echo configuration. This study showed the comparatively higher sensitivity to hole size for the $S_0$ mode in comparison to the $A_1$ and $S_1$ modes, which may be advantageous in discriminating between reflections from different areas of nozzle guide vanes. A similar study was repeated in a pitch-catch configuration.

A three dimensional finite element model has been developed to investigate more complex cooling hole configurations at higher temperatures than the experimental equivalent.

\subsection{General recommendations}

Considering the wide range of superalloy materials used for NGVs, the range of thermal barrier coating composites, and cooling hole configurations, there is not a frequency or mode applicable to all cases, however there are a number of general recommendations that can be made. 

In terms of material properties it has been shown that changes in temperature have the largest impact on Young's modulus, in comparison to Poisson's ratio and density (thermal expansion). This is true even for large temperature ranges, although the effect on Poisson's ratio becomes more influential at temperatures closer to the thermal limits of materials. The materials used for NGVs exhibit considerable changes in Young's modulus over the temperature ranges of interest, which has a measurable effect on wave velocity.

There are a number of factors involved that determine the most suitable operating mode and frequency. Firstly, the ability to identify a single mode in a multi-modal wave packet is critical to time of flight measurement. This can be achieved by attempting to separate the mode of interest in the time domain, by exciting a mode that is substantially faster or slower than other modes at the same frequency, or by selecting a mode that is likely to have a substantially larger amplitude than neighbouring modes. The mode of interest should be targeted at it's group velocity maxima to ensure time domain separation, however the bandwidth of the peak should also be considered, as high temperatures will cause the peak to shift, potentially into a more dispersive region with other modes present.

The lowest two modes, $A_0$ and $S_0$, propagate with the largest amplitude in comparison to higher order modes, which can be seen by analysing through thickness displacement profiles, as well as considering that the antisymmetric modes exhibit larger out of plane motion in comparison to symmetric modes, which is more easily detected by piezoelectric transducers. The symmetric modes are less effected by surface defects, and less scattering occurs from features such as holes.

The next factor to consider is wavelength in relation to cooling hole diameter and spacing. The A0 mode has a substantially shorter wavelength than $S_0$ at the same frequency, which aides in peak detection, as well as producing more easily detectable reflections. Operating higher in frequency is advisable, however the low frequency regions of each higher order mode (closer to their cut-off frequency) still exhibit relatively long wavelengths. The likelihood of mode conversion is also determined by wavelength in relation to hole diameter and spacing.

Temperature sensitivity is determined by both mode, and the dispersiveness of the curve at the excitation frequency. $A_0$ is the least sensitive mode to temperature because of its relatively low dispersion, however this can be advantageous in ensuring a clearly detectable peak even at high temperatures. $S_0$ on the other hand can provide a considerably higher temperature sensitivity in steep regions of its group velocity dispersion curve, while only exhibiting small amounts of dispersion to a wave packet propagating over a short distance. 

Temperature monitoring using guided waves can be thought of as providing an average of the through thickness temperature, considering that waves are guided by both surfaces. Lower frequency modes exhibit changes in displacement throughout their thickness, whereas higher order modes have greater displacement towards either surface. This implies that lower order modes provide a better average, whereas the higher order modes are more affected by surface temperatures.

Thermal barrier coatings mostly complicate the higher order mode regions, making it unlikely that a higher order mode can be identified over the whole temperature range of interest. It may still be possible to pick out $B_0$ or $B_1$ at higher frequencies however, in areas where the other modes are particularly dispersive and therefore don't exhibit strong peaks or large amplitudes. The lowest order modes exhibit through thickness displacement profiles comparable with $A_0$ and $S_0$. When analysing the material properties of TBCs for the generation of dispersion curves, careful consideration should be made for the application method, as it has a significant impact on the Young's modulus, particularly for the top coat. 

Response time is limited by excitation frequency, pulse width, and received wave packet length. If a long pulse width is used, or a large number of modes are excited, then a longer time between pulses is required to ensure that energy has dissipated before repeating a pulse. Operating at a higher frequency and/or modes with a shorter wavelength is advantageous for improving response time. 

Temperature resolution is limited by a combination of mode sensitivity, sampling rate, and pulse width (frequency, wavelength, and number of cycles). Assuming a high enough sampling rate then uncertainty is caused by the difficulty in detecting the true peak of a wave packet, which itself is determined by the number of cycles, and frequency/wavelength. It should be noted that although reducing the number of cycles is advantageous in this situation, it also causes a reduction in amplitude for a low number of cycles (<5), as well as an increase in bandwidth which could cause more modes to be excited than intended. Mode sensitivity can change with temperature as the excitation frequency falls on a more or less dispersive region. Targeting more dispersive regions (steeper) is advantageous for maximising temperature sensitivity, however time of flight accuracy is likely to be impacted if wave packets have less of a defined peak.

The effect of complex curved surfaces on wave propagation is yet to be investigated however the 3D model developed for the cooling hole study could be further adapted for this task. 

One of the biggest challenges to overcome for the implementation of such as system is in sensor configuration. The limited space, access points for a vane, power constraints, as well as the high temperatures, severely limit the available options. Ideally the transducer configuration developed for this application should be able to achieve single mode excitation, however this is unlikely considering that wedge transducers are unsuitable for permanent installation, and EMATS have a large power consumption. Two-sided excitation is the next best alternative, as it reduces the number of modes excited by half, although this may not be possible in an asymmetric composite such as a NGV with TBC applied to the outer surface. Operation at high temperatures is challenging for piezoelectric transducers, however a number of materials have been identified that can used in these conditions, or the sensors could be distanced from the harsh conditions of the gas path by the use of waveguides. This may have the added benefit of allowing adequate time/distance between excitation and the first reflection from a cooling hole. 

\section{Future research}

There is great scope for further research in this field:

\begin{itemize}
    \item Sensor design.
    \item Signal processing systems.
    \item Effects of complex geometry (curved surfaces) on wave propagation.
    \item Experimental testing of a system on an NGV.
    \item Application of technology to other systems.
\end{itemize}

Such a monitoring system has potential for other applications, especially if the installation of traditional sensors are impractical. For example, printed circuit boards will require such a technology for safety critical applications in the future, where the temperature monitoring of components would otherwise require a large number of individual sensors. Exciting and receiving waves from the edges of a board would free up precious board space, and potentially allows for otherwise impractical online monitoring. Another potential application is within batteries, which again are becoming increasingly prevalent in safety critical applications such as self driving cars and other autonomous systems. 