\chapter{Temperature sensitivity of guided waves}\label{chapter:tempsens}

In this chapter the temperature sensitivity of guided waves is investigated through prediction, experimentation, and simulation. Aluminium was chosen as a test material, as it's readily available for experimentation, and temperature dependant material properties for the particular grade of Aluminium (H14 1050) can be found in literature. Individual modes have been isolated through the use of variable angle wedges, to allow each mode to be investigated separately.

The changes in material properties of aluminium with respect to temperature are discussed below. The change in density (thermal expansion) and Poisson's ratio is negligible over the temperature of interest in this study (20\si{\degreeCelsius}--100\si{\degreeCelsius}) in comparison to the change in Young's modulus.

\subsection{Thermal expansion}
Thermal expansion will cause a change in material dimensions which will also affect the wave speed, although the effect is small. The coefficient of thermal expansion, $\alpha$, for Aluminium 1050 at 20\si{\degreeCelsius} is 2.3$\times 10^{-5}$, rising to 2.5$\times 10^{-5}$ at 100\si{\degreeCelsius}, as described by the equation:
%
\begin{equation}
    \alpha (T) = 1.243109{\times}10^{-5}+5.050772{\times}10^{-8}\times T^1-5.806556{\times}10^{-11}\times T^2+3.014305{\times}10^{-14} \times T^3
\end{equation}
%
Where $T$ is temperature in Kelvin, the equation is valid from 230 K to 900 K~\cite{Nix1941,Feder1958,Gibbons1958}.

The change in time of flight due to thermal expansion can be expressed using the equation~\cite{Croxford2007}:
%
\begin{equation}
    \delta t = \frac{d}{v} \left(\alpha - \frac{k}{v}\right)\delta T
\end{equation}
%
Where $d$ is the propagation distance (0.1 m), $v$ is the group velocity, and $k$ is the temperature sensitivity of group velocity, $\delta v/\delta T = k$. The group velocity can then be recalculated with the effect of thermal expansion included. The average change in velocity due to thermal expansion over the temperature range of interest (20\si{\degreeCelsius}--100\si{\degreeCelsius}) is -1.59 m s$^{-1}$ for the $S_0$ mode, -0.89 m s$^{-1}$ for the $A_1$ mode, and -1.47 m s$^{-1}$ for the $S_1$ mode. The change is sufficiently small that thermal expansion can be excluded from the COMSOL models used in the following chapters.

\subsection{Poisson's ratio}

The Poisson's ratio, $\nu$, for Aluminium 1050 at 20\si{\degreeCelsius} is 0.3310, rising to 0.3325 at 100\si{\degreeCelsius}, as described by the equation~\cite{Lalpoor2009}:
%
\begin{equation}
\begin{split}
    \nu (T) = 0.3238668+3.754548{\times}10^{-6}\times T^1+2.213647{\times}10^{-7}\times T^2-6.565023{\times}10^{-10}\times T^3 \\ +~4.21277{\times}10^{-13}\times T^4+3.170505{\times}10^{-16}\times T^5
\end{split}
\end{equation}
%
Where $T$ is temperature in Kelvin, the equation is valid from 0 K to 773 K. 

This is equivalent to a velocity change of 0.33 m s$^{-1}$ for the $S_0$ mode at 100\si{\degreeCelsius} assuming no change to density or Young's modulus.

\subsection{Young's modulus}
The change in Young's modulus with temperature in Aluminium is represented by the Equation~\cite{Sakai1996,McLellan1987,Stokes1960,Naimon1975}: 
%
\begin{equation}
    E\left( T \right) = 7.770329{\times}10^{10} + 2036488.0 \times T^1 - 189160.7\times T^2+425.2931 \times T^3-0.3545736 \times T^4 \label{eq:aluE}
\end{equation}
%
Where $T$ is temperature in Kelvin, and $E$ is Young's modulus in Pascals. 

This represents the largest change in wave velocity with respect to the other material properties. The $S_0$ mode at 1 MHz reduces in velocity by $\sim$150\unit{\meter\per\second} over the temperature range 20\si{\degreeCelsius}--100\si{\degreeCelsius}.

\section{Prediction study (Aluminium)}

The temperature dependant dispersion curves and through thickness displacement profiles in this section were generated using \href{https://www.dlr.de/zlp/en/desktopdefault.aspx/tabid-14332/24874_read-61142/#/gallery/33485}{The Dispersion Calculator}~\cite{Huber}. The change in density (thermal expansion) and Poisson's ratio is ignored for this study, as the temperature range (20-100~\si{\degreeCelsius}) is small. The change in Young's modulus with temperature is described by~\cref{eq:aluE}.

Temperature dependant dispersion curves for aluminium 1050 H14 are shown in Figure~\ref{fig:grouptempshuftalu}. The angles of excitation required to excite the $S_0$, $A_1$, and $S_1$ modes are shown based on the frequency-thickness products of 1~MHz-mm (28\degree), 2.5~MHz-mm (21\degree), and 4~MHz-mm (25\degree) respectively. These frequency-thickness products correspond closely to group velocity maxima for each of the targeted modes, which is advantageous for separating the modes in the time domain~\cite{Alleyne1992}.

The group velocity of the $S_0$, $A_1$, and $S_1$ modes have been extracted from the curves at frequency-thickness products of 1~MHz-mm, 2.5~MHz-mm, and 4~MHz-mm respectively. The extracted group velocities are shown in Figure~\ref{fig:alugroupvel}. The temperature sensitivities of the modes are: 1.45--1.52 m s$^{-1}$\si{\degreeCelsius}$^{-1}$ for the $S_0$ mode, 0.77--0.87 m s$^{-1}$\si{\degreeCelsius}$^{-1}$ for the $A_1$ mode, and 1.37--1.45 m s$^{-1}$\si{\degreeCelsius}$^{-1}$ for the $S_1$ mode, over the temperature range 10--110\si{\degreeCelsius} (see Equation~\ref{eqn: eq k}). These values are frequency and mode dependent. The temperature sensitivities increase with temperature, as the modes reduces in frequency (shift left), and reduces in wave velocity (shift down). 

Through thickness displacement profiles for each of the modes targeted are shown in \cref{fig:Aludisplacement}. The $S_0$ mode (\cref{fig:S0throughthickness}) shows comparatively larger in-plane motion than the other modes. Although the antisymmetric modes are associated with out-of-plane motion, the area of the spectrum targeted in this study exhibits larger in-plane motion (\cref{fig:A1throughthickness}). The $S_1$ mode shows a similar amount of displacement to the $A_1$ mode (\cref{fig:S1throughthickness}).    

\clearpage
\begin{figure}[p]
    \centering
    \begin{subfigure}[c]{0.75\textwidth}
        \centering
        \includegraphics[width=1\textwidth]{./figures/grouptempshift_new.eps}
        \caption{$A_0$, $S_0$, $A_1$, and $S_1$ group velocity dispersion curve shift with temperature from 20\si{\degreeCelsius} to 100\si{\degreeCelsius} for Aluminium 1050 H14.}\label{fig:grouptempshuftalu}
    \end{subfigure}
    \vskip\baselineskip
    \begin{subfigure}[c]{0.75\textwidth}
        \centering
        \includegraphics[width=1\textwidth]{./figures/1mhzshift_new.eps}
        \caption{$S_0$, $A_1$, and $S_1$ group velocity change with temperature from 10\si{\degreeCelsius} to 110\si{\degreeCelsius} for Aluminium 1050 H14.}\label{fig:alugroupvel}
    \end{subfigure}
    \caption{The effect of temperature on Lamb wave propagation for Aluminium 1050 H14.}
\end{figure}

\clearpage

\begin{figure}[p]
    \raggedright
        \begin{subfigure}[c]{0.48\textwidth}
            \centering
            \includegraphics[width=1\textwidth]{./figures/S0throughthickness.eps}
            \caption{$S_0$ Displacement.}\label{fig:S0throughthickness}
        \end{subfigure}
        \hfill
        \begin{subfigure}[c]{0.48\textwidth}
            \centering
            \includegraphics[width=1\textwidth]{./figures/A1throughthickness.eps}
            \caption{$A_1$ Displacement.}\label{fig:A1throughthickness}
        \end{subfigure}
        \vskip\baselineskip
        \begin{subfigure}[c]{0.65\textwidth}
            \centering
            \includegraphics[width=1\textwidth]{./figures/S1throughthickness.eps}
            \caption{$S_1$ Displacement.}\label{fig:S1throughthickness}
        \end{subfigure}
    \caption{Through thickness displacement profiles for Aluminium.}\label{fig:Aludisplacement}
\end{figure}

\clearpage

\section{Experimental study (Aluminium)}\label{experimentalstudy}

The group velocity temperature sensitivity of the $S_0$, $A_1$, and $S_1$ Lamb wave modes have been experimentally measured, to be validated against values extracted from dispersion curves.

Time of flight through a plate is measured, and group velocity calculated from the propagation distance. Variable angle acrylic wedges are used to preferentially excite the modes of interest, acting as both transmitters and receivers, in a pitch-catch configuration. Two piezoelectric transducers centred at 1~MHz are used for excitation/detection. A liquid couplant is used to couple the wedges to the plate. The thickness of the aluminium plate is varied to target different areas of the frequency-thickness spectrum. A 1~mm thick plate is used to target the $S_0$ mode at 1~MHz, a 2.5~mm plate to target the $A_1$ mode, and a 4~mm thick plate to target the $S_1$ mode.

A signal generator (GW Instek MFG-2230M) has been used to generate a 5-cycle Hamming windowed pulse at 1~MHz. Signals are digitised using a Picoscope 3406DMSO USB Oscilloscope. Based on a sampling rate of 5$\times$10$^8$ the theoretical maximum temporal resolution is 2~ns. Signal processing is carried out in MATLAB. A zero-phase bandpass filter is applied to the signals to remove unwanted noise. The bandwidth of the filter matches the bandwidth of the excitation pulse, 0.6-1.4 MHz. Time of flight is measured through generating envelopes of the transmitted/received wave packets, finding the peaks of those envelopes, and calculating the time between them. The calculation of group velocity is dependant on removing the propagation time through the wedges, leaving only the time through the plate. This method of time of flight measurement can also be applied to more dispersive signals, which cannot be achieved using cross correlation methods. It should be noted, however, that more dispersive signals (e.g. $A_1$ \& $S_1$ Vs. $S_0$) have a less defined central peak in their wave packets, which introduces a larger degree of error to the calculation of time of flight. Various signal processing techniques for time of flight measurement are discussed in detail by Guers~\cite{Guers2011}.

The temperature of the aluminium plate is controlled using a hot plate, and is increased from room temperature up to 100\si{\degreeCelsius}, in $\sim$10\si{\degreeCelsius} increments. As this method of heating does not produce a consistent temperature across the whole plate or in the wedges, an average temperature is calculated after measuring temperature using multiple thermocouples placed along the transmission path. 

\vfill
\begin{table}[h]
    \centering
    \begin{tabulary}{\textwidth}{L}
        \hline
        \textbf{Measurement Hardware}     \\
        \hline
        2x Olympus ABWX-2001 Variable angle wedges \\
        2x Olympus A539S-SM 1 MHz transducers \\
        Olympus ultrasonic couplant B \\
        GW Instek MFG-2203M Signal generator \\
        Picoscope 3406DMSO USB Oscilloscope \\
        Thermadata T-type temperature loggers \\
        VWR Hot plate \\
        \hline
    \end{tabulary} 
    \caption{Experimental measurement hardware.}\label{table:hardware}
\end{table}
\vfill

\clearpage

The study is made up of has three steps, step one is the measurement of time of flight through the plate, at a distance of 10 cm between wedges, repeated at each temperature step, as shown in \cref{fig:testdiagramtotal}.

\begin{figure}[h]
    \centering
    \includegraphics[width=.9\textwidth]{./figures/testdiagramsimple.eps}
    \caption{Cross-sectional diagram of total time-of-flight measurement setup.}\label{fig:testdiagramtotal}
\end{figure}

The second step is the measurement of wedge-to-wedge time of flight. This is repeated at each temperature step, immediately after total time of flight measurement, as shown in \cref{fig:testdiagramw2w}.

\begin{figure}[h]
    \centering
    \includegraphics[width=.7\textwidth]{./figures/w2wdiagram.eps}
    \caption{Cross-sectional diagram of wedge-to-wedge time-of-flight measurement setup.}\label{fig:testdiagramw2w}
\end{figure}

The third step is the measurement of time of flight through the plate at varying distances, for calculation of wedge foot offset. This is only carried out at room temperature.

Group velocity is calculated using Equation~\ref{velocitycalc}/\ref{velocitycalcfull}. The propagation time through the wedges ($t_{\textrm{wedge}}$) as measured using the configuration shown in Figure~\ref{fig:testdiagramw2w}, has been subtracted from the time of flight $t_{\textrm{aligned}}$ to ensure that only the propagation time through the plate is measured.
%
\begin{equation} \label{velocitycalc}
    v = \frac{d}{t_F}
    \tagaddtext{[\si{\meter\per\second}]}
\end{equation} 
%
\begin{equation} \label{velocitycalcfull}
    v = \left( \frac{d_{\textrm{aligned}} + d_{\textrm{offset}}}{t_{\textrm{aligned}} - t_{\textrm{wedge}}} \right)
    \tagaddtext{[\si{\meter\per\second}]}
\end{equation} 
%
\clearpage

The wedge foot offset ($d_{\textrm{offset}}$) is an unknown distance from the front edge of the wedge to where the wave enters the plate from the wedge. This distance must be accounted for in order to accurately calculate group velocity. This distance has been calculated by measuring wave velocity at room temperature for the mode of interest at multiple wedge spacings (0.08 m to 0.14 m in 0.01 m increments) and looping through a range of plausible offset distances (0.02 m to 0.06 m in 0.00001 m increments) until the standard deviation across the range of wedge spacings is at a minimum. This ensures that the variation in measurement results is due to measurement error (e.g. small variances in setting the distance between wedges) rather than an incorrect estimation of wedge foot offset. This value varies with wedge angle and is recalculated for each wave mode measured.

The wedges allow for careful selection of excitation angle so that modes of interest can be targeted. The angle is determined based on Snell's law:
%
\begin{equation} 
\text{Angle}\ \theta = \text{Sin}^{- 1} \left( \frac{\text{Longitudinal\ wedge\ velocity}}{\text{Lamb\ wave\ phase\ velocity}} \right)
\end{equation} 
%
The specific angle required to excite each mode is detailed in \cref{S0 experiments} for $S_0$, \cref{A1 experiments} for $A_1$, and \cref{S1 experiments} for $S_1$.

Measurement of wave velocity depends on measurement of time of flight ($t_F$), which can be described by the Equation~\cite{Croxford2007a}:
%
\begin{equation}\label{tofcalc}
t_F = \frac{d}{c}
\end{equation} 
%
Where $d$ is the distance travelled at wave speed $c$, both of which are functions of temperature, $T$. The sensitivity of the time of flight to temperature can then be expressed as:
%
\begin{equation} 
\delta t_{F} = \frac{d}{c}\left( \alpha - \frac{k}{c} \right) \delta \text{T}
\end{equation} 
%
Where $\alpha$ is the coefficient of thermal expansion of the medium and $k$ is the rate of change of wave velocity with temperature:
%
\begin{equation} {\label{eqn: eq k}}
k = \frac{\delta \text{c}}{\delta \text{T}}
\end{equation} 
%

\clearpage
\subsection{Test Method}

An aluminium test plate is placed on top of a temperature controlled hot plate. Double sided tape is used to centre the plate and ensure that it does not move during the tests. Couplant is applied to the base of the wedges and they are placed onto the plate, aligned with a 3D printed spacer. The temperature of the aluminium plate is monitored using a thermocouple placed in the centre of the plate at the hottest point. The transmission path is set to pass under the thermocouple location, with the wedges equidistant from this position. Time of flight through the plate ($t_{\textrm{aligned}}$) is measured until it stabilises using the test setup shown in Figure~\ref{fig:testdiagramtotal}. This ensures that the temperature gradient in the wedges has stabilised. Time of flight is now measured for the set temperature. Multiple measurements are taken after adjusting both wedge positions. The wedges are removed from the surface and placed together to measure the wedge-to-wedge time ($t_{\textrm{wedge}}$) as shown in Figure~\ref{fig:testdiagramw2w}. Multiple measurements are taken after adjusting wedge-to-wedge position. The measurement process is repeated after allowing the time of flight to re-stabilise. Velocity is calculated using Equation~\ref{velocitycalcfull}. A mean average is calculated from the results of the repeated time of flight measurements, and velocity is calculated for every wedge-to-wedge result. An average velocity is calculated along with standard deviation. 

The temperature gradient across the plate has been measured by placing four equally spaced thermocouples along the transmission path, from the centre of the plate to the furthest edge of a wedge transducer in 3 cm increments. The wedges are removed from the plate to place the thermocouples, and the temperature of the hot plate is raised to match the temperature recorded by the thermocouple placed in the centre of the plate during measurement of time of flight. Measurements are repeated after moving the thermocouples to the other half of the transmission path. A mean average temperature has been calculated for the total transmission path at each hot plate temperature setting. 

\vfill
\begin{figure}[ht]
    \centering
    \includegraphics[width=.9\textwidth]{./figures/expsetup.png}
    \caption{Photograph of test setup.}\label{fig:testsetup}
\end{figure}
\vfill

\clearpage
\subsection{S0 mode (1 MHz-mm)}\label{S0 experiments}
%
The wedge angle required for the $S_0$ mode is:
%
\begin{equation} 
27.9{^\circ} = \text{Sin}^{- 1} \left( \frac{2477}{5298} \right)
\end{equation} 
%
The $A_0$ mode cannot be excited using this method as the phase velocity at this frequency (2312 m s$^{-1}$) is slower than the longitudinal velocity of the wedge. If the $A_0$ mode is present in the signal it will not affect measurement of the $S_0$ mode as it's group velocity is significantly different than that of the $S_0$ mode, which will cause a distinct second wave packet. 

\vfill
\begin{figure}[h]
    \centering
    \includegraphics[width=.8\textwidth]{./figures/s0xcorr5cyc.eps}
    \caption{Time of flight ($t_F$) measurement of $S_0$ mode wave propagation using cross-correlation function at room temperature.}\label{fig:S0crosscorr}
\end{figure}
\vfill

\clearpage

\subsubsection{S0 mode results}

Figure~\ref{fig:result} shows experimentally measured wave velocity of the $S_0$ mode plotted against theoretical wave velocity extracted from dispersion curves. Error bars show the standard deviation from the mean. After accounting for the temperature gradient across the transmission path by calculating a temperature average the change in velocity is comparable with predicted velocity extracted from dispersion curves, within 4.89 $\pm$ 2.27 m s$^{-1}$ on average. The temperature sensitivity of the system is 1.26--1.78 m s$^{-1}$\si{\degreeCelsius}$^{-1}$ over the range 24\si{\degreeCelsius}--94\si{\degreeCelsius}. The sensitivity is extracted from a second-order polynomial fit of the data (r$^2$ = 0.9992). The slope away from predicted results (increasing with temperature) can be attributed to the increasing temperature gradient, both in the plate and in the wedges. The gradient is shown to be almost linear (r$^2$=0.9967) across the measurement distance. Increasing temperature is also likely to have an effect on the operation of the piezoelectric transducer (amplitude and centre frequency), however this effect is negligible over the tested temperature range. The wedge angle required to excite the $S_0$ mode will also vary with temperature, however the change is only around 1\si{\degree} between 20\si{\degreeCelsius} and 100\si{\degreeCelsius}.

\vfill
\begin{figure}[h]
    \centering
    \includegraphics[width=.9\textwidth]{./figures/aluplatemeasured.eps}
    \caption{Group velocity change with temperature for the $S_0$ mode in Aluminium 1050 H14.}\label{fig:result}
\end{figure}
\vfill

\clearpage 

\subsection{A1 mode (2.5 MHz-mm)}\label{A1 experiments}

The wedge angle required for the $A_1$ mode is:
%
\begin{equation} 
21.3{^\circ} = \text{Sin}^{- 1} \left( \frac{2477}{6808} \right)
\end{equation} 
%
\vfill
\begin{figure}[h]
    \centering
    \includegraphics[width=.9\textwidth]{./figures/a1xcorr5cyc.eps}
    \caption{Time of flight ($t_F$) measurement of $A_1$ mode wave propagation using cross-correlation function at room temperature.}\label{fig:A1crosscorr}
\end{figure}
\vfill

\clearpage

\subsubsection{A1 mode results}

Figure~\ref{fig:A1result} shows experimentally measured wave velocity of the $A_1$ mode plotted against theoretical wave velocity extracted from dispersion curves. Error bars show the standard deviation from the mean. After accounting for the temperature gradient across the transmission path by calculating a temperature average the change in velocity is comparable with predicted velocity extracted from dispersion curves, within 2.43 $\pm$ 1.97 m s$^{-1}$ on average. The temperature sensitivity of the system is 1.09--1.17 m s$^{-1}$\si{\degreeCelsius}$^{-1}$ over the range 26\si{\degreeCelsius}--97\si{\degreeCelsius}. The sensitivity is extracted from a second-order polynomial fit of the data (r$^2$ = 0.9990).

\vfill
\begin{figure}[h!]
    \centering
    \includegraphics[width=.9\textwidth]{./figures/a1moderesult.eps}
    \caption{Group velocity change with temperature for the $A_1$ mode in Aluminium 1050 H14.}\label{fig:A1result}
\end{figure}
\vfill

\clearpage

\subsection{S1 mode (4 MHz-mm)}\label{S1 experiments}

This region of frequency-thickness product is multi-modal, with both the $A_1$ and $S_1$ modes present. Similarities in phase velocity leads to similar excitation angles, which causes both modes to be excited. Using a cross-correlation method for measuring time-of-flight is no longer appropriate, as the received signal differs substantially from the input signal. An envelope peak method is employed instead, whereby analytic envelopes for both the excitation signal and received signal are generated by Hilbert FIR filtering, with a filter length of 2000 samples. This produces a smooth envelope with clearly defined peaks, as seen in Figure~\ref{fig:S1timedomain}. A peak finding algorithm is used to detect the envelope peaks, as denoted by the dashed lines. The $S_1$ mode ($\sim$~4550 m s$^{-1}$) arrives at the receiver before the $A_1$ mode ($\sim$~2550 m s$^{-1}$) as it has a considerably higher group velocity. At this propagation distance the two modes are clearly separated in the time domain, with the $A_1$ mode showing considerably more dispersion.  

The wedge angle required for the $S_1$ mode is:
%
\begin{equation} 
24.5{^\circ} = \text{Sin}^{- 1} \left( \frac{2477}{5971} \right)
\end{equation} 
%
\vfill
\begin{figure}[h]
    \centering
    \includegraphics[width=.9\textwidth]{./figures/Exp_S1_20c-100c.eps}
    \caption{Time of flight ($t_F$) measurement of $S_1$ mode using envelope peak method at 20\si{\degreeCelsius} (green) and 100\si{\degreeCelsius} (red).}\label{fig:S1timedomain}
\end{figure}
\vfill

\clearpage

\subsubsection{S1 mode results}

Figure~\ref{fig:S1result} shows experimentally measured wave velocity of the $S_1$ mode plotted against theoretical wave velocity extracted from dispersion curves. Error bars show the standard deviation from the mean. After accounting for the temperature gradient across the transmission path by calculating a temperature average the change in velocity is comparable with predicted velocity extracted from dispersion curves, within 4.44 $\pm$ 7.15 m s$^{-1}$ on average. The temperature sensitivity of the system is 1.80 m s$^{-1}$\si{\degreeCelsius}$^{-1}$ over the range 25\si{\degreeCelsius}--103\si{\degreeCelsius}. The sensitivity is extracted from a linear fit of the data (r$^2$ = 0.9777).

\vfill
\begin{figure}[h]
    \centering
    \includegraphics[width=.9\textwidth]{./figures/s1moderesult.eps}
    \caption{Group velocity change with temperature for the $S_1$ mode in Aluminium 1050 H14.}\label{fig:S1result}
\end{figure}
\vfill

\clearpage

\subsection{Experimental sensitivity analysis}

There are a number of experimental error sources to consider. The physical distance between wedges is controlled using 3D printed spacers that keep the wedges aligned at set distances. The movement of the wedges on the surface of the plate increases with temperature as the viscosity of the couplant decreases. Variations in placement cause the calculated velocity to vary by around $\pm$ 5 m s$^{-1}$ across multiple (30) wedge re-alignments. The measurement of wedge-to-wedge time to be subtracted from the total $t_F$ is temperature dependant and relies on accurate alignment of the wedge feet, as well as a good connection between them (signal amplitude is highly dependant on couplant). Variation in alignment causes around a $\pm$ 10 m s$^{-1}$ velocity change. The wedge foot offset in Equation~\ref{velocitycalcfull} has a large effect on the calculated wave velocity. The exact offset distance is unknown and is assumed to be the point at which the centre line of the transducer aligns with the plate surface. Varying this value raises or lowers the velocity of all results considerably ($\pm$ 1 mm = $\pm$ 35 m s$^{-1}$). The hot plate does not heat the test plate evenly, especially at distances greater than 10 cm between wedges where they overhang the edges of the hot plate. The gradient (the difference in temperature between the centre of the plate and the location of the wedges) increases with temperature. The measured velocity is monitoring the average temperature of the transmission path. The gradient has been measured by placing a number (4) of thermocouples along the transmission path, from the centre of the plate (maximum temperature) to the point at which a wave is transmitted between a wedge foot and the plate. The calculation of aluminium dispersion curves at different temperatures is based on a change in Young's modulus. This is predicted from Hopkin's formula \cite{Hopkins2012} that may not give the correct values for Aluminium 1050 H14, but Aluminium in general.

The largest source of error is the measurement of wedge-to-wedge $t_F$, as a small error in alignment causes a large change ($\pm$ 10 m s$^{-1}$) in the wave velocity calculation. This is accounted for through the averaging of multiple (30) measurements, the standard deviation for this range is shown using error bars on Figure~\ref{fig:result}. Variation in wedge foot offset distance dramatically shifts the calculated velocity. This value cannot be directly measured and so the result relies on accurate calculation as discussed in the method.

\clearpage

\subsection{Conclusion}

The theoretical effect of temperature on various Lamb wave modes in aluminium plates has been investigated by generating dispersion curves based on varying material properties (Figure~\ref{fig:alutempshift}). This can be repeated in the future for other materials at higher temperatures (e.g.\ Inconel 718 up to 1100\si{\degreeCelsius} in Figure~\ref{fig:groupshift}). The temperature sensitivities of the $S_0$, $A_1$, and $S_1$ modes at 1~MHz have been extracted from these curves (Figure~\ref{fig:alugroupvel}) to be validated experimentally.

An experimental investigation has been carried out in order to validate theoretical predictions. Wedge transducers in a pitch-catch configuration have been used to excite the $S_0$, $A_1$, and $S_1$ modes in 1 mm, 2.5 mm, and 4 mm thick aluminium plates respectively. The time of flight between transducers has been measured and group velocity calculated based on the distance between transducers. This confirms that the $S_0$ mode has been excited. The change in $S_0$ wave velocity due to temperature is in line with theoretical predictions over the range 20\si{\degreeCelsius}--100\si{\degreeCelsius} as shown in Figure~\ref{fig:result}. 

It is clear that wedge transducers are not a viable method of transmitting/receiving a wave through a nozzle guide vane at high temperatures. They cannot be permanently mounted to the structure due to the need for a liquid couplant, and their relatively large footprint would make finding a suitable mounting location a challenge. Their operation at high temperatures is limited by the wedge material, which in the case of acrylic melts at around 160\si{\degreeCelsius}. The wedge material needs to have a longitudinal wave velocity less than that of the targeted Lamb wave phase velocity, which limits the choice of material severely, mostly to plastics with low melting points. The great benefit of wedge transducers is the ability to selectively target Lamb wave modes, which reduces the complexity of data analysis compared with exciting multiple modes simultaneously. This is difficult to achieve using other transducer configurations but it may instead be possible to excite a higher order region that travels as a single wave packet. The use of PWAS transducers could allow for operation at high temperatures (assuming suitable choice of piezoelectric material) and would be relatively easy to mount to an NGV structure, having a small footprint, although the high temperatures are likely to make bonding difficult. Another option is to couple into the structure using waveguides, distancing the transducers from the high temperature environment. Future research will investigate the temperature sensitivity of higher order modes (such as $A_1$ and $S_1$), as operating at higher frequencies can improve resolution (allowing for the detection of smaller phase shifts) and response rates. The ability to monitor wave velocity variations in multi-modal wave packets will also be considered when investigating transducer configurations suitable for higher temperature operation.

\clearpage

\section{Simulation study (Aluminium)} \label{simstudymain}

The multi-physics simulation package COMSOL has been used to replicate the wedge based time of flight measurement system used in the previous experimental study. The model has been validated against experimental results, to be used in subsequent studies involving different materials and surface coatings. The literature covering the use of COMSOL for modelling Lamb wave excitation using wedge transducers is limited, however it has been shown that Lamb waves can be successfully generated using this method~\cite{Nikolaevtsev2016a}.

\subsection{Variable angle wedge simulation}

A 2D model has been produced of the experimental test setup described in~\cref{experimentalstudy}. This allows for validation of the time of flight measurements, and can be used to separate the effect of temperature on the wedges from the substrate. The effect of temperature on the Lamb wave alone can therefore be analysed. 

\subsubsection{Geometry}

The model consists of two variable angle wedges (PMMA), which are based on the geometry of the Olympus variable angle wedges used in the experimental investigation, placed on top of an aluminium plate. The thickness of the plate can be varied to target different Lamb wave modes at different frequency-thickness products. The initial thickness is set to 1 mm to target the $S_0$ mode at 1 MHz--mm. The transmitting wedge has a simplified piezoelectric transducer (PZT-5H from COMSOL's material library) attached to it's rotating block, to which the excitation signal is applied. The geometry can be seen in Figure \ref{fig:COMSOLdiagram}. The received signal is measured at the receiver wedge's rotating block boundary. More realistic transducer configurations are not considered in this study, as the focus is on the effect of temperature on the propagating wave. A boundary area is set underneath the plate to act as the heat source, again mimicking the experimental setup. This is simplified to allow the temperature to be directly set, rather than simulating a hot plate.

\begin{figure}[h]
    \centering
    \includegraphics[width=.9\textwidth]{./figures/2d-comsol-geometry.eps}
    \caption{COMSOL geometry diagram.}\label{fig:COMSOLdiagram}
\end{figure}

\subsubsection{Material properties}

The change in Young's Modulus with temperature is included in the material properties for both the wedges and the aluminium using piecewise functions. Unfortunately the specific material properties of the Olympus wedges used in the experimental study are unknown, and the values for PMMA found in literature vary quite dramatically. Figure~\ref{fig:PMMAE} shows five sources of temperature dependant Young's modulus for PMMA, where the shaded green region indicates the temperature range of interest in this study (20-100\si{\degreeCelsius}). Polynomial fits have been applied to the data sets to allow them to be used in COMSOL (apart from COMSOL's in-built values which are given as a function) and are provided in Table~\ref{table:PMMAE}, where $T$ is the temperature in Kelvin. The Alexandria~\cite{Abdel-Wahab2017} and Goodyear Aerospace~\cite{Hassard1973} sources have been disregarded as the large reduction in $E$ over the temperature range of interest is not considered realistic, based on the use of the Olympus wedges in the experimental study. The Birmingham sources~\cite{Sahputra2018} (provided at two different annealing temperatures, 600K and 1000K) differ greatly from the values provided by COMSOL's material library entry for PMMA~\cite{Fukuhara1995}. Room temperature values of Young's modulus (for which there are many more sources) range from 1.8-5.0 GPa, however they are mostly commonly given at $\sim$3.0 GPa. In order to accurately represent the wedge material used experimentally, the value of Young's modulus has been inferred from measurements of longitudinal wave velocity. 

\begin{table}[h]
    \centering
    \begin{tabulary}{\textwidth}{LLL}
        \toprule
        \textbf{Property} & \textbf{PMMA} & \textbf{Aluminium}  \\
        \midrule
        Heat capacity at constant pressure (J/(kg$\cdot$K)) & 1470 & 904 \\
        Density (kg/m$^3$) & 1190 & 2700\\
        Thermal conductivity (W/(m$\cdot$K)) & 0.18  & 237\\
        Poisson's ratio & 0.35 & 0.3375\\
        \bottomrule
    \end{tabulary} 
    \caption{COMSOL material properties for PMMA and aluminium.}\label{table:matprop}
\end{table}
%
\begin{figure}[h]
    \centering
    \includegraphics[width=.8\textwidth]{./figures/PMMAE.eps}
    \caption{A range of sources for the Young's modulus of PMMA.}\label{fig:PMMAE}
\end{figure}
%
\clearpage
\begin{table}[h]
    \centering
\begin{adjustbox}{width=1.2\textwidth,center=\textwidth}
    \begin{tabular}{ll}
        \toprule
        \textbf{Source} & \textbf{Function}\\
        \midrule
        COMSOL Material Library~\cite{Fukuhara1995} & $E = 4.3102{\times}10^9 + 6.9344{\times}10^7 \times T^1 - 5.2821{\times}10^5 \times T^2+ 1.5796{\times}10^3 \times T^3-1.7421\times T^4$ \\
        Birmingham 600K Annealing~\cite{Sahputra2018} & $E = 58.3 \times T^3 + -7.6500{\times}10^4 \times T^2 + 2.1367{\times}10^7 \times T + 2.8500{\times}10^9$ \\
        Birmingham 1000K Annealing~\cite{Sahputra2018}  & $E = -116.7 \times T^3 + 9.3000{\times}10^4 \times T^2 + -2.8333{\times}10^7 \times T + 7.0200{\times}10^9$ \\
        Alexandria~\cite{Abdel-Wahab2017} & $E = -2.6773{\times}10^4 \times T^3 + 2.4371{\times}10^7 \times T^2 + -7.4029{\times}10^9 \times T + 7.5540{\times}10^{11}$ \\
        Goodyear Aerospace~\cite{Hassard1973} & $E = -1.5514{\times}10^5 \times T^2 + 6.4090{\times}10^7 \times T + -1.5120{\times}10^9$ \\
        \bottomrule
    \end{tabular}
\end{adjustbox}
\caption{Functions for PMMA Young's modulus.}\label{table:PMMAE}
\end{table}     
%
\subsubsection{Determination of Young's modulus from wedge wave velocity}

The longitudinal wave velocity of the wedges has been measured experimentally by placing the two wedges together, as shown in Figure~\ref{fig:w2wonaxis}. Time of flight is measured using an envelope peak finding function, and wave velocity calculated using the propagation distance (0.0715 m). The measured velocity of 2477 m s$^{-1}$ differs from that provided by the manufacturer (2720 m s$^{-1}$). Now that the wave velocity is known, the material properties of the model can be adjusted until the simulated wave velocity matches that of the experimental measurements. The model was set to run a parametric sweep of Young's modulus for the wedges, from 3.5$\times$10$^9$ to 6$\times$10$^9$ in 0.5$\times$10$^9$ increments, matching the range of potential values from literature. The wave velocity for each of these steps is calculated, and a polynomial fit of the data is generated in MATLAB. The quadratic equation produced is used to find the value of E closest to the wave velocity measured experimentally. The model was then rerun at smaller increments of E (4.20$\times$10$^9$ to 4.40$\times$10$^9$ in increments of 0.05$\times$10$^9$) to improve the accuracy of the polynomial fit. The model is then computed using this value of $E$ to verify that the velocity matches the prediction.

In order to determine the temperature dependant Young's modulus for the wedge material, the test setup was placed inside of an oven, and time of flight was measured up to 45\si{\degreeCelsius}. Increasing the temperature of the oven above this caused the signal amplitude to decrease dramatically, making time of flight measurement unreliable. The temperature of the oven was allowed time to stabilise, along with time of flight. The velocity calculated at this temperature was then used to find the associated value of $E$, as described previously. These values were then entered into COMSOL using an interpolation function, extrapolating the value of $E$ for higher and lower temperatures linearly. This method is sufficiently accurate for the small temperature range used in this study, and better represents the real material than using values derived from literature. 

\begin{figure}[h]
    \centering
    \includegraphics[width=.7\textwidth]{./figures/w2wdiagramonaxis.eps}
    \caption{Cross-sectional diagram of on-axis wedge-to-wedge time-of-flight measurement setup.}\label{fig:w2wonaxis}
\end{figure}

The experimentally measured longitudinal wave velocity of the wedges is also used to calculate the wedge angle required to excite particular modes, based on Snell's law. 

\subsubsection{Aluminium plate properties}

The choice of temperature dependent $E$ for Aluminium is more straight forward, as sources for bulk aluminium and aluminium 1050 are provided by the COMSOL material library, and they are very similar to values provided by Hopkins~\cite{Hopkins2012}, as shown in Figure~\ref{fig:AluE}. The shaded green region indicates the temperature range of interest in this study (20-100\si{\degreeCelsius}). The functions used to generate the curves are given in Table~\ref{table:AluE}.

\begin{figure}[h]
    \centering
    \includegraphics[width=.8\textwidth]{./figures/Alu_E.eps}
    \caption{Young's modulus of Aluminium.}\label{fig:AluE}
\end{figure}

\begin{table}[h]
    \centering
\begin{adjustbox}{width=1.2\textwidth,center=\textwidth}
    \begin{tabular}{ll}
        \toprule
        \textbf{Source} & \textbf{Function}\\
        \midrule
        COMSOL Aluminium 1050 & $7.7703{\times}10^{10} + 2.0365{\times}10^6 \times T^1 -1.8916{\times}10^5 \times T^2 + 4.2529{\times}10^2 \times T^3 -3.5457 {\times}{10^{-1}} \times T^4 $ \\
        COMSOL Aluminium Bulk & $7.6593{\times}10^{10} + 2.0074{\times}10^6 \times T^1 -1.8646{\times}10^5 \times T^2 + 4.1922{\times}10^2 \times T^3 -3.4951 {\times}{10^{-1}} \times T^4 $\\
        Hopkins~\cite{Hopkins2012} & $-4{\times}10^7 \times T + 8{\times}10^{10}$\\
        \bottomrule
    \end{tabular}
\end{adjustbox}
\caption{Functions for Aluminium Young's modulus.}\label{table:AluE}
\end{table}    

The change in Poisson's ratio and density is assumed to negligible and is not included in the simulation. Thermal expansion is also considered to have a negligible effect on the propagation distance and is excluded (calculated to have an average reduction in wave velocity of the $S_0$ mode in aluminium of -1.20 m s$^{-1}$ over the temperature range 20-100\si{\degreeCelsius}). 

\clearpage

\subsubsection{Transducer configuration}

The simplified ultrasonic transducer used in this study is comprised of an active piezoelectric element and a backing layer. Lead Zirconate Titanate (PZT-5H) from COMSOL's material library is operated at the first through-thickness resonance frequency for 1 MHz, where the ceramic thickness is equal to half a wavelength (2 mm). 

\begin{equation}
    \text{Piezo thickness} = \left(\frac{v_p\times f_0}{2} \right)
    \tagaddtext{[mm]}
 \end{equation}

 \begin{equation}
    2.05 = \left(\frac{4101 \times 1}{2} \right)
    \tagaddtext{[mm]}
 \end{equation}

The backing layer is comprised of a highly attenuating material, with an acoustic impedance matching the piezo material as closely as possible. In this case an epoxy resin mixed with Tungsten powder is used~\cite{Rathod2020}. A matching layer is not employed as the signal amplitude is sufficient for this study.

The electrostatics module is set up as follows: A zero charge node is used for the edges of the piezoelectric material, initial values are set to 0 V, a ``Charge Conservation, Piezoelectric'' node is set for the piezoelectric material, a ground boundary is selected for the wedge side of the material, and a terminal node is set for the opposite boundary. Within the terminal node the type is set to Voltage and the input is set to ``V0(t)''. The excitation signal is a 1 MHz 5--cycle Hamming windowed sine pulse generated in MATLAB and included in COMSOL using an analytic function (Definitions$>$Functions$>$Analytic), given in Equation~\ref{eq:sinepulse} and shown in Figure~\ref{fig:excitation}. The receiver transducer is setup in the same way, except the terminal node type is set to ``charge''.

\subsubsection{Physics \& mesh settings}

The modules Solid Mechanics, Electrostatics, and Heat Transfer in Solids are used in this simulation, along with a multi-physics node to couple Solid Mechanics with Electrostatics for the piezoelectric effect. Both the wedges and the plate are set to isotropic linear elastic materials, with low reflecting boundaries applied to the wedges.

For the Heat Transfer in Solids module all the domains are set to solid, and initial values are set to 20\si{\degreeCelsius}. The boundaries that are exposed to the air are selected in a Heat Flux node, where convective heat flux is selected. A user defined heat transfer coefficient of 7~W/(m$^2\cdot$K) is used for the plate, and 1~W/(m$^2\cdot$K) for the wedges. These values were set to produce the temperature gradients measured experimentally in both the plate and the wedges. The external temperature is set to 20\si{\degreeCelsius}. The temperature of the boundary underneath the plate is adjusted as required (20\si{\degreeCelsius} to 100\si{\degreeCelsius} in 20\si{\degreeCelsius} increments for this study). An example of the temperature gradients produced from the stationary study step are shown in Figure~\ref{fig:COMSOLtemp100c}, where the temperature boundary underneath the plate is set to 100\si{\degreeCelsius}.

As the temperature decreases between the heat area and the end of the plate, the temperature under the wedges is less than temperature applied to the plate. This is important for the wedge-to-wedge study, as the heat boundary between wedges cannot be set to the temperature used for the boundary in the full study. Instead the full model is run first, and temperature probes are used to measure the temperature in the plate underneath the wedges in 5 mm increments from one side to the other. The study is carried out at heat boundary temperatures of 20.0\si{\degreeCelsius}, 46.7\si{\degreeCelsius}, 73.3\si{\degreeCelsius}, and 100.0\si{\degreeCelsius}. A mean average of these measurements is calculated and used for the boundary temperature in the wedge-to-wedge study, as shown in~\cref{table:S0gradients,table:A1gradients,table:S1gradients}.

\clearpage

\begin{table}[h]
    \centering
    \begin{tabulary}{\textwidth}{LLLLL}
        \toprule
        & \multicolumn{4}{c}{\textbf{Boundary temperature (\si{\degreeCelsius})}} \\
        \midrule
        \textbf{x (mm)}       & \textbf{20.0\si{\degreeCelsius}} & \textbf{46.7\si{\degreeCelsius}} & \textbf{73.3\si{\degreeCelsius}} & \textbf{100\si{\degreeCelsius}} \\
        \midrule
        0                & 20.0            & 41.7            & 63.4            & 84.4           \\
        5                & 20.0            & 42.0              & 64.1            & 85.4           \\
        10               & 20.0            & 42.4            & 64.8            & 86.6           \\
        15               & 20.0            & 42.8            & 65.5            & 87.7           \\
        20               & 20.0            & 43.2            & 66.3            & 88.9           \\
        25               & 20.0            & 43.6            & 67.0              & 90.2           \\
        30               & 20.0            & 44.0              & 67.9            & 91.5           \\
        35               & 20.0            & 44.4            & 68.8            & 92.9           \\
        40               & 20.0            & 44.9            & 69.8            & 94.5           \\
        45               & 20.0            & 45.4            & 70.7            & 96.0           \\
        \midrule
        \textbf{Average} & \textbf{20.0} & \textbf{43.4}   & \textbf{66.8}   & \textbf{89.8} \\
        \bottomrule
    \end{tabulary}%
    \caption{Boundary temperatures for wedge-to-wedge study at $S_0$.}\label{table:S0gradients}
    \bigskip
    \begin{tabulary}{\textwidth}{LLLLL}
        \toprule
        & \multicolumn{4}{c}{\textbf{Boundary temperature (\si{\degreeCelsius})}} \\
        \midrule
        \textbf{x (mm)}       & \textbf{20.0\si{\degreeCelsius}} & \textbf{46.7\si{\degreeCelsius}} & \textbf{73.3\si{\degreeCelsius}} & \textbf{100\si{\degreeCelsius}} \\
        \midrule
        0                & 20.0            & 44.3            & 68.7            & 93.1           \\
        5                & 20.0            & 44.5            & 69.0              & 93.5           \\
        10               & 20.0            & 44.7            & 69.3            & 94.0             \\
        15               & 20.0            & 44.8            & 69.6            & 94.5           \\
        20               & 20.0            & 45.0              & 70.0              & 95.1           \\
        25               & 20.0            & 45.2            & 70.4            & 95.6           \\
        30               & 20.0            & 45.4            & 70.8            & 96.2           \\
        35               & 20.0            & 45.6            & 71.2            & 96.8           \\
        40               & 20.0            & 45.8            & 71.6            & 97.5           \\
        45               & 20.0            & 46.1            & 72.0            & 98.1           \\
        \midrule
        \textbf{Average} & \textbf{20.0} & \textbf{45.1}   & \textbf{70.3}   & \textbf{95.4} \\
        \bottomrule
    \end{tabulary}%
    \caption{Boundary temperatures for wedge-to-wedge study at $A_1$.}\label{table:A1gradients}
    \bigskip
    \centering
    \begin{tabulary}{\textwidth}{LLLLL}
        \toprule
        & \multicolumn{4}{c}{\textbf{Boundary temperature (\si{\degreeCelsius})}} \\
        \midrule
        \textbf{x (mm)}       & \textbf{20.0\si{\degreeCelsius}} & \textbf{46.7\si{\degreeCelsius}} & \textbf{73.3\si{\degreeCelsius}} & \textbf{100\si{\degreeCelsius}} \\
        \midrule
        0                & 20.0            & 45.1            & 70.2            & 95.4           \\
        5                & 20.0            & 45.2            & 70.4            & 95.7           \\
        10               & 20.0            & 45.3            & 70.6            & 96.0             \\
        15               & 20.0            & 45.5            & 70.8            & 96.4           \\
        20               & 20.0            & 45.6            & 71.1            & 96.7           \\
        25               & 20.0            & 45.7            & 71.3            & 97.1           \\
        30               & 20.0            & 45.8            & 71.6            & 97.4           \\
        35               & 20.0            & 45.9            & 71.9            & 97.9           \\
        40               & 20.0            & 46.1            & 72.1            & 98.3           \\
        45               & 20.0            & 46.3            & 72.4            & 98.7           \\
        \midrule
        \textbf{Average} & \textbf{20.0} & \textbf{45.7}   & \textbf{71.2}   & \textbf{97.0} \\
        \bottomrule
    \end{tabulary}%
\caption{Boundary temperatures for wedge-to-wedge study at $S_1$.}\label{table:S1gradients}
\end{table}

\clearpage

The mesh size for each material is determined by excitation frequency. The excitation wavelength for each of the materials is calculated by dividing their longitudinal wave speed by $f_0$. A free triangular mesh is created for each of the materials, and the maximum element size for each of them is set to LocalWavelength/N. If higher frequency content is expected, the wavelength for each material should be based on the highest frequency expected rather than $f_0$. In order to accurately resolve a wave, at least 10--12 elements per local wavelength are required~\cite{COMSOL2013}. This assumes linear discretization for all modules. Using 12 elements results in an average skewness rating (measure of element quality, 0--1) of 0.9345 over 154728 elements~\cite{COMSOL2017}. This is equivalent to a sample rate of 1.2$\times$10$^8$.

\begin{figure}[h]
    \centering
    \includegraphics[width=.9\textwidth]{./figures/wedge_mesh.png}
    \caption{Mesh skewness quality plot. Green elements represent values close to 1.}\label{fig:wedgemesh}
\end{figure}

\subsubsection{Study settings}

This study has two steps, firstly a stationary study to simulate the effect of temperature on the system until an equilibrium is reached, and secondly a time dependant study to simulate wave propagation. The initial conditions of the time dependant study are set by the stationary study. The stationary study solves only for heat transfer and not electrostatics/the piezoelectric effect. 

\begin{figure}[h]
    \centering
    \includegraphics[width=.9\textwidth]{./figures/temp_gradients.png}
    \caption{Simulated temperature gradients from stationary study at 100\si{\degreeCelsius}. Scale is given in degrees Celsius (\si{\degreeCelsius}).}\label{fig:COMSOLtemp100c}
\end{figure}

The time dependant study includes electrostatics/the piezoelectric effect to allow for wave generation, but does not include heat transfer. This reduces computation time as it is not necessary to model changing temperature as the time dependant model solves, only to use the fixed values of material properties that have been passed on from the stationary study. The time dependant study has its ``Output times'' set to: range(0,dt,sim\textunderscore length) where ``dt'' is a global definition parameter equal to CFL/(N$\times f_0$). The CFL (Courant Friedrichs Lewy) number is suggested by COMSOL~\cite{COMSOL2021} to be less than 0.2, optimally 0.1 (when the default second order, quadratic, mesh elements are used). This value represents the relationship between wave speed, $c$, maximum mesh size, $h$, and time step length, $\Delta t$: $CFL = c\Delta t/h$. This can be rewritten in terms of frequency as the maximum mesh size $h$ has already been manually defined by $N$, the number of elements per local wavelength for each material: $CFL = fN\Delta t$. This can then be rearranged to give the time step: $\Delta t = CFL/Nf$. 

Under ``Values of Dependant Variables'' the settings are changed to user controlled, method is changed to Solution, and the study is set to the stationary study. The time step is manually set under Solver Configurations$>$Solution 1$>$Time dependant solver$>$Time stepping. Here the ``Steps taken by solver'' parameter is changed to ``Manual'' and the ``Time Step'' is set to: $CFL/(N\times f_0)$. 

To reduce file size only the data at the wedge boundaries is stored by the solver. This can be achieved by adding an ``Explicit Selection'' node in the Geometry section, and selecting both the transmit and receive wedge boundaries. Within the time dependant study settings select ``For selection'' under ``Store fields in output'' and select the boundary group~\cite{COMSOL2021a}. File size can be further reduced by only storing the field components of interest. Within ``Solver Configurations''$>$Solution 1$>$Dependant Variables 2, the option for storing the displacement field in the output can be deselected, as only the electric potential is necessary. 

A parametric sweep node was used to cycle through the temperature boundary values (20.0\si{\degreeCelsius}, 46.7\si{\degreeCelsius}, 73.3\si{\degreeCelsius}, and 100.0\si{\degreeCelsius}) and save the output of the time dependant model for each value. This is repeated for the model in the wedge-to-wedge configuration (at temperatures shown in~\cref{table:S0gradients,table:A1gradients,table:S1gradients}), mimicking the experimental setup shown in Figure~\ref{fig:testdiagramw2w}. The simulations were computed on the University of Southampton's IRIDIS 5 supercomputing platform~\cite{Southampton2021}.

\subsection{Simulation results}

A simplified model where PZT is applied directly to the plate shows that both the $A_0$ and $S_0$ modes are excited at a frequency of 1 MHz. \cref{fig:simmodes} shows exaggerated deformation of pressure in the plate, which makes the presence of the $A_0$ and $S_0$ modes clearly visible. $A_0$ exhibits large out of plane motion, whereas $S_0$ exhibits large in plane motion. The modes are separated in the time domain after a short distance ($\sim$~50 mm) due to their differences in group velocity. When wedges are applied to the plate selective mode excitation can be achieved, however it is necessary to use wedges as both transmitters and receivers to most effectively isolate the mode of interest.

\begin{figure}[h]
    \centering
    \includegraphics[width=.9\textwidth]{./figures/A0S0_deformation.eps}
    \caption{Exaggerated deformation of pressure showing the presence of the $A_0$ \& $S_0$ modes.}\label{fig:simmodes}
\end{figure}

\clearpage

To visualise wave propagation and calculate time of flight the pressure at both transmitter and receiver wedge boundaries is exported, and the time of flight is measured using an envelope peak extraction method, to allow direct comparison with experimental results. 

Wedge foot offset (the distance a wave travel under each wedge foot) is calculated in the same way for both the simulation and the experiments, however the value differs, which indicates a difference in geometry between them. Despite this, the difference in calculated velocities is small, as using accurate estimations of wedge foot offset corrects for the difference in total time of flight. Time of flight in the wedge-to-wedge configuration is in line with experimental measurements, which suggests that the geometry and material properties of the wedges are realistic. The material properties of the aluminium plate are the same as those used in the theoretical study, which should (in theory) mean that the velocity in the simulated plate is the same as was extracted from dispersion curves.

%
\begin{table}[h]
    \centering
    \begin{tabulary}{\textwidth}{LLLLL}
        \toprule
        & \multicolumn{3}{c}{\textbf{Temperature sensitivity} (m s$^{-1}$\si{\degreeCelsius}$^{-1}$)} \\
        \midrule
        \textbf{Wave mode} & \textbf{Predicted} & \textbf{Simulated} & \textbf{Experimental}\\
        \midrule
        $S_0$ & -1.47 & -1.47 & -1.58 \\
        $A_1$ & -0.80 & -0.99 & -1.13 \\
        $S_1$ & -1.33 & -1.75 & -1.89 \\
        \bottomrule
    \end{tabulary}
\caption{Average temperature sensitivity of $S_0$, $A_1$, and $S_1$ Lamb wave modes in Aluminium from 20\si{\degreeCelsius} to 100\si{\degreeCelsius}.}\label{table:sensitivityresults}
\end{table}    
%
Time-displacement data is transformed into frequency-wavenumber data using 2D-FFT from spatial B-scan data, using 90 point probes equally spaced 0.8 mm apart. This allows individual modes to be identified, and compared with dispersion curves for verification. Out-of-plane ($y$-axis) displacement is monitored, to reflect the response detected when using wedge transducers. It should be noted, however, that the displacement response in the plate measured using a point probe will differ from the response received at a second receiver wedge. The use of a second wedge further isolates a particular mode of interest, while the displacement in the plate may still show the presence of other modes.

\clearpage

\subsubsection{$S_0$ Mode simulations}

Figure~\ref{fig:S0waveprop} shows the wave propagation of a pulse exciting the $S_0$ mode at 20\si{\degreeCelsius} and 100\si{\degreeCelsius}. The black dotted lines indicate the peak of the envelopes used to calculate time of flight.
Figure~\ref{fig:s0resultfull} shows the change in velocity with temperature for the $S_0$ Lamb wave mode in Aluminium, comparing predicted temperature sensitivity extracted from dispersion curves, experimental measurement data (Section~\ref{S0 experiments}), and COMSOL simulations of the experimental setup. The experimental result is within 35.93 $\pm$ 3.06 m s$^{-1}$ or 0.71\% $\pm$ 0.06\% of the predicted velocity on average. The COMSOL results are within 10.61 m s$^{-1}$ $\pm$ 0.73 m s$^{-1}$ or 0.21\% $\pm$ 0.01\% of the predicted result on average. The standard deviation of group velocity across four wedge spacings (80 mm, 90 mm, 100 mm, 110 mm) at the calculated offset value of 45.3 mm is 0.74 m s$^{-1}$, which indicates that the simulation and time of flight measurement method are producing accurate results.

\cref{fig:2DFFT-S0-Alu} shows the 2D-FFT response of the propagating wave packet. The solid and dashed lines represent the anti-symmetric and symmetric dispersion curves respectively. Almost pure excitation of the $S_0$ mode is achieved.

\vfill
\begin{figure}[h]
    \centering
    \includegraphics[width=.9\textwidth]{./figures/COMSOL_S0_20c-100c.eps}
    \caption{Wave propagation of $S_0$ Lamb wave mode in Aluminium at 20\si{\degreeCelsius} (green) and 100\si{\degreeCelsius} (red).}\label{fig:S0waveprop}
\end{figure}
\vfill

\begin{figure}[h]
    \centering
    \includegraphics[width=.9\textwidth]{./figures/2DFFT-S0-1MHz-Ydisp.eps}
    \caption{2D-FFT of $S_0$ excitation in 1 mm thick Aluminium at 20\si{\degreeCelsius}. Solid and dashed lines represent numerically calculated dispersion curves. Areas of high intensity (darker colours) show where modes have been detected.}\label{fig:2DFFT-S0-Alu}
\end{figure}

\begin{figure}[h]
    \centering
    \includegraphics[width=.9\textwidth]{./figures/s0fullresultedit.eps}
    \caption{Velocity change with temperature for $S_0$ Lamb wave mode in Aluminium. Comparison between predicted, experimental, and simulated results.}\label{fig:s0resultfull}
\end{figure}

\clearpage
\subsubsection{$A_1$ Mode simulations}

Figure~\ref{fig:A1waveprop} shows the wave propagation of a pulse exciting the $A_1$ mode at 20\si{\degreeCelsius} and 100\si{\degreeCelsius}. The black dotted lines indicate the peak of the envelopes used to calculate time of flight.
Figure~\ref{fig:A1resultfull} shows the change in velocity with temperature for the $A_1$ Lamb wave mode in Aluminium, comparing predicted temperature sensitivity extracted from dispersion curves, experimental measurement data (Section~\ref{A1 experiments}), and COMSOL simulations of the experimental setup. The experimental result is within 49.05 $\pm$ 7.90 m s$^{-1}$ or 1.35\% $\pm$ 0.23\% of the predicted velocity on average. The COMSOL results are within 19.22 m s$^{-1}$ $\pm$ 5.93 m s$^{-1}$ or 0.53\% $\pm$ 0.17\% of the predicted result on average. The standard deviation of group velocity across eight wedge spacings (80 mm to 150 mm in 10 mm increments) at the calculated offset value of 46.2 mm is 6.94 m s$^{-1}$, which indicates that the simulation and time of flight measurement method are not producing results as accurately as at $S_0$.

\cref{fig:2DFFT-A1-Alu} shows the 2D-FFT response of the propagating wave packet. The solid and dashed lines represent the anti-symmetric and symmetric dispersion curves respectively. Almost pure excitation of the $A_1$ mode is achieved. The presence of the $S_0$ mode is not evident when analysing the time history shown in \cref{fig:A1waveprop}.

\vfill
\begin{figure}[h]
    \centering
    \includegraphics[width=.9\textwidth]{./figures/COMSOL_A1_20c-100c.eps}
    \caption{Wave propagation of $A_1$ Lamb wave mode in Aluminium at 20\si{\degreeCelsius} and 100\si{\degreeCelsius}.}\label{fig:A1waveprop}
\end{figure}
\vfill

\begin{figure}[h]
    \centering
    \includegraphics[width=.9\textwidth]{./figures/2DFFT-A1-1MHz-Ydisp.eps}
    \caption{2D-FFT of $A_1$ excitation in 2.5 mm thick Aluminium at 20\si{\degreeCelsius}. Solid and dashed lines represent numerically calculated dispersion curves. Areas of high intensity (darker colours) show where modes have been detected.}\label{fig:2DFFT-A1-Alu}
\end{figure}

\begin{figure}[h]
    \centering
    \includegraphics[width=.9\textwidth]{./figures/A1fullresultedit.eps}
    \caption{Velocity change with temperature for $A_1$ Lamb wave mode in Aluminium. Comparison between predicted, experimental, and simulated results.}\label{fig:A1resultfull}
\end{figure}

\clearpage
\subsubsection{$S_1$ Mode simulations}

Figure~\ref{fig:S1waveprop} shows the wave propagation of a pulse exciting the $S_1$ mode at 20\si{\degreeCelsius} and 100\si{\degreeCelsius}. The black dotted lines indicate the peak of the envelopes used to calculate time of flight.
Figure~\ref{fig:S1resultfull} shows the change in velocity with temperature for the $S_1$ Lamb wave mode in Aluminium, comparing predicted temperature sensitivity extracted from dispersion curves, experimental measurement data (Section~\ref{S1 experiments}), and COMSOL simulations of the experimental setup. The experimental result is within 60.04 $\pm$ 14.00 m s$^{-1}$ or 1.34\% $\pm$ 0.32\% of the predicted velocity on average. The COMSOL results are within 89.83 m s$^{-1}$ $\pm$ 13.00 m s$^{-1}$ or 2.01\% $\pm$ 0.31\% of the predicted result on average. The standard deviation of group velocity across six wedge spacings (80 mm to 130 mm in 10 mm increments) at the calculated offset value of 44.69 mm is 7.89 m s$^{-1}$, which indicates that the simulation and time of flight measurement method are not producing results as accurately as at $S_0$.

\cref{fig:2DFFT-A1-Alu} shows the 2D-FFT response of the propagating wave packet. The solid and dashed lines represent the anti-symmetric and symmetric dispersion curves respectively. The response is dominated by the $A_1$ mode, which can be seen as the second more dispersive wave packet in \cref{fig:S1waveprop}, as it has considerably larger out-of-plane ($y$-axis) displacement than the $S_1$ mode. The use of a receiver wedge for the time-of-flight simulations further isolates the $S_1$ mode, which reduces the amplitude of the $A_1$ mode to a comparable level to that of $S_1$.

\vfill
\begin{figure}[h]
    \centering
    \includegraphics[width=.9\textwidth]{./figures/COMSOL_S1_20c-100c.eps}
    \caption{Wave propagation of $S_1$ Lamb wave mode in Aluminium at 20\si{\degreeCelsius} and 100\si{\degreeCelsius}.}\label{fig:S1waveprop}
\end{figure}
\vfill

\begin{figure}[h]
    \centering
    \includegraphics[width=.9\textwidth]{./figures/2DFFT-S1-1MHz-Ydisp.eps}
    \caption{2D-FFT of $S_1$ excitation in 4 mm thick Aluminium at 20\si{\degreeCelsius}. Solid and dashed lines represent numerically calculated dispersion curves. Areas of high intensity (darker colours) show where modes have been detected.}\label{fig:2DFFT-S1-Alu}
\end{figure}

\begin{figure}[h]
    \centering
    \includegraphics[width=.9\textwidth]{./figures/S1fullresultedit.eps}
    \caption{Velocity change with temperature for $S_1$ Lamb wave mode in Aluminium. Comparison between predicted, experimental, and simulated results.}\label{fig:S1resultfull}
\end{figure}

\clearpage
\section{Inconel 718 simulation study}

Applying a guided wave based temperature monitoring system to nozzle guide vanes involves different materials and a higher temperature range than investigated in the previous tests. To test the feasibility of the system (minus an appropriate transducer) the material of the model has been replaced with Inconel 718, a commonly used superalloy for high temperature aero engine components. The temperature range of the test has been extended to 1027\si{\degreeCelsius}. Although the wedges cannot be used in reality for this application, they can still be used in the model for single mode excitation. The heat transfer physics model has been disabled for the wedges, with only the plate affected by a change in temperature. The large temperature range causes a large change to the material properties of the Inconel 718, which in turn causes a large change in wave speed. As the change is so large the wedge angle has to be adjusted to continually target the same area of the frequency-thickness spectrum, as shown in Table~\ref{table:wedgeangleinconelmodel}. The longitudinal velocity of the wedge material is 2477~\unit{\metre\per\second}, as measured experimentally.

\vfill
\begin{table}[h]
    \centering
    \begin{tabulary}{\textwidth}{LLL}
        \toprule
    \textbf{\textbf{Temperature (\si{\degreeCelsius})}} & \textbf{Phase velocity (\unit{\metre\per\second})} & \textbf{Wedge angle (\degree)} \\
        \midrule
    27           & 5110.74            & 29.31                \\
    227          & 5020.12            & 29.89                \\
    427          & 4889.88            & 30.78                \\
    627          & 4705.97            & 32.12                \\
    827          & 4450.95            & 34.20                \\
    1027         & 4090.86            & 37.71                \\
    \bottomrule
    \end{tabulary}%
    \caption{Wedge angle required for $S_0$ mode excitation in Inconel 718 from 27\si{\degreeCelsius} to 1027\si{\degreeCelsius}.}\label{table:wedgeangleinconelmodel}
\end{table}
\vfill
\begin{figure}[h]
    \centering
    \includegraphics[width=.8\textwidth]{./figures/Inc718matpropsV2.eps}
    \caption{Temperature dependent Young's modulus, density, and calculated Poisson's ratio for Inconel 718.}\label{fig:dpe}
\end{figure}
\vfill

Figure~\ref{fig:inconelcomsol} shows the change in group velocity with temperature for the $S_0$ Lamb wave mode, from 27\si{\degreeCelsius} to 1027\si{\degreeCelsius}, comparing predicted results extracted from dispersion curves with simulated results from COMSOL. The average temperature sensitivity for the predicted result is -1.23 m s$^{-1}$\si{\degreeCelsius}$^{-1}$ $\pm$ 0.70 m s$^{-1}$\si{\degreeCelsius}$^{-1}$. The average temperature sensitivity for the simulated result is -1.31 m s$^{-1}$\si{\degreeCelsius}$^{-1}$ $\pm$ 0.67 m s$^{-1}$\si{\degreeCelsius}$^{-1}$. The temperature sensitivity increases with temperature for both results, from a minimum of -0.37 m s$^{-1}$\si{\degreeCelsius}$^{-1}$ to a maximum of -2.93 m s$^{-1}$\si{\degreeCelsius}$^{-1}$. The simulated group velocity is within 6.78 m s$^{-1}$ $\pm$ 27.68 m s$^{-1}$ or 0.10 \% $\pm$ 0.63\% of the predicted group velocity on average.

\cref{table:Inc718tempdata} shows time of flight data calculated from COMSOL simulations at each temperature step, along with the group velocity calculated using \cref{velocitycalcfull}. Where $d_{\textrm{aligned}}$ is 0.1~m, and $d_{\textrm{offset}}$ is 0.04696~m. The offset value was calculated from simulations of five propagation distances (0.08 to 0.12 m in 0.01 m increments), where the velocity variation at this offset value is 0.012~\unit{\metre\per\second}.

Figure~\ref{fig:Inc718_20c_1020c} shows the wave propagation of a pulse exciting the $S_0$ mode at 27\si{\degreeCelsius} (green) and 1027\si{\degreeCelsius} (red). The black dotted lines indicate the peak of the envelopes used to calculate time of flight. As temperature increases the change in material properties causes a dispersion curve shift, and a more dispersive region of the $S_0$ curve is excited. This can be seen in the response at 1027~\si{\degreeCelsius}, where the wave packet is visibly more spread out in comparison to the response at 20~\si{\degreeCelsius}.

\vfill
\begin{table}[h]
    \centering
    \begin{tabulary}{\textwidth}{LLLL}
        \toprule
    \textbf{\textbf{Temperature (\si{\degreeCelsius})}} & \textbf{$t_{\textrm{aligned}}$ (\unit{\second})} & \textbf{$t_{\textrm{wedge}}$ (\unit{\second})} & \textbf{Group velocity (\unit{\metre\per\second})} \\
        \midrule
        27          & $5.8906{\times}10^{-5}$ & $2.9615{\times}10^{-5}$ & 5017.13  \\
        227         & $5.9594{\times}10^{-5}$ & $2.9615{\times}10^{-5}$ & 4902.07  \\
        427         & $6.0573{\times}10^{-5}$ & $2.9604{\times}10^{-5}$ & 4745.43  \\
        627         & $6.2063{\times}10^{-5}$ & $2.9604{\times}10^{-5}$ & 4527.65  \\
        827         & $6.4781{\times}10^{-5}$ & $2.9594{\times}10^{-5}$ & 4176.48  \\
        1027        & $6.9208{\times}10^{-5}$ & $2.9552{\times}10^{-5}$ & 3705.85  \\
    \bottomrule
    \end{tabulary}%
    \caption{COMSOL simulation data for Inconel 718 model.}\label{table:Inc718tempdata}
\end{table}
\vfill


\begin{figure}[h]
    \centering
    \includegraphics[width=.8\textwidth]{./figures/Inc718_S0_27c-1027c.eps}
    \caption{Wave propagation of $S_0$ Lamb wave mode in Inconel 718 at 27\si{\degreeCelsius} and 1027\si{\degreeCelsius}.}\label{fig:Inc718_20c_1020c}
\end{figure}

\begin{figure}[h]
    \centering
    \includegraphics[width=.8\textwidth]{./figures/inconelcomsolresult.eps}
    \caption{Velocity change with temperature for $S_0$ Lamb wave mode in Inconel 718.}\label{fig:inconelcomsol}
\end{figure}



\printbibliography[title={Chapter~\thechapter~Bibliography}]
