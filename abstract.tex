\thispagestyle{plain}

\hspace{0pt}
\vfill

\rule{\linewidth}{0.5pt}
\begin{center}
    \Large
    \textbf{Abstract}
\end{center}
\rule{\linewidth}{0.5pt}

\vspace{1cm}

This study investigates the use of ultrasonic guided waves (Lamb waves) for the temperature monitoring of nozzle guide vanes (NGVs). These components are found in the turbine section of jet engines and are operated at extremely high temperatures. A literature review has been carried out that covers the existing temperature monitoring systems for turbine blades and nozzle guide vanes. Both offline and online methods are presented and their advantages and disadvantages are examined. The use of offline systems is well established but their online equivalents are difficult to implement because of the limited access to components. There is the need for an improved sensor that is capable of mapping temperature in real time with minimum interference to the operating conditions of the engine, allowing operating temperatures to be increased to the limits of the components and maximising efficiency. Acoustic monitoring techniques are already used for a large number of structural health monitoring (SHM) applications and have the potential to be adapted for use in temperature monitoring for turbine blades and NGVs. High temperatures severely affect the response of ultrasonic transducers. However, waveguides and buffer rods can be used to distance transducers from extreme conditions, while piezoelectric materials such as YCOB and AlN have been developed for use at high temperatures. The geometry of turbine blades and NGVs allows Lamb waves to propagate through their structure, and the presence of numerous cooling holes will produce acoustic reflections that have the potential to be utilised for temperature mapping.

A number of studies are underway to investigate the effects of the physical environment on wave propagation, and to determine the Lamb wave mode best suited to the application. The generation of dispersion curves from material properties allows theoretical temperature sensitivity to be determined, which has been verified experimentally. A test system has been developed to target modes of interest, and analyse the effect of temperature (and other factors) on wave propagation. A COMSOL model has also been developed that will be used in future studies to investigate the effect of cooling holes, surface coatings, and curved surfaces on wave propagation. The model will also be used to investigate different sensor configurations that can be implemented on an NGV. 

\vfill
\hspace{0pt}