\thispagestyle{plain}

\hspace{0pt}
\vfill

\rule{\linewidth}{0.5pt}
\begin{center}
    \Large
    \textbf{Abstract}
\end{center}
\rule{\linewidth}{0.5pt}

\vspace{1cm}

This study investigates the use of ultrasonic guided waves (Lamb waves) for the temperature monitoring of nozzle guide vanes (NGVs). These components are found in the turbine section of jet engines and are operated at extremely high temperatures. A literature review has been carried out that covers the existing temperature monitoring systems for turbine blades and nozzle guide vanes. Both offline and online methods are presented and their advantages and disadvantages are examined. The use of offline systems is well established but their online equivalents are difficult to implement because of the limited access to components. There is the need for an improved sensor that is capable of mapping temperature in real time with minimum interference to the operating conditions of the engine, allowing operating temperatures to be increased to the limits of the components and maximising efficiency. Acoustic monitoring techniques are already used for a large number of structural health monitoring (SHM) applications and have the potential to be adapted for use in temperature monitoring for turbine blades and NGVs. High temperatures severely affect the response of ultrasonic transducers. However, waveguides and buffer rods can be used to distance transducers from extreme conditions, while piezoelectric materials such as YCOB and AlN have been developed for use at high temperatures. The geometry of turbine blades and NGVs allows Lamb waves to propagate through their structure, and the presence of numerous cooling holes will produce acoustic reflections that have the potential to be utilised for temperature mapping.

A number of studies have been undertaken to investigate the effects of the physical environment on wave propagation, and to determine the Lamb wave mode best suited to the application. The generation of dispersion curves from material properties allows theoretical temperature sensitivity to be determined, which has been verified experimentally. A test system has been developed to target modes of interest, and analyse the effect of temperature (and other factors) on wave propagation. A finite element model mimicking the experimental setup has been developed and validated against experimental results. Investigations into the effect of cooling holes on wave propagation have been carried out using an adapted model. The model was further used to investigate the effects of thermal barrier coatings on wave propagation.

Results indicate that an ultrasonic guided wave based temperature monitoring system would be able to provide temperature resolution and accuracy comparable with traditional temperature sensors, if operated in a frequency region where a single mode can be identified. Peak envelope time-of-flight methods are suitable for detecting changes in wave velocity with temperature, assuming a clear peak can be identified. Prior knowledge of component geometry and material properties are neccessary in order to generate the dispersion curves required to identify suitable modes and frequencies of operation, as well as to provide baselines for comparison.

Cooling hole structures have a significant impact on wave propagation, however careful selection of wave mode based on wavelength allows wave packet propagation across the structure with limited scattering, and temperature monitoring is still achievable. Using reflections from these structures to monitor temperature at multiple locations is unlikely to be achievable in areas with dense arrays of holes (across the leading edge, for example), however it may be possible to implement a system in the less geometrically complex regions towards the trailing edge of the vane.

The addition of thermal barrier coatings has a significant impact on wave propagation, causing large variations in through-thickness displacement. It has been shown that the generation of dispersion curves for multi-layered composites such as these is still an effective method of predicting acoustic response, which has been confirmed through simulation. Above the lowest two modes the curves are more complex with a TBC applied than without, which makes the use of these modes less viable.


\vfill
\hspace{0pt}