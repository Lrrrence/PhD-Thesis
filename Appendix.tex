\chapter{Appendices}\label{appendix}
\section{Appendix 1 - MATLAB Code}
\subsection{Time of flight measurement}

\begin{lstlisting}[style=Matlab-editor,basicstyle=\small]

clc;
clf;
warning('OFF', 'MATLAB:table:ModifiedAndSavedVarnames')

set(figure(1), 'Renderer', 'painters');
set(0,'DefaultLineLineWidth',2)

myfile = uigetfile('*.csv','MultiSelect','on');
M1 = readtable(myfile);
A1 = table2array(M1);
A1(1,:) = [];

Time = A1(:,1);
input = A1(:,2);
output = A1(:,3);

filterbw = [1e6 2e6];
filterlengthin = 300;
filterlengthout = 10;

Fs = 1/((Time(2,1))-(Time(1,1)));

% reduce signal length
n=200e-6;
[val,idx]=min(abs(Time-n));

Time = Time(1:idx);
input = input(1:idx);
output = output(1:idx);

%low pass filter
bpinput = bandpass(input, filterbw, Fs,'ImpulseResponse','iir','Steepness',0.8);

%low pass filter
bpoutput = bandpass(output, filterbw, Fs,'ImpulseResponse','iir','Steepness',0.8);

%calc max amplitudes
inmax = max(input);
outmax =max(bpoutput);

%create envelope of Input
[upin,loin] = envelope(input,filterlengthin,'analytic');

%find peak of envelope Input
[pkkin,locsin] = findpeaks(upin,Fs,'MinPeakProminence',0.0001);
peaklocin = locsin(1,1)+min(Time);

%create envelope of Output
[upout,loout] = envelope(bpoutput,filterlengthout,'peak');

minpeak = max(bpoutput)-(0.1*max(bpoutput));

%find peak of envelope Output
[pkkout,locsout] = findpeaks(upout,Fs,'MinPeakHeight',minpeak);
peaklocout = locsout+min(Time);

%calc ToF
ToF = (peaklocout(1,1)-peaklocin);
disp(ToF)

%calculate fft
Fs = 1/((Time(2,1))-(Time(1,1)));
St = 1/Fs;     % Sampling period
                   
L = length(bpoutput);   % Length of signal
t = (0:L-1)*St;        % Time vector

Y1 = fft(input);
Y2 = fft(bpoutput);

PI2 = abs(Y1/L);
PI1 = PI2(1:L/2+1);
PI1(2:end-1) = 2*PI1(2:end-1);

PO2 = abs(Y2/L);
PO1 = PO2(1:L/2+1);
PO1(2:end-1) = 2*PO1(2:end-1);

f = Fs*(0:(L/2))/L;

%plot FFT
figure(1)

tiledlayout(2,1)
ax1=nexttile;

plot(f,PI1) 
title('Single-Sided Amplitude Spectrum of Excitation Pulse')
xlabel('f (Hz)')
xlim([0 3e6])
ylabel('|P1(f)|')
grid on
set(gca,'LineWidth',1.2) %for grid lines
ax = gca;
ax.GridColor = 'k';

set(gca,'fontname','Libertinus Sans','fontsize', 20)
 
ax2=nexttile;
plot(f,PO1) 
title('Single-Sided Amplitude Spectrum of Output')
xlabel('f (Hz)')
xlim([0 3e6])
ylabel('|P1(f)|')
grid on
set(gca,'LineWidth',1.2) %for grid lines
ax = gca;
ax.GridColor = 'k';

x0=10;
y0=10;
width=1600;
height=1000;
set(gcf,'position',[x0,y0,width,height])
set(gca,'fontname','Libertinus Sans','fontsize', 20)
set(gcf, 'Color', 'w');

%plot signal propagation

set(groot,{'DefaultAxesXColor','DefaultAxesYColor','DefaultAxesZColor'},{'k','k','k'})

figure(2)
hold on
grid on
set(gca,'LineWidth',1.2) %for grid lines
ax = gca;
ax.GridColor = 'k';
%title('Simulated Signals')

yyaxis left
plot(Time*1e6,input,'LineWidth',1.5)
plot(Time*1e6,upin,'-','LineWidth',4.0)
%ylim([-10 10])
%yticks(-10:2:10)
ylabel('Amplitude (V)')

xlim([0 max(Time*1e6)])
xticks(0:10:max(Time*1e6))

plot([peaklocin*1e6 peaklocin*1e6],[0 max(upin)],'b:','LineWidth',5.0)

yyaxis right
hold on
plot(Time*1e6,bpoutput*1e3,'LineWidth',1.5)
plot(Time*1e6,upout*1e3,'-','LineWidth',4.0)
%ylim([-5 5])

plot([peaklocout(1,1)*1e6 peaklocout(1,1)*1e6],[0 max(upout)*1e3],'b:','LineWidth',5.0)

ylabel('Amplitude (mV)')
set(gca,'fontname','Libertinus Sans','fontsize', 20)
xlabel('Time (μs)')
hold off

x0=10;
y0=10;
width=1600;
height=1000;
set(gcf,'position',[x0,y0,width,height])
set(gca,'fontname','Libertinus Sans','fontsize', 20)

%export_fig S1fullresult.eps -painters -nocrop
\end{lstlisting} 

\subsection{2D-FFT}

\begin{lstlisting}[style=Matlab-editor,basicstyle=\small]

set(0,'DefaultLineLineWidth',1)
set(groot,{'DefaultAxesXColor','DefaultAxesYColor','DefaultAxesZColor'},{'k','k','k'})

close all
myfile = uigetfile('*.csv','MultiSelect','on');
%myfile = '1MHz_S1_fullwedge_y.csv';
M1 = readtable(myfile,'ReadVariableNames', false);
A1 = table2array(M1);
A1(1,:) = [];

A1 = A1(8:end,:);

Fs = 1/((A1(2,1))-(A1(1,1))); %sample rate
St = 1/Fs;     % Sampling period

filterbw = [2e6 5e6]; %filter bandwidth
%A1 = bandpass(A1, filterbw, Fs,'ImpulseResponse','iir'); % filter 

A1 = A1(:,2:end); % remove time column

A1 = padarray(A1,[100000 2000], 0,'post'); % zero pad

Nx = size(A1,2); % Number of samples collected along first dimension
Nt = size(A1,1); % Number of samples collected along second dimension
dx = 0.0008;  % Distance increment (i.e., Spacing between each column)
dt = St; % Time increment (i.e., Spacing between each row)
Nyq_k = 1/dx; % Nyquist of data in first dimension
Nyq_f = 1/dt; % Nyquist of data in second dimension
dk = 1/(Nx*dx);   % Wavenumber increment
df = 1/(Nt*dt);   % Frequency increment
k = -Nyq_k/2:dk:Nyq_k/2-dk; % wavenumber (m)
kr = k*(2*pi)/1000; % wavenumber (rad/mm)
f =-Nyq_f/2:df:Nyq_f/2-df; % frequency (Hz)

krhalf = kr(:,((size(kr,2))+1)/2:end); % take positive quadrant of kr
fhalf = f(:,((size(f,2))+1)/2:end); % take positive quadrant of f

fft2result = fftshift(fft2(A1))*dx*dt; %fft2
fft2resultB =fft2result.'; %transpose
fft2resultB = flip(fft2resultB,2); %flip

z = abs(fft2resultB); %absolute value

fft2data = z((size(z,1)+1)/2:end,(size(z,2)+1)/2:end); % take positive quadrant of z

%fft2data = normalize(fft2data,'scale');
znorm = rescale(fft2data,0,1); % normalise 0-1

figure(1);

imagesc(fhalf/1e6,krhalf,znorm);
set(gca,'YDir','normal')
colorbar;
%colormap(brewermap([],'Spectral'))
colormap(flipud(colormap('viridis_white')))

xlim([0 1.6])
ylim([0 4])

ylabel('Wavenumber (rad/mm)')
xlabel('Frequency (MHz)')
hold on
%grid on
box on
ax = gca;
ax.LineWidth = 2;

x0=0;
y0=0;
width=1200;
height=1200;
set(gcf,'position',[x0,y0,width,height])
set(gca,'fontname','Libertinus Sans','fontsize', 25)
set(gcf, 'Color', 'w');

%export_fig 2DFFT-B1-0.8MHz-Ydisp.eps -nocrop

\end{lstlisting}
    