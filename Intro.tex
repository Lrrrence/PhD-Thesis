\chapter{Introduction}
Nozzle guide vanes (NGVs) are static components found in the turbine section of jet engines (Figure~\ref{fig:crossngv}). Temperature monitoring of these components is important for a number of reasons: identifying potential failures before they occur, evaluating the need for maintenance, investigating ways of improving engine efficiency, and reducing fuel consumption. Online temperature monitoring of NGVs is difficult to achieve with currently available technologies (see Chapter~\ref{literev}) due to the harsh conditions of the turbine, as well as space, access, and power constraints. Ultrasonic guided waves are used for a wide range of structural health monitoring applications, and can often offer a number of benefits over traditional alternatives. In many implementations of guided wave based sensors the effect of temperature on their operation and response is compensated for, however the relationship between wave velocity and temperature can be used for temperature monitoring applications. This thesis investigates a number of aspects of guided wave based temperature monitoring that need to be considered for implementing a system on a nozzle guide vane. The complex wave propagation of guided waves is investigated, and the suitability of a range of wave modes are considered for temperature monitoring. The signal processing requirements for accurately monitoring changes in time of flight are explored, taking into account the dispersive nature of guided waves. Elements such as temperature sensitivity, temperature range, and response time are considered, as well the potential error associated with such a system. The geometry of nozzle guide vanes is also considered, as features such as cooling hole arrays and thermal barrier coatings will have substantial effects on wave propagation, and therefore the ability to monitor changes in temperature. 

\clearpage

The study can be classified into five sections:
\begin{itemize}
    \item \cref{literev} -- A literature review of the currently available temperature monitoring methods for nozzle guide vanes, both offline and online.
    \item \cref{ultrasonicmonitoring} -- An investigation into the potential for guided wave based temperature monitoring systems.
    \item \cref{chapter:tempsens} -- The temperature sensitivity of guided waves.
    \item \cref{chapter:holes} -- The effect of cooling holes on wave propagation.
    \item \cref{chapter:TBC} -- The effect of thermal barrier coatings on wave propagation
\end{itemize}

The literature review considers the currently available options for the temperature monitoring of nozzle guide vanes, including both offline and online systems. The outcome of the review is in identifying the benefits of online monitoring systems, in terms how they can impact reliability, safety, and costs. 

The potential of an ultrasonic based temperature monitoring system is considered in the following chapter, based on the usage of ultrasound for other applications, and the limited usage of ultrasound for temperature sensing. A number of sensor options are considered, and their suitability for operation at high temperatures is evaluated. Guided waves are a subset of ultrasound that are considered for this application, due to the geometry of the structure. Dispersion curves and the multi-modal nature of Lamb wave are discussed. The effect of temperature on different material properties is considered, and the impact on wave propagation is discussed.

The next chapter considers the temperature sensitivity of guided waves through prediction, experimentation, and simulation. Theoretical evaluation of the temperature sensitivity of Lamb wave modes highlights the potential of the method. The sensitivities of the $S_0$, $A_1$, and $S_1$ modes for a 1 mm think Aluminium plate have been extracted from dispersion curves generated from material properties. The prediction indicates that group velocity will reduce with increasing temperature. An experimental test system has been developed to validate the theoretical predictions. Two variable angle wedge transducers have been used to target the modes of interest, and a measurement of time of flight between transducers has been used to calculate the group velocity. Results are in good agreement with theoretical predictions, showing that a time of flight based measurement system is capable of monitoring a change in temperature effectively. A two-dimensional finite element model (COMSOL) has been developed and validated against experimental results, to investigate the effect of other factors on wave propagation. The effect of extending the temperature range of a monitoring system up to that of a gas turbine ($\sim$1000\si{\degreeCelsius}) in a Nickel-based super alloy (Inconel 718) is investigated through theoretical prediction and 2D-FE simulation.  

In the next chapter, the effect of cooling holes on wave propagation and the extent to which the reflections from these holes can be used to monitor temperature at different locations is investigated both experimentally and through 3D-FE simulation. 

In the next chapter, the effect of thermal barrier coatings (TBCs) on wave propagation is investigated through theoretical prediction and 2D-FE simulation, considering variances in material properties, layer thicknesses, and application techniques. 

The results of these studies is summarised in the conclusion chapter, and the most suitable method of implementing a guided wave based temperature monitoring system is discussed. The range of further research necessary in this field is also discussed, which includes the effect of curved geometry on wave propagation, and determining the most suitable transducer configuration that can operate in the harsh conditions of a turbine. 

\newpage

\vspace*{\fill}

\begin{figure}[!htbp]
    \centering
    \includegraphics[width=\textwidth]{./figures/turbine plus ngv.png}
    \caption{Jet engine cross-section with NGV.}\label{fig:crossngv}
\end{figure}

\vspace{\fill}

\newpage

\section{Research objectives}

The overarching research objective is to determine if a guided wave based temperature monitoring system is suitable for use on gas turbine nozzle guide vanes. Ideally a system will be capable of monitoring temperature with a resolution, accuracy, and response time, comparable with traditional temperature monitoring systems. With this in mind there are a number of points to be considered, including sensor configuration, wave propagation, environmental impact, and a signal processing. 

One key objective is to understand the effects of the physical environment on wave propagation in an NGV-like structure. This includes evaluating the effects of the propagation medium (material properties, curved surfaces, cooling holes, and surface coatings), and environmental conditions (temperature, acoustic noise, and gas flow) on wave propagation. The conditions listed previously will have differing effects on different Lamb wave modes, which leads to an additional objective of identifying the most appropriate mode (or mode group) for temperature monitoring in this environment. In order to evaluate these factors a test system will be developed that can transmit/receive Lamb waves in an NGV-like structure, which will also allow investigation into the signal processing requirements of a real system. 

Another objective is to identify a suitable sensor configuration that can survive in high temperatures, and operate under the restrictions of the environment. This includes low power operation, strict space constraints, and the ability to last for the lifetime of turbine with minimal servicing. Additionally, the type of sensor and the interface between the sensor and the NGV structure will impact the types of waves that can be generated, which further complicates the selection of suitable sensor configurations.

A requirement for the monitoring of NGVs is the ability to monitor temperature spatially, in order to identify hot spots or likely points of failure. An objective of this study is to determine if the monitoring of temperature at multiple locations is possible, through the acoustic reflections created by wave interaction with cooling holes.

These objectives raise the following research questions:

\subsection{Research questions}
\begin{enumerate}
    \item How can the temperature dependence of Lamb waves be utilised for temperature monitoring?\label{itm:1}
    \item How can the theoretical temperature sensitivity of individual Lamb wave modes be verified experimentally?\label{itm:2}
    \item What is the effect of the physical environment on Lamb wave propagation?\label{itm:3}
\begin{itemize}
    \item High temperatures
    \item Holes
    \item Curved surfaces
\end{itemize}
    \item Which of the Lamb wave modes (or group of modes) is most appropriate for temperature monitoring of NGVs?\label{itm:4}
    \item To what extent can acoustic reflections from cooling holes be used to monitor temperature at a number of locations across the structure of an NGV?\label{itm:5}
    \item What is the most suitable transducer configuration for exciting Lamb waves in NGVs?\label{itm:6}
\end{enumerate}

\clearpage

Upon completion of the PhD the research questions listed above will have been addressed, which will lead to a conclusion on how a guided wave based temperature monitoring system can be implemented on a nozzle guide vane. The effect of the physical environment on wave propagation will be known, and the most suitable modal region of temperature monitoring will have been identified. The extent to which the wave interaction with cooling holes can be used to monitor temperature at a number of locations will be known. A measurement system that can theoretically operate at high temperatures will have been identified. The outcome of these studies can form the basis of future investigations into the design of a measurement system suitable for use at high temperatures, and the testing of a system on an installed NGV. 

\section{Research output}
\begin{itemize}
    \item Student poster ``Temperature monitoring of nozzle guide vanes (NGVs) using ultrasonic guided waves''~\cite{Yule2020} presented at \href{https://event.asme.org/Turbo-Expo}{ASME Turbo Expo 2020.} Virtual conference 21--25 September 2020.
    \item Journal paper ``Surface temperature condition monitoring methods for aerospace turbomachinery: exploring the use of ultrasonic guided waves''~\cite{Yule2021} published to \href{https://iopscience.iop.org/article/10.1088/1361-6501/abda96}{IOP Measurement, Science, and Technology.}
    \item Conference paper ``Towards in-flight temperature monitoring for nozzle guide vanes using ultrasonic guided waves''~\cite{Yule2021a} published to \href{https://www.aiaa.org/propulsionenergy}{AIAA Propulsion Energy}, 9--11 August 2021.
    \item Journal paper ``Modelling and Validation of a Guided Acoustic Wave Temperature Monitoring System'' \cite{Yule2021b} published to \href{https://www.mdpi.com/1424-8220/21/21/7390}{MDPI Sensors Special Issue ``Sensors for Severe Environments''}.
\end{itemize}

\printbibliography[title={Chapter~\thechapter~Bibliography}]
