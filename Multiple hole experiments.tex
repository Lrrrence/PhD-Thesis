\chapter{Experimental investigation of multiple reflections}

In this chapter the ability to monitor temperature from the reflections of multiple holes is explored. When considering the application of this technology on a nozzle guide vane, test parameters have been determined by typical geometry features. 

\begin{itemize}
    \item Holes can vary in size/shape from the internal surface to the external surface~\cite{Kiyici2016}.
    \item They can be angled to direct air flow to a certain area.
    \item They can have different diameters along the leading edge.
    \item They are often arranged in rows, with each adjacent row offset by the hole spacing alternately.
    \item Hole diameters typically range from 0.3-1.5 mm. 
    \item The spacing between holes is around 2-3 mm~\cite{Clum2014}.
\end{itemize}

As it may be possible to place transducers on both sides on the vane, both through-hole transmission (on-axis transducers, pitch-catch) and a quasi-pulse-echo method are investigated. 

\section{Pitch-catch experiments}

Wedges are placed 10 cm apart, on-axis. A baseline signal with no holes present is captured first, for comparison. 1.5 mm holes are drilled in the test plate equidistant from the wedges. Increasing from 1 hole, to 3, to 5, directly in-line, spaced 2.5 mm apart (measured from hole centre). 

Amplitude is reduced with increasing numbers of holes. There is a negligible effect on signal shape or time of flight. This suggests that although this transducer configuration is not suitable for measuring/monitoring temperature at a number of locations across the substrate, it can still be used to monitor average temperature of the whole substrate.  

\section{Pulse-echo experiments}

