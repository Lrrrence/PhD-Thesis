\chapter{Experimental investigation of Lamb wave temperature sensitivity}\label{experiments}

The wave velocity temperature sensitivity of the $S_0$, $A_1$, and $S_1$ Lamb wave modes have been experimentally measured, to be validated against theoretical predictions in Chapter~\ref{theory}.

Two 1 MHz piezoelectric transducers attached to acrylic wedges (Olympus variable angle wedge) in a pitch-catch configuration have been coupled to a 1~mm thick aluminium plate with a liquid couplant (Figure~\ref{fig:testsetup}). A signal generator (GW Instek MFG-2203M) has been used to generate a 5-cycle Hamming windowed tone burst at 1~MHz. Signals are digitised using a Picoscope 3406DMSO USB Oscilloscope. Based on a sampling rate of 5$\times$10$^8$ the theoretical maximum temporal resolution is 2~ns. Signal processing is carried out in MATLAB. A zero-phase bandpass filter is applied to the signals to remove unwanted noise. Time of flight ($t_F$) is measured between transducers and wave velocity is calculated from the distance between transducers. The temperature of the aluminium plate is controlled using a hot plate.

\begin{table}[b]
    \centering
    \begin{tabulary}{\textwidth}{L}
        \hline
        \textbf{Measurement Hardware}     \\
        \hline
        2x Olympus ABWX-2001 Variable angle wedges \\
        2x Olympus A539S-SM 1 MHz transducers \\
        Olympus ultrasonic couplant B \\
        GW Instek MFG-2203M Signal generator \\
        Picoscope 3406DMSO USB Oscilloscope \\
        Thermadata T-type temperature loggers \\
        VWR Hot plate \\
        \hline
    \end{tabulary} 
    \caption{Experimental measurement hardware.}\label{table:hardware}
\end{table}

\newpage

\begin{figure}[h]
    \centering
    \includegraphics[width=.9\textwidth]{./figures/testdiagramsimple.eps}
    \caption{Cross-sectional diagram of total time-of-flight measurement setup.}\label{fig:testdiagramtotal}
\end{figure}

\begin{figure}[h]
    \centering
    \includegraphics[width=.7\textwidth]{./figures/w2wdiagram.eps}
    \caption{Cross-sectional diagram of wedge-to-wedge time-of-flight measurement setup.}\label{fig:testdiagramw2w}
\end{figure}

Wave velocity is calculated using Equation~\ref{velocitycalc}/\ref{velocitycalcfull}. The propagation time through the wedges (measured using the configuration shown in Figure~\ref{fig:testdiagramw2w}) has been subtracted from the total $t_F$ to ensure that only the propagation time through the plate is measured.  
%
\begin{equation} \label{velocitycalc}
v = \frac{d}{t_F}
\end{equation} 
%
\begin{equation} \label{velocitycalcfull}
v = \left( \frac{d\;\text{between wedges} + d\;\text{wedge foot offset}}{\text{Total}\;t_F - \text{Wedge-to-wedge}\;t_F} \right)
\end{equation} 
%
\\
Where the $d$ wedge foot offset is an unknown distance from the front edge of the wedge to where the wave enters the plate from the wedge. This distance has been calculated by measuring wave velocity at room temperature for the mode of interest at multiple wedge spacings (0.08 m to 0.14 m in 0.01 m increments) and looping through a range of plausible offset distances until the standard deviation across the range of wedge spacings is at a minimum. This ensures that the variation in measurement results is due to measurement error (e.g. small variances in setting the distance between wedges) rather than an incorrect estimation of wedge foot offset. This value varies with wedge angle and is recalculated for each wave mode measured.

The wedges allow for careful selection of excitation angle so that modes of interest can be targeted. The angle is determined based on Snell’s law:
%
\begin{equation} 
\text{Angle}\ \theta = \text{Sin}^{- 1} \left( \frac{\text{Longitudinal\ wedge\ velocity}}{\text{Lamb\ wave\ phase\ velocity}} \right)
\end{equation} 
%
Measurement of wave velocity depends on measurement of time of flight ($t_F$), which can be described by the Equation~\cite{Croxford2007a}:
%
\begin{equation}\label{tofcalc}
t_F = \frac{d}{c}
\end{equation} 
%
Where $d$ is the distance travelled at wave speed $c$, both of which are functions of temperature, $T$. The sensitivity of the time of flight to temperature can then be expressed as:
%
\begin{equation} 
\delta t_{F} = \frac{d}{c}\left( \alpha - \frac{k}{c} \right) \delta \text{T}
\end{equation} 
%
Where $\alpha$ is the coefficient of thermal expansion of the medium and $k$ is the rate of change of wave velocity with temperature:
%
\begin{equation} {\label{eqn: eq k}}
k = \frac{\delta \text{c}}{\delta \text{T}}
\end{equation} 
%
\newpage
\section{Test Method}

A hot plate is used to raise the temperature of the aluminium plate to the desired temperature. The temperature of the aluminium plate is monitored using a thermocouple placed in the centre of the plate at the hottest point. The total $t_F$ is measured until it stabilises using the test setup shown in Figure~\ref{fig:testdiagramtotal}. The temperature of the entire system must be allowed to stabilise before taking the measurement to ensure that the temperature of the wedge is the same as the plate. Total $t_F$ is now measured for the set temperature. Multiple measurements are taken after adjusting both wedge positions. The wedges are removed from the surface and placed together to measure the wedge-to-wedge $t_F$ as shown in Figure~\ref{fig:testdiagramw2w}. Multiple measurements are taken after adjusting wedge-to-wedge position. The $t_F$ measurement process is repeated after allowing the total $t_F$ to re-stabilise. Velocity is calculated using Equation~\ref{velocitycalcfull}. A mean average is calculated from the results of the repeated total $t_F$ measurements, and velocity is calculated for every wedge-to-wedge result. An average velocity is calculated along with standard deviation. 

The temperature gradient across the plate has been measured by placing four equally spaced thermocouples along the transmission path, from the centre of the plate to the furthest edge of a wedge transducer in 3 cm increments. The wedges are removed from the plate to place the thermocouples, and the temperature of the hot plate is raised to match the temperature recorded by the thermocouple placed in the centre of the plate during measurement of total $t_F$. Measurements are repeated after moving the thermocouples to the other half of the transmission path. A mean average temperature has been calculated for the total transmission path at each hot plate temperature setting. 

\begin{figure}[ht]
    \centering
    \includegraphics[width=.8\textwidth]{./figures/hjkhh7UmEx.png}
    \caption{Photograph of test setup.}\label{fig:testsetup}
\end{figure}

\clearpage
\section{S0 mode (1 MHz-mm)}\label{S0 experiments}
%
The wedge angle required for the $S_0$ mode is:
%
\begin{equation} 
31{^\circ} = \text{Sin}^{- 1} \left( \frac{2720}{5258} \right)
\end{equation} 
%
The $A_0$ mode cannot be excited using this method as the phase velocity at this frequency (2312 m s$^{-1}$) is slower than the longitudinal velocity of the wedge. If the $A_0$ mode is present in the signal it will not affect measurement of the $S_0$ mode as it’s group velocity is significantly different than that of the $S_0$ mode, which will cause a distinct second wave packet. 

\begin{figure}[h]
    \centering
    \includegraphics[width=.8\textwidth]{./figures/s0xcorr5cyc.eps}
    \caption{Time of flight ($t_F$) measurement of $S_0$ mode wave propagation using cross-correlation function at room temperature.}\label{fig:S0crosscorr}
\end{figure}

\subsection{S0 mode results}

\begin{figure}[h]
    \centering
    \includegraphics[width=.8\textwidth]{./figures/aluplatemeasured.eps}
    \caption{Group velocity change with temperature for the $S_0$ mode in Aluminium 1050 H14.}\label{fig:result}
\end{figure}

Figure~\ref{fig:result} shows experimentally measured wave velocity of the $S_0$ mode plotted against theoretical wave velocity extracted from dispersion curves. Error bars show the standard deviation from the mean. After accounting for the temperature gradient across the transmission path by calculating a temperature average the change in velocity is comparable with predicted velocity extracted from dispersion curves, within 4.89 $\pm$ 2.27 m s$^{-1}$ on average. The temperature sensitivity of the system is 1.26--1.78 m s$^{-1}$\si{\degreeCelsius}$^{-1}$ over the range 24\si{\degreeCelsius}--94\si{\degreeCelsius}. The sensitivity is extracted from a second-order polynomial fit of the data (r$^2$ = 0.9992). The slope away from predicted results (increasing with temperature) can be attributed to the increasing temperature gradient, both in the plate and in the wedges. The gradient is shown to be almost linear (r$^2$=0.9967) across the measurement distance. Increasing temperature is also likely to have an effect on the operation of the piezoelectric transducer (amplitude and centre frequency), however this effect is negligible over the tested temperature range. The wedge angle required to excite the $S_0$ mode will also vary with temperature, however the change is only around 1\si{\degree} between 20\si{\degreeCelsius} and 100\si{\degreeCelsius}.
\newpage
\section{A1 mode (2.5 MHz-mm)}

The wedge angle required for the $A_1$ mode is:
%
\begin{equation} 
24{^\circ} = \text{Sin}^{- 1} \left( \frac{2720}{6654} \right)
\end{equation} 
%
\begin{figure}[h]
    \centering
    \includegraphics[width=.8\textwidth]{./figures/a1xcorr5cyc.eps}
    \caption{Time of flight ($t_F$) measurement of $A_1$ mode wave propagation using cross-correlation function at room temperature.}\label{fig:A1crosscorr}
\end{figure}

\subsection{A1 mode results}

Figure~\ref{fig:A1result} shows experimentally measured wave velocity of the $A_1$ mode plotted against theoretical wave velocity extracted from dispersion curves. Error bars show the standard deviation from the mean. After accounting for the temperature gradient across the transmission path by calculating a temperature average the change in velocity is comparable with predicted velocity extracted from dispersion curves, within 2.43 $\pm$ 1.97 m s$^{-1}$ on average. The temperature sensitivity of the system is 1.09--1.17 m s$^{-1}$\si{\degreeCelsius}$^{-1}$ over the range 26\si{\degreeCelsius}--97\si{\degreeCelsius}. The sensitivity is extracted from a second-order polynomial fit of the data (r$^2$ = 0.9990).

\begin{figure}[h!]
    \centering
    \includegraphics[width=.8\textwidth]{./figures/a1moderesult.eps}
    \caption{Group velocity change with temperature for the $A_1$ mode in Aluminium 1050 H14.}\label{fig:A1result}
\end{figure}
\newpage
\section{S1 mode (4 MHz-mm)}

This region of frequency-thickness product is multi-modal, with both the $A_1$ and $S_1$ modes present. Similarities in phase velocity leads to similar excitation angles, which causes both modes to be excited. Using a cross-correlation method for measuring time-of-flight is no longer appropriate, as the received signal differs substantially from the input signal. An envelope peak method is employed instead, whereby analytic envelopes for both the excitation signal and received signal are generated by Hilbert FIR filtering, with a filter length of 2000 samples. This produces a smooth envelope with clearly defined peaks, as seen in Figure~\ref{fig:S1timedomain}. A peak finding algorithm is used to detect the envelope peaks, as denoted by the dashed lines. The $S_1$ mode ($\sim$~4550 m s$^{-1}$) arrives at the receiver before the $A_1$ mode ($\sim$~2550 m s$^{-1}$) as it has a considerably higher group velocity. At this propagation distance the two modes are clearly separated in the time domain, with the $A_1$ mode showing considerably more dispersion.  

The wedge angle required for the $S_1$ mode is:
%
\begin{equation} 
28{^\circ} = \text{Sin}^{- 1} \left( \frac{2720}{5874} \right)
\end{equation} 
%
\subsection{S1 mode results}

Figure~\ref{fig:S1result} shows experimentally measured wave velocity of the $S_1$ mode plotted against theoretical wave velocity extracted from dispersion curves. Error bars show the standard deviation from the mean. After accounting for the temperature gradient across the transmission path by calculating a temperature average the change in velocity is comparable with predicted velocity extracted from dispersion curves, within 4.44 $\pm$ 7.15 m s$^{-1}$ on average. The temperature sensitivity of the system is 1.80 m s$^{-1}$\si{\degreeCelsius}$^{-1}$ over the range 25\si{\degreeCelsius}--103\si{\degreeCelsius}. The sensitivity is extracted from a linear fit of the data (r$^2$ = 0.9777).

\begin{figure}[h!]
    \centering
    \includegraphics[width=.8\textwidth]{./figures/s1hilbert.eps}
    \caption{Time of flight ($t_F$) measurement of $S_1$ mode using envelope peak method at room temperature.}\label{fig:S1timedomain}
\end{figure}

\begin{figure}[h!]
    \centering
    \includegraphics[width=.8\textwidth]{./figures/s1moderesult.eps}
    \caption{Group velocity change with temperature for the $S_1$ mode in Aluminium 1050 H14.}\label{fig:S1result}
\end{figure}

\newpage
\section{Experimental sensitivity analysis}

There are a number of experimental error sources to consider. The physical distance between wedges is controlled using 3D printed spacers that keep the wedges aligned at set distances. The movement of the wedges on the surface of the plate increases with temperature as the viscosity of the couplant decreases. Variations in placement cause the calculated velocity to vary by around $\pm$ 5 m s$^{-1}$ across multiple (30) wedge re-alignments. The measurement of wedge-to-wedge time to be subtracted from the total $t_F$ is temperature dependant and relies on accurate alignment of the wedge feet, as well as a good connection between them (signal amplitude is highly dependant on couplant). Variation in alignment causes around a $\pm$ 10 m s$^{-1}$ velocity change. The wedge foot offset in Equation~\ref{velocitycalcfull} has a large effect on the calculated wave velocity. The exact offset distance is unknown and is assumed to be the point at which the centre line of the transducer aligns with the plate surface. Varying this value raises or lowers the velocity of all results considerably ($\pm$ 1 mm = $\pm$ 35 m s$^{-1}$). The hot plate does not heat the test plate evenly, especially at distances greater than 10 cm between wedges where they overhang the edges of the hot plate. The gradient (the difference in temperature between the centre of the plate and the location of the wedges) increases with temperature. The measured velocity is monitoring the average temperature of the transmission path. The gradient has been measured by placing a number (4) of thermocouples along the transmission path, from the centre of the plate (maximum temperature) to the point at which a wave is transmitted between a wedge foot and the plate. The calculation of aluminium dispersion curves at different temperatures is based on a change in Young's modulus. This is predicted from Hopkin's formula \cite{Hopkins2012} that may not give the correct values for Aluminium 1050 H14, but Aluminium in general.

The largest source of error is the measurement of wedge-to-wedge $t_F$, as a small error in alignment causes a large change ($\pm$ 10 m s$^{-1}$) in the wave velocity calculation. This is accounted for through the averaging of multiple (30) measurements, the standard deviation for this range is shown using error bars on Figure~\ref{fig:result}. Variation in wedge foot offset distance dramatically shifts the calculated velocity. This value cannot be directly measured and so the result relies on accurate calculation as discussed in the method.

\section{Conclusion}

The theoretical effect of temperature on various Lamb wave modes in aluminium plates has been investigated by generating dispersion curves based on varying material properties (Figure~\ref{fig:alutempshift}). This can be repeated in the future for other materials at higher temperatures (e.g.\ Inconel 718 up to 1100\si{\degreeCelsius} in Figure~\ref{fig:groupshift}). The temperature sensitivity of the $S_0$ mode at 1~MHz has been extracted from these curves (Figure~\ref{fig:alugroupvel}) to be validated experimentally.

An experimental investigation has been carried out in order to validate theoretical predictions. Wedge transducers in a pitch-catch configuration have been used to excite the $S_0$ mode in a 1mm thick aluminium plate. The time of flight between transducers has been measured using a cross-correlation method and wave velocity calculated based on the distance between transducers. This confirms that the $S_0$ mode has been excited. The change in $S_0$ wave velocity due to temperature is in line with theoretical predictions over the range 20\si{\degreeCelsius}--100\si{\degreeCelsius} as shown in Figure~\ref{fig:result}. 

It is clear that wedge transducers are not the optimum method of transmitting/receiving a wave through a nozzle guide vane at high temperatures. They cannot be permanently mounted to the structure due to the need for a liquid couplant, and their relatively large footprint would make finding a suitable mounting location a challenge. Their operation at high temperatures is limited by the wedge material, which in the case of acrylic melts at around 160\si{\degreeCelsius}. The wedge material needs to have a longitudinal wave velocity less than that of the targeted Lamb wave phase velocity, which limits the choice of material severely, mostly to plastics with low melting points. The great benefit of wedge transducers is the ability to selectively target Lamb wave modes, which reduces the complexity of data analysis compared with exciting multiple modes simultaneously. This is difficult to achieve using other transducer configurations but it may instead be possible to excite a higher order region that travels as a single wave packet. The use of PWAS transducers could allow for operation at high temperatures (assuming suitable choice of piezoelectric material) and would be relatively easy to mount to an NGV structure, having a small footprint, although the high temperatures are likely to make bonding difficult. Another option is to couple into the structure using waveguides, distancing the transducers from the high temperature environment. Future research will investigate the temperature sensitivity of higher order modes (such as $A_1$ and $S_1$), as operating at higher frequencies can improve resolution (allowing for the detection of smaller phase shifts) and response rates. The ability to monitor wave velocity variations in multi-modal wave packets will also be considered when investigating transducer configurations suitable for higher temperature operation.

\printbibliography